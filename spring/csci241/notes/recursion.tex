\documentclass{report}

\input{~/latex/template/preamble.tex}
\input{~/latex/template/macros.tex}

\title{\Huge{Recursion}}
\author{\huge{Matt Warner}}
\date{\huge{}}
\pagestyle{fancy}
\fancyhf{}
\rhead{}
\lhead{\leftmark}
\cfoot{\thepage}
% \usepackage[default]{sourcecodepro} \usepackage[T1]{fontenc}

\pgfpagesdeclarelayout{boxed}
{
  \edef\pgfpageoptionborder{0pt}
}
{
  \pgfpagesphysicalpageoptions
  {%
    logical pages=1,%
  }
  \pgfpageslogicalpageoptions{1}
  {
    border code=\pgfsetlinewidth{1.5pt}\pgfstroke,%
    border shrink=\pgfpageoptionborder,%
    resized width=.95\pgfphysicalwidth,%
    resized height=.95\pgfphysicalheight,%
    center=\pgfpoint{.5\pgfphysicalwidth}{.5\pgfphysicalheight}%
  }%
}

\pgfpagesuselayout{boxed}

\begin{document}
  \maketitle
\section{Overview}
Recursion is a general programming technique used to solve problems with a ``divide and conquer'' strategy
\bigbreak \noindent
\nt{
  most computer programming langauges support recursion by allowing a function or member function to call itself within the program text.
}
\bigbreak \noindent
\textbf{Example of recursion}
\begin{mdframed}
\begin{minted}{c++}
int factorial(int n)
{
  if (n == 1)
  {
  return 1 ;
  }
  else
  {
  return n * factorial(n-1);
  }
}
\end{minted}
\end{mdframed}
\bigbreak \noindent
A recursive function call will always by conditional. There must tbe at least one base case for which the function produces a result trivially without a recursive call. For example:
\begin{mdframed}
\begin{minted}{c++}
int factorial(int n){
  if (n == 1 ){   // Base case - no recursion
    return 1;
  }
  else
  {
    return n * factorial(n-1);
  }
}
\end{minted}
\end{mdframed}
\bigbreak \noindent
A function withno base cases leads to ``infinite recursion'' (similar to an infinite loop)
\bigbreak \noindent
In addition to the base cases, a recursive function will have one or more recursive cases. The job of a recursive case can be seen as breaking down complex inputs into simpler ones.
\bigbreak \noindent
In a properly designed recursive function, with each recursive call, the input problem must be simplified in such a way that eventually the base case must be reached. For example
\end{document}

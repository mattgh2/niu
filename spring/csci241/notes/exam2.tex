\documentclass{report}

\input{~/latex/template/preamble.tex}
\input{~/latex/template/macros.tex}

\title{\Huge{Exam 2 Review}}
\author{\huge{Matt Warner}}
\date{\huge{}}
\pagestyle{fancy}
\fancyhf{}
\rhead{}
\lhead{\leftmark}
\cfoot{\thepage}
% \usepackage[default]{sourcecodepro} \usepackage[T1]{fontenc}

\pgfpagesdeclarelayout{boxed}
{
  \edef\pgfpageoptionborder{0pt}
}
{
  \pgfpagesphysicalpageoptions
  {%
    logical pages=1,%
  }
  \pgfpageslogicalpageoptions{1}
  {
    border code=\pgfsetlinewidth{1.5pt}\pgfstroke,%
    border shrink=\pgfpageoptionborder,%
    resized width=.95\pgfphysicalwidth,%
    resized height=.95\pgfphysicalheight,%
    center=\pgfpoint{.5\pgfphysicalwidth}{.5\pgfphysicalheight}%
  }%
}

\pgfpagesuselayout{boxed}

\begin{document}
  \maketitle
  \section{OOP basics (GOOD)}
  \begin{itemize}
    \item Given a description of a class, be able to write a header file containing a declaration for that class, including header guards, data member declarations, friend function declarations, and member function prototypes. 
    \item Be able to wrtie common types of member functions such as \textbf{default constructor that takes no args}, \textbf{a constructor that takes args, accessor and mutator member functions (get and set functions)}.
  \end{itemize} 
  \begin{mdframed}
  \begin{minted}{c++}
#ifndef HEADER_H 
#define HEADER_H
  
class Entity {

  // Data member declarations
  int x, y; 

  // friend function declaration
  friend std::ostream& operator<<(std::ostream&, const Entity&); 

  // Constructors
  Entity();
  
  Entity(int x, int y)
  Entity(const Entity&); // copy constructor (used with pointers)

  // Member functions
  void push(int value);
  void pop();

  // Accessor and mutator functions (get and set)
  void set_x(int x);
  void get_x() const;
};
  \end{minted}
  \end{mdframed}
\begin{itemize}
\end{itemize}
\section{Binary Search Algorithm (GOOD)}
\begin{itemize}
  \item Be able to trace the execution of the binary search algorithm as it searches an array for a particular search key. You should be able to track the values of the subscripts \textbf{low, high, mid} as the algo progresses.
\end{itemize}
\section{C++ Pointers and References}
\begin{itemize}
  \item Be able to declare a pointer, a pointer to constant data, a constant pointer, or a constant pointer to constant data.

\end{itemize}
\begin{mdframed}
\begin{minted}{c++}
// regular pointer
int* ptr = nullptr
int* ptr = &num;

// pointer to constant data
int* const ptr = nullptr;

// const pointer
const int* ptr;

// const pointer to const data.
const int* const ptr;
\end{minted}
\end{mdframed}
\begin{itemize}
  \item Be able to answer questions about pointer syntax, the indirection operator (unary *), the ``address of'' operator (unary \&), and the relationship between a pointer and the variables it points to.
  
\end{itemize}
\section{The const keyword}
\begin{itemize}
  \item Be able to declare a pointer to const data, a const pointer, or a const pointer to const data. Know what restrictions doing so places on using the pointer.
\end{itemize}
\bigbreak \noindent
\begin{itemize}
  \item int* const ptr (pointer to const data)
    \begin{itemize}[label=$\circ$]
      \item pointer to const data cannot change the value stored at the address being pointed to. 
      
    \end{itemize}
  \item const int* ptr (const pointer)
    \begin{itemize}[label=$\circ$]
      \item Cannot change the address stored in the pointer variable
    \end{itemize}
  \item const int* const ptr; (const pointer to const data)
    \begin{itemize}[label=$\circ$]
      \item this prevents you from changing the value stored at the address being pointed to and it prevents you from changing the address stored in the pointer variable.
    \end{itemize}
\end{itemize}
\bigbreak \noindent \bigbreak \noindent
    \begin{itemize}
      \item Be able to declare a reference to const data. Know what restrictions this places on using the reference.
    \begin{itemize}[label=$\circ$]
      \item const string\& s (Cannot change the value of the variable that the reference var refers to)
    \end{itemize}
  \item Be able to list the things that can't be done in a const member function.
    \begin{itemize}[label=$\circ$]
      \item Cannot change the values of data members
        \item cannot call non const methods.
        \item an object that is not const can call a const member function or a non-const member function. AN object that is const (or a pointer to a const object or a reference to a const object) can only call member functions that are const.
    \end{itemize}
    \end{itemize} 
    \section{Default Function Arguments}
    \begin{itemize}
      \item Default values for function and member function parameters may be coded as part of a prototype.
      \item Parameters with default values must be trailing parameters in the function prototype parameter list.
      \item When a function defined with default parameter values is called with trailing arguments missing, the default values are used.
    \end{itemize}
    \section{Function and Member function overloading}
    \begin{itemize}
      \item You should know the criteria used by the compiler to distinguish between two or more functions or member functiosn with the same name and in the same scope
    \begin{itemize}[label=$\circ$]
      \item The number of args
      \item The data types of the args
      \item The order of the data types
      \item Whether or not a member function is const
    \end{itemize}
    \end{itemize}
    \newpage
    \section{The this$\rightarrow$ pointer}
    \begin{itemize}
      \item The this pointer points to the object that called the method. 
      \item For a member function of class \textit{class\_name}, the data type of the this pointer is either class\_name* (if the member funciton is not const) or const class\_name* (for a  const member function)
      \item Standalone functions and static member functions do not have a this pointer since they are not called for a specific object.
    \end{itemize}
    \section{The friend Keyword}
    \begin{itemize}
      \item Know how to declare a class or standalone function to be a friend class.
        \begin{itemize}[label=$\circ$]
          \item For a class, code the keyword friend followed by the class name.
          \item For a function, code the keyword friend followed by the function's prototype
        \end{itemize}
      \item Friendship grants direct access to the private members of a class.
      \item Friendship must always be explicity declared.
        \begin{itemize}[label=$\circ$]
          \item If A is a friend of B, B is not automatically a friend of a.
          \item If A is a friend of B and B is a friend of C, A is not automatically a friend of C (no transitive property applies)
        \end{itemize}
    \end{itemize}
    \section{Operator Overloading}
    \begin{itemize}
      \item Be able to list the aspects of an operator that cannot be changed by operator overloading
        \begin{itemize}[label=$\circ$]
          \item Precedence
          \item number of arguments (operands)
          \item direction evaluation
          \item and how the operator works with built-in data types
        \end{itemize}
      \item Know which operators must be overloaded as member functions, and when an operator must be overloaded as a standalone function
        \begin{itemize}[label=$\circ$]
          \item The stream insertion operator (" << ")
          \item any other function that has an lhs operand that is not a object of our class.
        \end{itemize}
      \item Know what the function call generated by the compiler for an overloaded operator will look like.
        \begin{itemize}[label=$\circ$]
          \item a.operator+(b) (this is what it looks like for member functions)
          \item operator+(a, b); (this is what it looks like for non-member functions)
        \end{itemize}
      \item Be able to write overloaded operator functions similar to those used on programming assignments and recitation projects - stream insertion operator, relational operators, arithmetic operators, subscript operators.
    \end{itemize}
    \section{Dynamic Storage Allocation}
    \begin{itemize}
      \item Know how to use the new[] operator to allocate dynamic storage for an array.
      \item A dynamically-allocated array of objects create with new[] will call the class's default constructor for each object of the array.
      \item Know how to use the delete[] operator to de-allocate a dynamic array.
      \item A dynamically-allocated array of objects will have the destructor called for each object of the array when it is deleted with delete[].
      \item A ``shallow'' copy of an object copies only the object but not the dynamic storage that it owns. A ``deep'' copy of an object copies the object and the dynamic storage that it owns.
      \item A class that allocates dynamic storage for one or more of its data members requires coding all three of the following functions.
        \begin{itemize}[label=$\circ$]
          \item destructor
          \item copy constructor
          \item copy assignment operator
        \end{itemize}
  \end{itemize}
  \section{Destructor}
  \begin{itemize}
    \item A destructor is called for a class object when it goes out of scope, is deleted, or the program ends.
    \item Be able to write a destructor for a class that dynamically created an array.
  \end{itemize}
  \section{copy constructor}
  \begin{itemize}
    \item Be able to list the three situations which may result in a call to the copy constructor.
      \begin{itemize}[label=$\circ$]
        \item When a new object is created and initialized with an exisiting object.
        \item When an object is passed to a function or method by value
        \item when an object is returned from a function or method by value.
      \end{itemize}
  \end{itemize}
  \section{Copy Assignment Operator}
  \begin{itemize}
    \item Be able to write a copy assignment operator for a class that dynamically creates an array.
  \end{itemize}
  \section{Abstract Data Type Definition}
  \begin{itemize}
    \item An \textbf{abstract data type} is a data type defined in terms of what may be stored and the operations that may be performed on it. It does not specify \textbf{how} the data is represented in memory.

  \end{itemize}
  \section{Stack and Queue ADTs}
  \begin{itemize}
    \item Know the behavior that the stack and queue ADTs produce (``last in, first out'' vs. ``first in, first out'')
      \begin{itemize}[label=$\circ$]
        \item stack is last in first out (LIFO)
        \item queue is first in first out (FIFO)
      \end{itemize}
    \item Know the types of errors that can occur when using a stack or queue (underflow on pop(), top(), front(), back())
    \item Be able to add an item to a stack or queue (array implementation only)
    \item Be familiar with the other typical operations performed on a stack or queue (size(), empty()), copy constructor, copy assignement operator, destructor, etc.
  \end{itemize}
\end{document}

\documentclass{report}

\input{~/latex/template/preamble.tex}
\input{~/latex/template/macros.tex}

\title{\Huge{Maps and Unordered Maps}}

\author{\huge{Matt Warner}}
\date{\huge{}}
\pagestyle{fancy}
\fancyhf{}
\rhead{}
\lhead{\leftmark}
\cfoot{\thepage}
% \usepackage[default]{sourcecodepro} \usepackage[T1]{fontenc}

\pgfpagesdeclarelayout{boxed}
{
  \edef\pgfpageoptionborder{0pt}
}
{
  \pgfpagesphysicalpageoptions
  {%
    logical pages=1,%
  }
  \pgfpageslogicalpageoptions{1}
  {
    border code=\pgfsetlinewidth{1.5pt}\pgfstroke,%
    border shrink=\pgfpageoptionborder,%
    resized width=.95\pgfphysicalwidth,%
    resized height=.95\pgfphysicalheight,%
    center=\pgfpoint{.5\pgfphysicalwidth}{.5\pgfphysicalheight}%
  }%
}

\pgfpagesuselayout{boxed}

\begin{document}
  \maketitle
  \section{Maps} 
  Maps can be used by adding the map header file to your program:
  \begin{minted}{c++}
  #include <map>
  \end{minted}
\bigbreak \noindent
Maps are \textbf{associative containers} that store elements in a mapped fashion. Each element has a key value and a mapped value. No two mapped values can have the same key value.
\begin{mdframed}
 Some basic functions associate with std::map are: 
 \bigbreak \noindent
 \begin{itemize}
   \item begin() - Returns an iterator to the first element in the map. 
   \item end() - Returns an iterator to the theoretical element that follows the last element in the map
   \item size() - Returns the number of elements in the map.
   \item max\_size() - Returns the maximum number of elements that the map can hold
   \item empty() - Returns whether the map is empty.
   \item pair insert(keyvalue, mapvalue)  - Adds a new element to the map.
   \item erase(iterator position) - Removes the elements at the position pointed by the iterator
   \item erase(const g) - Removes the key-value 'g' from the map.
   \item clear() - Removes all the elements from the map.
 \end{itemize}
\end{mdframed}
\bigbreak \noindent
\subsection*{Examples of std::map}
The following examples shows how to perform basic operations on map containers
\bigbreak \noindent
\textbf{Example 1: using .begin() and .end()}
\begin{minted}{c++}
#include <iostream>
#include <map>
#include <string>
using namespace std;

int main()
{
  map<string, int> mp;    // Creates a map of strings to integers

  // Insert some values into the map
  mp["one"] = 1;
  mp["two"] = 2;
  mp["three"] = 3;
  
  // Get an iterator pointing to the first element in the map
  map<string, int>::iterator it = mp.begin();

  // Iterate through the map and print the elements
  while (it != mp.end())
  {
    cout << "Key: " << it->first
         << ", Value: " << it->second endl;
    it++;
  }
  return 0;
}
\end{minted}
 \noindent
\textbf{Output:}
\begin{mdframed}
 Key: one, Value: 1 
 \bigbreak \noindent
 Key: three, Value: 3 
 \bigbreak \noindent
 Key: two, Value: 2 
\end{mdframed}
\nt{
  Maps are implemented as a balanced binary tree, and it automatically sorts its elements based on the key.
  \bigbreak \noindent
  The sorting is done in ascending order according to the key's value.
}
\textbf{Example 2: Using size() function}
\begin{minted}{c++}
int main(}
{
  map<string,int> map;

  map["one"] = 1;
  map["two"] = 2;
  map["three"] = 3;

  cout << "Size of map: " << map.size() << endl;

  return 0;
}
\end{minted}
\textbf{Output:}
\begin{mdframed}
  \vspace{.5mm}
  Size of map: 3  
\end{mdframed}
  \vspace{.5mm}
  \textbf{Example 3: inserting elements }
  \begin{minted}{c++}
  std::map<int,int> mp;

  mp.insert(pair<int,int>(1,40));  // First method for inserting elements

  mp[1] = 3;     // Second method
  \end{minted}
\newpage
\noindent
We can iterate through an unordered\_map as such.
\begin{minted}{c++}
for (const auto& pair : mp)
;
\end{minted}
\subsection{Example:}
Lets say that we iterated through an array of integers and created a hash map to hold their frequencys
\begin{minted}{c++}
for (const auto& i : array) ++mp[i];
\end{minted}
Now, lets say we wanted to find the element that appears the most amount of times in the array. \\ We can do so by setting up two variables. One to hold the value (that is the highest count) \\ and one to hold the key (that is the element we are searching for)
\bigbreak \noindent
Now, just like linear search, we can iterate through the pairs in the hash map, and find the largest value, while also storing its key in our targetelement variable.
\begin{mdframed}
\begin{minted}{c++}
int largestFrequency = 0; // starting point for comparison
int targetelement; // will end up holding our target element

for (const auto& pair : mp) {
  if (pair.second > largestFrequency)
  {
    largestFrequency = pair.second;
    targetelement = pair.first
  }
  std::cout << targetelement;
}
\end{minted}
\end{mdframed}
if we wanted to build on this and find the next most frequent element, we can use the \colorbox{lightgray}{.erase(key)} method to remove it from the map and continue. 
\bigbreak \noindent
That would look something like this:
\begin{mdframed}
\begin{minted}{c++}
int i = 0; // loop counter
int k = 3; // k most frequent elements
std::vector<int> solution; // vector to hold the k most frequent elements
While (i < k) {
  int largestFrequency = 0;
  int targetElement;
  for (const auto& pair : mp) // iterate through the key,value pairs in mp
  {
    if (pair.second > largestFrequency)
    {
      largestFrequency = pair.second; // update largestFrequency 
      targetElement = pair.first // update targetElement
    }
  }
  solution.push_back(targetElement); // append element to solution vector
  mp.erase(targetElement); // remove from hash map
  i++; // increment loop count

}
\end{minted}
\end{mdframed}
\end{document}

\documentclass{report}

\input{~/latex/template/preamble.tex}
\input{~/latex/template/macros.tex}

\title{\Huge{Shell}}
\author{\huge{Matt Warner}}
\date{\huge{}}
\pagestyle{fancy}
\fancyhf{}
\rhead{}
\lhead{\leftmark}
\cfoot{\thepage}
% \usepackage[default]{sourcecodepro} \usepackage[T1]{fontenc}

\pgfpagesdeclarelayout{boxed}
{
  \edef\pgfpageoptionborder{0pt}
}
{
  \pgfpagesphysicalpageoptions
  {%
    logical pages=1,%
  }
  \pgfpageslogicalpageoptions{1}
  {
    border code=\pgfsetlinewidth{1.5pt}\pgfstroke,%
    border shrink=\pgfpageoptionborder,%
    resized width=.95\pgfphysicalwidth,%
    resized height=.95\pgfphysicalheight,%
    center=\pgfpoint{.5\pgfphysicalwidth}{.5\pgfphysicalheight}%
  }%
}

\pgfpagesuselayout{boxed}

\begin{document}
  \maketitle
  \section{UNIX Command Interpreters}
 A command Interpreter is commonly refered to as a \textbf{shell} \\
 Ever UNIX system has a ``Bourne shell compatible'' shell  \\
 The shell is where is simply where your commands are typed, and then interpreted.
 \subsection{History}
 \begin{itemize}
   \item sh: original Bourne shell, written in 1978 by Steve Bourne 
   \item ash: Almquist shell, BSD-licensed replacement of sh
 \end{itemize}
 \subsection{Today}
 \begin{itemize}
   \item bash: Bourne-again shell, GNU replacement of sh 
   \item dash: Debian Almquist shell, small scripting shell
 \end{itemize}
\section{Variables}
The shell remembers values stored in variables. \\ Variables have a name and a type, these types are limited to: \textbf{strings}, \textbf{numbers}, \textbf{arrays}.
\bigbreak \noindent
To set a string variable we would write
\begin{minted}{c++}
$ variable=value
\end{minted}
To display our variable we would write,
\begin{minted}{c++}
$ echo $variable
\end{minted}
We can also change the value stored in our variable by just repeating the same syntax used to create it.

\subsection{Variable Scope}
Varible holds values for the duration of shell invocation, i.e., when the shell is exited, the variable goes away.
\bigbreak \noindent
Variables can be exported into enviroment: \\ The enviroment is basically a set of key,value pairs, where the key is the variable name and the value if the value stored in the variable.
\\ So if a variable is exported into an enviroment, if the shell ends, its still in the enviroment. Therefore it has changed to an enivorment variable.
the command we use is
\begin{minted}{c++}
$ export varname
\end{minted}
Basically, when the user logs in, there is an enviroment for the user. The enviroment is basically a set of variables
\bigbreak \noindent
Here are some examples of those variables.
\begin{itemize}
  \item HOME: full pathname of your home directory
  \item PATH: list of directories to search for commands
  \item USER: Your user name, also UID for user id
  \item SHELL: full pathname of your login shell
  \item PWD Current working directory
  \item HOSTNAME: current hostname of the system
  \item HISTSIZE: Number of commands to remember
  \item PS1: primary prompt
  \item ?: Return status of most recently executed command
  \item \$: Process id of current process.
\end{itemize}
\subsection{The PATH Variable}
One variable that is quite useful is the PATH variable.
The path variable lists a set of directories. \\
The shell finds commands in these directories.
\bigbreak \noindent
\subsection{Bash shell prompt}
the prompt can be changed vua the PS1 shell variable
\begin{minted}{c++}
PS1="$USER > "
\end{minted}
\textbf{Special ``PS1'' shell variable settings:}
\begin{itemize}
  \item \textbackslash w \ \ \ current working directory
  \item \textbackslash h \ \ \ hostname
  \item \textbackslash u \ \ \ username
  \item \textbackslash d \ \ \ date
  \item \textbackslash t \ \ \ time
  \item \textbackslash a righ the ``bell''
\end{itemize}
\textit{\textbf{Example:}}
\begin{minted}{c++}
$ PS1="\u@\h \w \$ "
student@csci330 ~ $
\end{minted}
\section{Shell aliases}
Allows you to assign a different name to a command
\bigbreak \noindent
To check current aliases:
\begin{minted}{c++}
$ alias
\end{minted}
to set alias:
\begin{minted}{c++}
$ alias ll="ls -al"
\end{minted}
to remove alias
\begin{minted}{c++}
$ unalias ll
\end{minted}
\newpage
\section{Keeping variables}
Variables set on the command line end when the shell end. \\
What we can do is create a file and write our aliases there. After we write them to the file, we can run the command \colorbox{lightgray}{source} on the file and run it as a shell script.
\begin{minted}{c++}
source aliases
\end{minted}
Now, all the alias created in the file will be able to be used.
\section{Shell History}
The shell can remember commands that you have typed in. This is called the shell history \\
Commands in the shell history can be:
\begin{itemize}
  \item re-called
  \item edited
  \item re-executed
 \end{itemize}
 The size of the history is set via shell variables 
 \begin{itemize}
   \item per session \ \ \ \ \ \ \ \ HISTSIZE=500 
   \item per user \ \ \ \ \ \ HISTFILESIZE=100
\end{itemize}
To view the history buffer:
\begin{minted}{c++}
history [-c] [count]
\end{minted}
\section{Command Sequence}
allows series of commands all at one
\begin{minted}{c++}
date;pwd;ls
\end{minted}
\section{Command substiution}
command surrounded by back quotes is run and replaced by its standard output
\bigbreak \noindent
newlines in the output are replaced by spaces
\begin{minted}{c++}
ls -l `which passwd`
var=`whoami`;echo $var
\end{minted}
We can also perform command substitution like this
\begin{minted}{c++}
$ (command)
\end{minted}
Here is an example of that:
\begin{minted}{c++}
echo User $(whoami) is on $(hostname)
\end{minted}
\section{Output redirection}
\textit{\textbf{Syntax:}} command $>$ file \\
sends command output to file, instead of the terminal
\bigbreak \noindent
\textit{\textbf{Examples:}}
\begin{minted}{c++}
ls > listing
cat listing > filecopy
\end{minted}
\vspace{1.5mm}
\noindent
  NOTE: if ``file'' exists, it is overwritten
\newpage
\section{Input Redirection}
\textit{\textbf{Syntax:}}
\begin{minted}{c++}
command < file
\end{minted}
command reads (takes input) from file instead of keyboard
\bigbreak \noindent
\textit{\textbf{Example:}}
\begin{minted}{c++}
tr a-z A-Z < listing
\end{minted}
this translates all the lowercase characters from the file \textit{listing} to uppercase.
\subsection{Examples: Output / input}
redirecting input and output
\begin{minted}{c++}
tr A-Z a-z < r.in > r.out
\end{minted}
Output of command becomes input to next:
\begin{minted}{c++}
ls > /tmp/out.txt; wc< /tmp/out.txt
\end{minted}
Eliminate the middleman by using the pipe command
\begin{minted}{c++}
ls | wc
\end{minted}
\section{Appending Output}
\textit{\textbf{Sytax}}
\begin{minted}{c++}
command >> file
\end{minted}
This adds output of command at the end of file \\ 
if the file does not exist, the shell creates it
\bigbreak \noindent
\textit{\textbf{Examples:}}
\begin{minted}{c++}
date > usage-status
ls -l >> usage-status
du -s >> usage-status
\end{minted}
\subsection{Here Document}
read unput for current source, uses $<<$ symbol \\
\textit{\textbf{Syntax:}}
\begin{minted}{c++}
command << LABEL
\end{minted}
This reads following lines until line starting with ``LABEL''
\bigbreak \noindent
\textit{\textbf{Example}}
\begin{minted}{c++}
wc - l << done
> line one
> line two
> DONE
2
\end{minted}
\section{File Descriptor}

\begin{itemize}
  \item positive integer for every open file
  \item process tracks its open files with this number
    \bigbreak \noindent
    \subitem 0 - standard input
    \subitem 1 - standard output
    \subitem 2 - standard error output
\end{itemize}
\nt{ 
  bash can use file descriptor to refer to a file
}
\section{Redirecting syntax}
\begin{itemize}
  \item Output
    \subitem $>$ or 1$>$ filename
\subitem  \ \ \ \ \ \ \ 2$>$ filename
\item input
  \subitem $<$ or 0$<$
\item Combining outputs
  \subitem 2$>$\&1 \ \ \ or \ \ \ \&$>$ \ \ \ or \ \ \ $>$\&
\end{itemize}
\textit{\textbf{Example:}}
\begin{minted}{c++}
cat mouse > /tmp/out.txt 2>&1

cat mouse &> /tmp/out.txt

\end{minted}

\end{document}

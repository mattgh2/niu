\documentclass{report}

\input{~/latex/template/preamble.tex}
\input{~/latex/template/macros.tex}

\usepackage{booktabs}
\title{\Huge{Shell Scripts}}
\author{\huge{Matt Warner}}
\date{\huge{}}
\pagestyle{fancy}
\fancyhf{}
\rhead{}
\lhead{\leftmark}
\cfoot{\thepage}
% \usepackage[default]{sourcecodepro} \usepackage[T1]{fontenc}

\pgfpagesdeclarelayout{boxed}
{
  \edef\pgfpageoptionborder{0pt}
}
{
  \pgfpagesphysicalpageoptions
  {%
    logical pages=1,%
  }
  \pgfpageslogicalpageoptions{1}
  {
    border code=\pgfsetlinewidth{1.5pt}\pgfstroke,%
    border shrink=\pgfpageoptionborder,%
    resized width=.95\pgfphysicalwidth,%
    resized height=.95\pgfphysicalheight,%
    center=\pgfpoint{.5\pgfphysicalwidth}{.5\pgfphysicalheight}%
  }%
}

\pgfpagesuselayout{boxed}

\begin{document}
  \maketitle
\tableofcontents
\newpage
  \section{Overview}
Shell scripts can do what can be done on command line
\bigbreak \noindent
Shell scripts simplify recurring tasks. If you cannot find an existing utility to accomplish a task, you can build one using a shell script
\nt{
  Much of UNIX administration and house keeping is done via shell scripts
}
\section{Shell Script features}
\begin{itemize}
  \item Variables for storing data 
\item Decision-making control (e.g. if and case statements)
\item Looping abilities (e.g. for and while loops)
\item Functions for modularity
\item Any UNIX command
  \subitem file manipulation: cat, cp, mv, ls, wc, tr, ...
  \subitem utilities: grep, sed, awk, ...
\item Comments: lines starting with \#
\end{itemize}
\section{The basics}
\textit{\textbf{First line is always shebang}}
\begin{minted}{bash}
#! /bin/bash
\end{minted}
\textit{\textbf{To run shell scripts}}
\begin{minted}{bash}
bash script
\end{minted}
\textit{\textbf{Or, make executable}}
\begin{minted}{bash}
chmod +x script
./script
\end{minted}
\subsection{Simple Script}
\begin{mdframed}
  \begin{minted}{bash}
#! /bin/bash
date > usage-status
ls -l >> usage-status
du -s * >> usage-status
\end{minted}
\end{mdframed}
\section{Bash Shell Programming Features}
\begin{itemize}
  \item Variables 
    \subitem string, number, array
  \item input/output
    \subitem echo, printf
    \subitem command line args, read from user
  \item Decision
    \subitem conditional execution, if-then-else, case
  \item Repetition
    \subitem while, until, for
  \item Functions
\end{itemize}
\section{User-defined shell variables}
\textit{\textbf{Syntax:}}
\begin{minted}{bash}
varname=value
\end{minted}
\textit{\textbf{Example:}}
\begin{minted}{bash}
rate=moderate
echo "Rate today: $rate"
\end{minted}
\nt{

  use double quotes if value of variable contains white spaces 
  \bigbreak \noindent
\textit{\textbf{Example:}}
\bigbreak \noindent
name=``Thomas William Flowers''
}
\section{Output via echo command}
\begin{itemize}
  \item Simplest form of writing to standard output 
\end{itemize}
\textit{\textbf{Syntax:}}
\begin{minted}{bash}
echo [-ne] arguments
\end{minted}
-n suppresses trailing newline
\vspace{2mm}

\noindent -e enables escape sequences:
\\ \\ 
\textbackslash t horizontal tab \\
\textbackslash b backspace \\
\textbackslash a alert \\
\textbackslash n newline
\subsection{Examples: shell scripts with output}
\begin{minted}{bash}
#! /bin/bash
echo "You are running these processes:"
ps


#! /bin/bash
echo -ne "Dear $USER:\nWhat's up this month:"
cal
\end{minted}
\section{Command line arguments}
\begin{itemize}
  \item Use arguments to modify script behavior 
  \item command line arguments become positional parameters to shell script
  \item positional paramters are numbered variables
    \subitem \$1, \$2, \$3 \ldots
\end{itemize}
\subsection{Meanings}
\$1 \ \ \ first parameter \vspace{2mm}

\noindent \$2 \ \ \ second parameter \vspace{2mm}

\noindent     \$\{10\}  \ \ 10th parameter \ \ (prevents ``\$1'' misunderstanding) \vspace{2mm}

\noindent \$0  \ \ name of the script \vspace{2mm}

\noindent \$*  \ \ all positional parameters \vspace{2mm}

\noindent \$\#  \ \ the number of arguments
\subsection{Example: Command Line Arguments}
\begin{mdframed}
  \begin{minted}{bash}
#! /bin/bash
# Usage: greetings name1 name2

echo $0 to you $1 $2
echo Today is `date`
echo Good Bye $1
\end{minted}
\end{mdframed}
Make sure to protect complete argument 
\begin{minted}{bash}
#! /bin/bash
# counts lines in directory listing

ls -l "$1" | wc -l
\end{minted}
If we had a bash script as such:
\begin{minted}{bash}
#! /bin/bash
ls -l $1 | wc -l
\end{minted}
And had a file called ``file example'' \\
We would not be able to use this file as a parameter, since our argument is not protected.
\section{Arithmetic expressions}
\textit{\textbf{Syntax:}}
\begin{minted}{bash}
$((expression))
\end{minted}
This can be used for simple arithmetic
\begin{minted}{bash}
count=1
count=$((count+20))
echo $count
\end{minted}
\section{Array variables}
\textit{\textbf{Syntax:}}
\begin{minted}{bash}
varname=(list of words)
\end{minted}
Accessed via index:
\begin{minted}{bash}
${varname[index]}
${varname[0]}     first word in array
${varname[*]}     all words in array
\end{minted}
\subsection{Using array variables}
\textit{\textbf{Examples}}
\begin{minted}{bash}
ml=(mary ann bruce linda dara)
echo $ml
**prints mary**

echo ${ml[*]}
**prints mary ann bruce linda dara**

echo ${ml[2]}
**prints bruce**

ml[2]=john
echo ${ml[*]}
**prints mary ann john linda dara**
\end{minted}
\section{Output: printf command}
\textit{\textbf{Syntax:}}
\begin{minted}{bash}
printf format[arguments]
\end{minted}
Writes formatted arguments to standard output under the control of ``format''
\bigbreak \noindent
Format string may contain:
\begin{itemize}
  \item plain characters: printed to output 
  \item escape characters: e.g. \textbackslash t, \textbackslash n, \textbackslash a \ldots
  \item format specifiers: prints next successive arguement
\end{itemize}
\subsection{printf format specifiers}
\begin{minted}{bash}
%d  number
\end{minted}
\textit{\textbf{also}}
\begin{minted}{bash}
%10d    10 chars wide
%-10d   left justified
\end{minted}
\begin{minted}{bash}
%s  string
\end{minted}
\textit{\textbf{also}}
\begin{minted}{bash}
%20s    20 chars wide
%-20s   left justifed
\end{minted}
\newpage
\begin{mdframed}
  \begin{minted}{bash}
printf "random number\n"

printf "random number %d\n" 12

printf "random number %d\n" $RANDOM

printf "random number %10d\n" $RANDOM

printf "rando number %-10d %s\n" $RANDOM $USER
\end{minted}
\end{mdframed}
\section{User input: read command}
\textit{\textbf{Syntax:}}
\begin{minted}{bash}
read [-p "prompt"] varname [more vars]
\end{minted}
words entered by user are assigned to \vspace{2mm}

\noindent varname and "more vars"
\bigbreak \noindent
Last variable gets rest of input file
\subsection{Example: Accepting User input}
\begin{minted}{bash}
#! /bin/bash
read -p "enter your name: "  first last

echo "First name: $first""
echo "Last name: $last"
\end{minted}
\nt{
  -p for prompt
}
\section{exit Command}
This terminates the current shell, the running script.
\bigbreak \noindent
\textit{\textbf{Syntax}}
\begin{minted}{bash}
exit [status]
\end{minted}
\textit{\textbf{Note:}} The default exit status is 0

\nt{
  Generall convention: use negative exit status is something went wrong.
\bigbreak \noindent
0 indicates exit success \\ 
1 indicates something minor went wrong 
}
\bigbreak \noindent
Predefined variable ``?'' holds exit status of last command \vspace{2mm}

\noindent ``0'' indicates success, all else if failure.
\newpage
\noindent \textbf{\underline{Examples:}}
\begin{minted}{bash}
ls > /tmp/out
echo $?

grep -q "root" /var/log/auth.log # grep in quiet mode (exits with zero status if any match is found)
echo $?
\end{minted}
\section{Conditional Execution}
Operators $\|$ and \&\& allow conditonal execution
\bigbreak \noindent
\textit{\textbf{Syntax:}}
\begin{minted}{bash}
cmd1 && cmd2 # cmd2 executed if cmd1 succeeds

cmd1 || cmd2 # mcd2 executed if cmd1 fails
\end{minted}
\subsection{Conditional Execution: Examples}
\begin{minted}{bash}
grep $USER /etc/passwd && echo "$USER found"

grep student /etc/group || echo "no student group"
\end{minted}
\section{test command}
\textit{\textbf{Syntax:}}
\begin{minted}{bash}
test expression
[ expression ]
\end{minted}
Evaluates `expression' and returns true or false
\begin{mdframed}
  \begin{minted}{bash}
if test $name = "Joe"
then
  echo "Hello Joe"
fi

if [ $name = "Joe" ]
then 
  echo "Hello Joe"
fi
\end{minted}
\end{mdframed}
\section{if statements}
\begin{minted}{bash}
if [ condition ]; then
  statements
elif [ condition ]; then
  statements
else
  statements
fi
\end{minted}
\newpage
\subsection{test Relation Operators}
\begin{table}[h]
\centering
\begin{tabular}{@{}lll@{}}
\toprule
\textbf{Meaning}          & \textbf{Numeric} & \textbf{String}    \\ \midrule
Greater than              & -gt              &                    \\
Greater than or equal     & -ge              &                    \\
Less than                 & -lt              &                    \\
Less than or equal        & -le              &                    \\
Equal                     & -eq              & =                  \\
Not equal                 & -ne              & !=                 \\
String length is zero     &                  & -z str             \\
String length is non-zero &                  & -n str             \\
file1 is newer than file2 &                  & file1 -nt file2    \\
file1 is older than file2 &                  & file1 -ot file2    \\ \bottomrule
\end{tabular}
\caption{test relation operators}
\label{your-label-here}
\end{table}
\subsection{Compound logical expressions}
\bigbreak \noindent
\noindent\textbf{\texttt{! expression}}
\begin{itemize}[leftmargin=*,label={}]
  \item True if \texttt{expression} is false.
\end{itemize}

\noindent\textbf{\texttt{expression -a expression}}
\begin{itemize}[leftmargin=*,label={}]
  \item True if both \texttt{expressions} are true.
\end{itemize}

\noindent\textbf{\texttt{expression -o expression}}
\begin{itemize}[leftmargin=*,label={}]
  \item True if one of the \texttt{expressions} is true.
\end{itemize}
\subsection{Example: compound logical expressions}
\begin{minted}{bash}
if [ ! $Years -lt 20 ]; then
  echo "You can retire now."
fi

if [ "$Status" = "H" ] && [ "$Shift" = 3 ]; then
  echo "shift $Shift gets \$$Bonus bonus"
fi

if [ "$Calls" -gt 150 ] || [ "$Closed" -gt 50 ]; then
  echo "You are entitled to a bonus"
fi
\end{minted}
\subsection{File Testing operators}
 \begin{table}[ht]
\centering
\begin{tabular}{@{}ll@{}}
\toprule
\textbf{Command} & \textbf{Meaning} \\
\midrule
\texttt{-d file} & true if `file' is a directory \\
\texttt{-f file} & true if `file' is a regular file \\
\texttt{-r file} & true if `file' is readable \\
\texttt{-w file} & true if `file' is writable \\
\texttt{-x file} & true if `file' is executable \\
\texttt{-s file} & true if length of `file' is nonzero \\
\bottomrule
\end{tabular}
\end{table}
\newpage
\section{Debugging using ``set"}
The ``set'' command is a shell built-in command. It has options to allow tracing of execution.
\vspace{1.5mm}

-v print shell input lines as they are read \vspace{1.5mm}

-x option displays expanded commands and its arguments
\bigbreak \noindent
Options can be turned on or off \vspace{1.5mm}

To turn on the option: \ \ \textbf{set -xv} \vspace{1.5mm}

To turn off the options: \ \ \textbf{set +xv}
\bigbreak \noindent
Options can also be set via she-bang line \vspace{1.5mm}
\begin{minted}{bash}
#! /bin/bash -xv
\end{minted}
\bigbreak \noindent
\textbf{\underline{Example:}} Using the following script
\begin{minted}{bash}
#!/bin/bash
read -p "enter your name: " first last
set -v
echo "First name: $first"
set +v
echo "Last name: $last"
\end{minted}
This will add the following line to the output
\begin{minted}{bash}
echo "First name: $first"
\end{minted}
Additionally, using -x will instead print the whats inside the variable.
\section{The case Statement}
To make decision that is based on multiple choices, we use case statements. This is can be thought of in the same way as a switch statement in C++.
\bigbreak \noindent
\textit{\textbf{Syntax:}}
\begin{minted}{bash}
case word in
  pattern1) command-list1
  ;;
  pattern2) command-list2
  ;;
  patternN) command-listN
  ;;
esac
\end{minted}
Additionally, the case pattern can contain meta characters, such as: \vspace{1.5mm}

* \vspace{1.5mm}

? \vspace{1.5mm}

[ \ldots ] \vspace{1.5mm}

[ :class: ]
\nt{
  Multiple patterns can be listed via $|$ (pipeline)
}
\newpage
\begin{mdframed}
  \begin{minted}{bash}
#!/bin/bash
echo "Enter Y to see all files including hidden files"
echo "Enter N to see all non-hidden files"
echo "Enter Q to quit"

read -p "Enter your choice: " reply

case "$reply" in
  Y|YES) echo "Displaying all (really...) files"
         ls -a ;;

  N|NO) echo "Displaying all non-hidden files..."
        ls ;;

  Q)    exit 0 ;;

  *) echo "Invalid choice!"; exit 1 ;;
esac
\end{minted}
\end{mdframed}
\section{The while loop}
This executes ``command-list'' as long as ``test-command'' evaluates successfully
\bigbreak \noindent
\textit{\textbf{Syntax:}}
\begin{minted}{bash}
while test-command
do
  command-list
done
\end{minted}
\bigbreak \noindent
\textbf{\underline{Example:}}
\begin{mdframed}
\begin{minted}{bash}
#!/bin/bash
#script shows users's active processes

cont="y"
while [ "$cont" = "y" ]; do
  ps
  read -p "again (y/n)? " cont
done
echo "done"

\end{minted}
\end{mdframed}
\bigbreak \noindent
\textbf{\underline{Example:}}
\begin{mdframed}
\begin{minted}{bash}
#!/bin/bash
# copies files from home into webserver directory
# a new directory is created every hour
PICSDIR=/home/student/pics
WEBDIR=/var/www/webcam
while true; do
  DATE=`date +%Y%m%d` 
  HOUR=`date +%H`
  mkdir $WEBDIR/$DATE 
  while [ "$HOUR != "00" ]; do
    mkdir $WEBDIR/$DATE/$HOUR
    mv $PICSDIR/*.jpg $WEBDIR/$DATE/$HOUR
    sleep 3688
    HOUR=`date +%H`
  done
done
\end{minted}
\end{mdframed}
\section{The until Loop}
executes ``command-list'' as long as \\ ``test-command'' does \textbf{not} evaluate successfully
\bigbreak \noindent
\textit{\textbf{Syntax:}}
  \begin{minted}{bash}
untill test-command
do
  command-list
done
\end{minted}
\bigbreak \noindent
\textbf{\underline{Example:}}
\begin{mdframed}
  \begin{minted}{bash}
#!/bin/bash
# script shows user's active processes

stop="n"
untill [ "$stop" = "y" ]; do
  ps
  read -p "done (y/n)? " stop
done
echo "done"
\end{minted}
\end{mdframed}
\section{The for Loop}
executes "commands" as many times as the number of words in the ``word-list''
\bigbreak \noindent
\textit{\textbf{Syntax:}}
\begin{minted}{bash}
for variable in word-list
do
  commands
done
\end{minted}
\bigbreak \noindent
\textbf{\underline{Example:}}
\begin{mdframed}
  \begin{minted}{bash}
#!/bin/bash

for index in 7 6 5 4 3 2
do
  echo $index
done
\end{minted}
\end{mdframed}
\clearpage \noindent
\textbf{\underline{Example:}}
\begin{mdframed}
\begin{minted}{bash}
#!/bin/bash
# compute average weekly temperature
TempTotal=0
for day in 1 2 3 4 5 6 7
do
  read -p "Enter temp for $day: " Temp
  TempTotal=$((TempTotal+Temp))
done
AvgTemp=$((TempTotal/7))
echo "Average temperature: " $AvgTemp
\end{minted}
\end{mdframed}
\nt{
  instead of explicitly writting out 1 2 3 4 5 6 7 
  \bigbreak \noindent
  We can just write: for day in `seq 7`
  \bigbreak \noindent
  Or we could write: for day in Mon Tue Wed Thu Fri Sat Sun
  \bigbreak \noindent
  Better yet: If we had the days of the week in a file, we could write: for day in `cat day-file`
}
\section{Looping over arguments}
Simplest form will iterate over all command line arguments
\bigbreak \noindent
\textbf{\underline{Example:}}
\begin{mdframed}
\begin{minted}{bash}
#!/bin/bash
for parm
do
  echo $parm
done
\end{minted}
\end{mdframed}
\section{break and continue}
using the keywords \textbf{break} or \textbf{continue} will interrupt for, while or until loop
\bigbreak \noindent
\textit{\textbf{break statement:}} \vspace{1.5mm}

\noindent Terminates execution of the loop \vspace{1.5mm}

\noindent transfers control to the statement AFTER the done statement
\bigbreak \noindent
\textit{\textbf{continue statement:}} \vspace{1.5mm}

\noindent skips the rest of the current iteration \vspace{1.5mm}

\noindent continues execution of the loop
\pagebreak
\section{Shell Functions}
must be defined before they can be referenced.
\nt{
  Usually placed at the beginning of the script
}
\bigbreak \noindent
\textbf{\textit{Syntax:}}
\begin{minted}{bash}
function-name () {
  statements
}
\end{minted}
\bigbreak \noindent
\textbf{\underline{Example:}}
\begin{mdframed}
\begin{minted}{bash}
#!/bin/bash

funky () {
  # This is a simple function
  echo "This is a funky function"
}

# declaration must precede call:

funky # function call
\end{minted}
\end{mdframed}
\subsection{Function parameters}
\begin{itemize}
  \item Need not be declared 
  \item Arguments provided via function call are accessible inside function as \$1, \$2, \$3, \ldots
\end{itemize}
\bigbreak \noindent
\textbf{\underline{Example:}}
\begin{mdframed}
\begin{minted}{bash}
#!/bin/bash
checkfile() {
  for file
  do
    if [ -f "$file" ]; then
      echo "$file is a file"
    else
      if [ -d "$file" ]; then
        echo "$file is a directory"
        fi
    fi
  done
}
checkfile . funtest
\end{minted}
\end{mdframed}
\subsection{Local Variables in Functions}
Variables defined within functions are global, i.e their values are known throughout the entire script
\bigbreak \noindent
Keyword \textbf{local} inside a function defines variables that are local to that function, i.e not visible outside
\pagebreak
\bigbreak \noindent
\textbf{\underline{Example:}}
\begin{mdframed}
\begin{minted}{bash}
#!/bin/bash
global="pretty good variable"

foo () {
  local inside="not so good variable"
  echo $global
  echo $inside
  global="better variable"
}

echo $global
foo
echo $global
echo $inside
\end{minted}
\end{mdframed}
\subsection{return from function}
\bigbreak \noindent
\textbf{\textit{Syntax:}}
\begin{minted}{bash}
return [status]
\end{minted}
\bigbreak \noindent
Ends exeuction of function \vspace{1.5mm}

\noindent optional numeric argument sets return status

default is ``return 0''
\bigbreak \noindent
\textbf{\underline{Example:}}
\begin{mdframed}
\begin{minted}{bash}
#!/bin/bash
testfile() {
  if [ $# -gt 0 ]; then
    if [ ! -r $1 ]; then
      return 1
    fi
  fi
}
if testfile funtest; then
  echo "funtest is readable"
fi
\end{minted}
\end{mdframed}


\end{document}

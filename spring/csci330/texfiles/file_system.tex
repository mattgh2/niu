\documentclass{report}

\input{~/latex/template/preamble.tex}
\input{~/latex/template/macros.tex}

\tite{\Huge{File System}}
\author{\huge{Matt Warner}}
\date{\huge{}}
\pagestyle{fancy}
\fancyhf{}
\rhead{}
\lhead{\leftmark}
\cfoot{\thepage}
% \usepackage[default]{sourcecodepro} \usepackage[T1]{fontenc}

\pgfpagesdeclarelayout{boxed}
{
  \edef\pgfpageoptionborder{0pt}
}
{
  \pgfpagesphysicalpageoptions
  {%
    logical pages=1,%
  }
  \pgfpageslogicalpageoptions{1}
  {
    border code=\pgfsetlinewidth{1.5pt}\pgfstroke,%
    border shrink=\pgfpageoptionborder,%
    resized width=.95\pgfphysicalwidth,%
    resized height=.95\pgfphysicalheight,%
    center=\pgfpoint{.5\pgfphysicalwidth}{.5\pgfphysicalheight}%
  }%
}

\pgfpagesuselayout{boxed}

\begin{document}
  \maketitle
  \section{Some Basic Commands}
  List of basic commands shown in the lecture
  \begin{mdframed}
  \begin{itemize}
    \item passwd - change password 
    \item ls - list files
    \item more - show content of file, page by page
    \item logout - logout from system
    \item date - display date and time
    \item who - display who is on the screen
    \item clear - clears the terminal
    \item man - find and display system manual pages
  \end{itemize}
\end{mdframed}
  \section{RTFM - The man Command}
  \subsection{Overview}
  shows pages from system manual
  \bigbreak \noindent
  \textit{\textbf{Syntax:}} man [option] [-S section] command-name
  \begin{itemize}
    \item \$ man date 
    \item \$ man - k date
    \item \$ man crontab
    \item \$ man -S 5 crontab
  \end{itemize}
  \bigbreak \noindent
  Caveates: \vspace{1mm}

  Some commands are aliases \vspace{1mm}

  Some commands are part of shell
  \subsection{Section info}
  \begin{enumerate}
    \item User commands 
    \item System calls
    \item C libray function
    \item Special system files
    \item File formats
    \item Games
    \item Misc. features
    \item System admin utilities
  \end{enumerate} 
  \newpage
  \section{The Unix file system}
  The UNIX file system has a hierarchical organinzation of files and it is organized as a single tree.
  \nt{
    The typical Unix file system type is ext4
    \bigbreak \noindent
    It also supports FAT, NTFS, UDF, ...
  } 
  \subsection{Directory terminology}
  \begin{itemize}
  \item Root Directory: /
    \subitem top-most directory in any UNIX file structure
  \item Home Directory: \~{}
    \subitem directory owned by a user
    \subitem default location where user logs in
  \item Current Directory: .
    \subitem default location for working with files
  \item Parent Directory: ..
    \subitem directory immediately above the current directory
  \end{itemize}    
  \section{Paths}
  a path is simply a list of names seperated by ``/''. There are two types of paths to refer to a file or directory.
  \begin{itemize}
    \item \textit{\textbf{Absolute Path}} 
    \subitem Traces a path from root to a file or a directory
     \subitem Always begins with the root (/) directory
     \subitem \textit{\textbf{Example:}} /home/student/Desktop/assign1.txt
   \item \textit{\textbf{Relative Path}}
     \subitem Traces a path from the current directory
     \subitem No initial forward slash (/)
     \subsubitem dot (.) refers to current directory
     \subsubitem two dots (..) refers to one level up in directory hierarchy
     \subitem \textit{\textbf{Example:}} Desktop/assign1.txt
  \end{itemize}
  \section{Linking Files}
  Allows one file to be known by a different name 
  \vspace{1.5mm}

\noindent  Link is a reference to another file stored elsewhere 
\bigbreak \noindent
There are 2 types of links, we have
\begin{itemize}
  \item Hard link (default) 
  \item Symbolic link (a.k.a ``soft link'')
\end{itemize}
\textit{\textbf{Syntax:}} ln [-s] target local
\newpage
\subsection{Symbolic Link}
\begin{itemize}
  \item Refers to target file via path 
  \item Created without checking the existence or permissions of target file
  \item Can be circular linked to another symbolic link
  \item Can cross physical file systems
\end{itemize}
\subsection{Hard Link}
\begin{itemize}
  \item Refers to target file by its inode number 
  \subitem inode number of a file is unique only
  \subitem within physical device file system
  \item Checks for the existence of target file
  \item Other file continues to exist as long as at least one directory contains it
  \item Cannot link to a file in a different physical file system
\end{itemize}
\section{Locating Files: find}
\textit{\textbf{Syntax:}} find path-list expression(s)
\bigbreak \noindent
``find'' recursively descends through directories in path-list and applies expression to every file
\bigbreak \noindent
\textit{\textbf{Examples:}}
\begin{itemize}
  \item find . -name ``*.txt'' 
  \item find /tmp -empty -delete
\end{itemize}
\end{document}

\input{~/latex/template/pset.tex}
\input{~/latex/template/macros.tex}


% envirmments:
%   - homeworkProblem

% commands
%   - \partr (for roman)
%   - \parta (for arabic)
%   - \partal (for alpha)
%   - \alg 
%   - \deriv
%   - \pderiv
%   - \dx
%   - \solution
% Probability commands
%   - \E
%   - \Var
%   - \Cov
%   - \Bias

\begin{document}
% {HW NUMBER}{date}{class}{KEEP EMPTY}{prof}{\textbf{author}}
\setHWinfo{1}{01/22/25}{MATH 210}{}{Scott Rexford}{\textbf{Matt Warner}}

\maketitle
\newpage
\begin{homeworkProblem}
    Determine $i$, $n$, $P$, and $F$ (in the notation from the text and notes) for the following situations.
    \begin{enumerate}[label=\roman*)]
        \item $\$600$ invested at $4.5\%$ $APR$ compounded monthly grows to $\$785$ in $6$ years.
        \item $\$900$ invested at $4.5\%$ $APR$ compounded quarterly from January $1^{st}$ 2010 to October $1^{st}$ 2021 grows to some future value $F$. \textit{so find the given $i$, $n$, $P$ and this future value $F$}.
    \end{enumerate}
\begin{mdframed}
    \textbf{Solution:}
    \\ \\
    \textbf{Part i)} 
    \\
    From the problem statement, we have:
    \begin{align*}
        P &= 600 \\
        r &= 0.045 \\
        n &= 6  \\
        m &= 12 \\
    \end{align*}
    To calculate $F$, we use the formula
\end{mdframed}
\end{homeworkProblem}

\begin{homeworkProblem}
    Suppose that $\$8,000$ is invested with compound interest and $8\%$ annual percentage rate ($APR$.) Determine the future value in 4 years in the compounding is:
    \begin{enumerate}[label=\roman*)]
        \item  quarterly,
        \item monthly,
        \item daily.
    \end{enumerate}

\end{homeworkProblem}
\begin{homeworkProblem}
Suppose that $\$3,500$ is invested at $4.8\%$ $APR$ and quarterly compounding.
    \begin{enumerate}[label=\roman*)]
        \item What is the balance after 4 years?
        \item What is the interest accrued in these 4 years?
        \item What is the return on investment $(ROI)$ for these 4 years?
        \item 
    \end{enumerate}

\end{homeworkProblem}

\begin{homeworkProblem}
\end{homeworkProblem}

\begin{homeworkProblem}
\end{homeworkProblem}

\begin{homeworkProblem}
\end{homeworkProblem}

\begin{homeworkProblem}
\end{homeworkProblem}

\begin{homeworkProblem}
\end{homeworkProblem}

\begin{homeworkProblem}
\end{homeworkProblem}

\begin{homeworkProblem}
\end{homeworkProblem}


\end{document}

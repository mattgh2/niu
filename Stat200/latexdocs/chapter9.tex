\documentclass{report}

\input{~/latex/template/preamble.tex}
\input{~/latex/template/macros.tex}

\title{\Huge{Chapter 9 Notes}}
\author{\huge{Matt Warner}}
\date{\huge{}}
\pagestyle{fancy}
\fancyhf{}
\rhead{}
\lhead{\leftmark}
\cfoot{\thepage}
% \usepackage[default]{sourcecodepro} \usepackage[T1]{fontenc}

\pgfpagesdeclarelayout{boxed}
{
  \edef\pgfpageoptionborder{0pt}
}
{
  \pgfpagesphysicalpageoptions
  {%
    logical pages=1,%
  }
  \pgfpageslogicalpageoptions{1}
  {
    border code=\pgfsetlinewidth{1.5pt}\pgfstroke,%
    border shrink=\pgfpageoptionborder,%
    resized width=.95\pgfphysicalwidth,%
    resized height=.95\pgfphysicalheight,%
    center=\pgfpoint{.5\pgfphysicalwidth}{.5\pgfphysicalheight}%
  }%
}

\pgfpagesuselayout{boxed}

\begin{document}
  \maketitle
\chapter*{Chapter 9 – Hypothesis Tests Based on a Single Sample}
\section*{Section 9.1 – The Parts of a Hypothesis Test and Choosing the Alternative Hypothesis}
\textbf{Statistical inference} – using a sample statistic (e.g. $\bar{x}$ = sample mean or $\hat{p}$ = sample proportion) to make some statement or some conclusion about a population parameter (e.g. $\mu$ = population mean or $p$ = population proportion). There are 2 types.
\begin{itemize}
  \item \textbf{Confidence interval}
    \begin{itemize}[label=$\circ$]
    \item Uses a sample statistic to estimate the unknown value of a population parameter.
    \item We write the estimate in the form of an interval that we believe captures the actual or true parameter value.
  \end{itemize}
  \item \textbf{Hypothesis test (or significance test)}
  \begin{itemize}
    \item There are 2 hypotheses.
    \begin{itemize}
      \item Null hypothesis – labeled $H_0$
      \item Alternative hypothesis (or research hypothesis) – labeled $H_A$ or $H_1$
    \end{itemize}
    \item Each hypothesis states one or more possible values for a parameter.
    \item The purpose of the hypothesis test is to decide whether or not the sample data gives evidence against $H_0$ and in favor of $H_A$.
  \end{itemize}
\end{itemize}
\section*{Example (Examples 9.1 – 9.3, pp. 392 – 393)}
For each of the following, state the null and alternative hypotheses.

\begin{enumerate}[label=(\alph*)]
  \item Trust, in general, has been declining for many years; most concerning, trust has been decreasing in public health and safety institutions and workers. According to a recent survey, only 58\% of Americans trust doctors. Suppose a national advertising campaign is conducted to address confidence in doctors and medical leaders, and an experiment is conducted to determine whether it has been effective.
  $$ H_o : p = .58$$
  $$ H_A : p > .58$$
  \item Even though the roads are crowded, it was reported that the mean number of kilometers
driven per year by each Australian driver is 13,716. Over the past year, more public
transportation has been made available in most large cities to encourage people to drive
less and to use buses and trains instead. An observational study is conducted to
determine whether the mean number of kilometers driven each year has decreased
  $$ H_o : \mu = 13716$$
  $$ H_A : \mu < 13716$$
\item The variance in thickness for 20-lb printer paper at a manufacturing plant is known to be
0.0007. A new process is developed that uses more recycled fiber, and an experiment is
conducted to detect any difference in the variance in paper thickness.
  $$ H_o : \sigma^2 = .0007$$
  $$ H_A : \sigma^2 \neq .0007$$
\end{enumerate}
  \bigbreak \noindent
  \nt{
    We always believe the null hypothosies untill we can show the alternative is true
  }
  \bigbreak \noindent
  \section{General Approach or Overview}
When conducting hypothesis testing, it is essential to follow a structured approach to ensure clarity and rigor in the analysis. The general approach consists of the following steps:

\begin{enumerate}
  \item \textbf{State the Null and Alternative Hypotheses:} Begin by clearly stating the null hypothesis ($H_0$) and the alternative hypothesis ($H_A$). Additionally, decide on the level of significance, denoted as $\alpha$.

  \item \textbf{Calculate the Test Statistic:} Calculate the test statistic. This statistic provides a measure of how close or consistent the sample data is to the hypothesized value of the parameter.

  \item \textbf{Give the Rejection Region:} Define the rejection region, which is the range of values of the test statistic that would lead to the rejection of $H_0$.

  \item \textbf{Give a Decision and Write a Conclusion:}
    \begin{itemize}
      \item If the test statistic falls within the rejection region:
        \begin{itemize}
          \item Reject $H_0$ and conclude in favor of $H_A$.
          \item The data or results are considered statistically significant.
          \item For the conclusion, write: "The sample does have enough evidence, at level $\alpha$, to support $H_A$."
        \end{itemize}
      \item If the test statistic does not fall within the rejection region:
        \begin{itemize}
          \item Do not reject $H_0$.
          \item The data or results are not considered statistically significant.
          \item For the conclusion, write: "The sample does not have enough evidence, at level $\alpha$, to support $H_A$."
        \end{itemize}
    \end{itemize}
\end{enumerate}

\textbf{IMPORTANT NOTES:}
\begin{itemize}
  \item The primary goal is to provide support for $H_A$.
  \item Initially, we assume that $H_0$ is true until the sample data provides sufficient evidence to support $H_A$. This concept is analogous to the legal system's presumption of innocence until guilt is proven.
  \item If the sample data fails to support $H_A$, it does not imply that $H_0$ is proven true. This is akin to the legal system, where failure to prove guilt does not equate to proving innocence.
  \item Avoid using phrases like "Accept $H_0$" or "There is evidence to support $H_0$."
\end{itemize}
\pagebreak
\section*{Section 9.2 - Hypothesis Test Errors}
Our hypothesis testing procedure could lead to an incorrect decision
\bigbreak \noindent

\begin{table}[ht]
  \centering
  \renewcommand{\arraystretch}{1.5}
  \begin{tabular}{l|c|c}
    \toprule
    \textbf{ } & \textbf{Reject $H_0$} & \textbf{Do not reject $H_0$} \\
    \hline
    \midrule
    $H_0$ is true &  \parbox{5.5cm}{\vspace{0.5cm}  Type I Error - Rejecting $H_0$ even though $H_0$ is true.} &  \parbox{5.5cm}{\vspace{0.5cm} \centerline{Correct}} \\
    \hline
    $H_A$ is true &  \parbox{4cm}{\vspace{0.5cm}  \centerline{Correct}} &  \parbox{5.5cm}{\vspace{0.5cm} Type II Error - Not Rejecting the null when $H_A$ is true.} \\
    \hline
    \bottomrule
  \end{tabular}
\end{table}
\section{Type I and Type II Errors in Hypothesis Testing}

\vspace{10pt}
In the context of statistical hypothesis testing, it's essential to understand the concepts of Type I and Type II errors, which play a crucial role in decision-making and drawing conclusions. These errors are often associated with the acceptance or rejection of null hypotheses.
\vspace{10pt}

\subsection{Type I Error (False Positive)}

\vspace{10pt}
\textbf{Definition:} Type I error, also known as a false positive, occurs when a statistical test incorrectly rejects a null hypothesis that is actually true. In other words, it represents a false alarm or an erroneous positive result.
\bigbreak \noindent
\textbf{Symbol:} Type I error is typically denoted as $\alpha$ (alpha).
\bigbreak \noindent
\textbf{Example:} An example of Type I error is when a study concludes that a new drug is effective in treating a medical condition when, in reality, the drug has no therapeutic effect.
\vspace{3mm}

\subsection{Type II Error (False Negative)}

\vspace{10pt}
\textbf{Definition:} Type II error, also known as a false negative, occurs when a statistical test fails to reject a null hypothesis that is actually false. In this case, the test overlooks a genuine effect or difference.
\bigbreak \noindent
\textbf{Symbol:} Type II error is often denoted as $\beta$ (beta).
\bigbreak \noindent
\textbf{Example:} An example of Type II error is when a security system fails to detect unauthorized access to a computer network, missing a legitimate security breach.
  \bigbreak \noindent

  \pagebreak
\section*{Example (Exercise 9.47, p.401) – Hypothesis Testing for Highway 405}

Highway 405, running from northern to southern California, is famously known as the busiest interstate road in the United States, with approximately 374,000 vehicles using it daily, often referred to as "Carmageddon." Transportation officials are considering raising tolls in California to fund planned repairs. To make this decision, they will conduct a hypothesis test to determine whether there is evidence suggesting that the mean number of vehicles per day on Highway 405 has increased.
\bigbreak \noindent \bigbreak \noindent
\noindent \begin{large}{\textbf{State the Null and Alternative hypothesis}}\end{large}
\bigbreak \noindent
  $$ H_o : \mu = 374,000$$
  $$ H_A : \mu > 374,000$$
  \bigbreak \noindent
  \begin{large}{\textbf{Describe the type I and type II errors.}}\end{large}
  \bigbreak \noindent
  \begin{center}
  Type I - deciding traffic increased when it really didnt
\bigbreak \noindent
  Type II - Deciding traffic didnt increase when it in fact has
\end{center}
\bigbreak \noindent \bigbreak \noindent
\begin{large}{\textbf{If a type I error is committed, who is more angry?}\end{large}
  \bigbreak \noindent
  \begin{center}
  Deciding higher traffic brings higher tolls even though traffic really didn't increase
  \bigbreak \noindent
  Therefore, The drivers are more angry
\end{center}
\bigbreak \noindent
\begin{large}{\textbf{If a type II error is committed, who is more angry?}\end{large}
  \bigbreak \noindent
  \begin{center}
  Deciding traffic didn't increase means state misses out on toll money. 
\bigbreak \noindent
  So, the transportation officials are more angry
  \end{center}

\pagebreak
\subsection*{The Logic of Hypothesis Testing}
\bigbreak \noindent
Suppose that we want to test the following hypotheses.
$$ H_0 : \text{ coin is fair}$$
$$ H_A : \text { coin is unfair}$$
\bigbreak \noindent
\noindent Sample Data - Suppose that 20 tosses of the coin result in 11 heads and 9 tails
\bigbreak \noindent
\textit{\textbf{What would be the decision?}}
\bigbreak \noindent
Reject $H_0$ \\ 
Do not reject $H_0$ $\rightarrow$ This one
\bigbreak \noindent
\textit{\textbf{Why?}}
\begin{itemize}
  \item When $H_0$ is true we would expect to get around 10 heads and 10 tails 
  \item The observed 11 heads and 9 tails is
    \begin{itemize}[label=$\circ$]
      \item Close to what we expect 
      \item likely to happen
    \end{itemize}
\end{itemize}
\bigbreak \noindent
Sample Data - Suppose the 20 tosses of the coin results in 19 heads and 1 tail
\bigbreak \noindent
\textit{\textbf{What would be the decision?}}
\bigbreak \noindent
Reject $H_0$ $\rightarrow$ This one \\
Dont reject $H_0$
\bigbreak \noindent
\textit{\textbf{Why?}}
\begin{itemize}
  \item When $H_0$ is true we would expect to get around 10 heads and 10 tails 
  \item The observed 19 heads and 1 tail is
    \begin{itemize}[label=$\circ$]
      \item Not close to what we expect 
      \item Unlikely to happen
    \end{itemize}
\end{itemize}
\bigbreak \noindent
\nt{
  The observed data being close to/far from what we expect will be determined by the rejection region of the test
  \bigbreak \noindent
  The observed data being likely/unlikely to happen will be determined by the $p$ value of the test.
}

\pagebreak
\section*{Section 9.3 - Hypothesis Tests Concerning a Population Mean When $\sigma$ is Known}
\bigbreak \noindent
Provided
\begin{itemize}
  \item Underlying population has a normal distribution or $n$ is large ($ n \geq 30)$
  \item $\sigma$ (population standard deviation) has a known value
\end{itemize}
\bigbreak \noindent
To test the null hypothesis

$$ H_0 : \mu = \mu_0$$
versus one of these alternative hypotheses

$$ H_A : \mu < \mu_0$$
$$H_A : \mu > \mu_0$$
$$ H_A : \mu \neq \mu_0$$
\bigbreak \noindent
Use the test statistic

$$ Z = \dfrac{\bar{x} - \mu_0}{\dfrac{\sigma}{\sqrt{n}}}$$
\bigbreak \noindent
\nt{
  The standard deviation
  $$ \sigma_{\bar{x}} = \dfrac{\sigma}{\sqrt{n}}$$
  Is often called the standard error of $\bar{x}$
}
\bigbreak \noindent
Rejection Region
\bigbreak \noindent
If (i): \hspace{10mm} Reject $H_0$ if $Z \leq - Z_a$
\bigbreak \noindent
If (ii): \hspace{10mm} Reject $H_0$ if $Z \geq Z_a$
\bigbreak \noindent
If (iii): \hspace{10mm} Reject $H_0$ if $Z \leq -Z_{\frac{a}{2}}$ or if $Z \geq Z_{\frac{a}{2}}$

\pagebreak
\subsection*{Example}
The principal of a high school claims that \( \mu \), the mean SAT score for all graduates
of his school, is higher than the national average of 500. To prove his claim, he selects a sample
of 75 recent graduates and calculates a sample mean of \( \bar{x} = 520 \). Does the sample provide
enough evidence to support his claim? We may assume that the SAT scores for all graduates of his high
school follow a normal distribution with a population standard deviation of \( \sigma = 100 \). Test using
a level of significance \( \alpha = 0.05 \).
\bigbreak \noindent
\textbf{Problem 1.} Give the population and the parameter
\bigbreak \noindent
\textit{\textbf{Our population:}}
$$\text{all grads of this highschool who took the sat}$$
\bigbreak \noindent
\textit{\textbf{our parameter:}}
$$ \mu = \text { Mean SAT score for this population}$$
\bigbreak \noindent
\textbf{Problem 2.} State the null and alternative hypotheses
$$ H_0 : \mu = 500$$
$$ H_A : \mu > 500$$
\bigbreak \noindent
\textbf{Problem 3.} Calculate the test statistic
\bigbreak \noindent
\textit{\textbf{Using the formula}}
$$ Z = \dfrac{\bar{x} - \mu_0}{\frac{\sigma}{\sqrt{n}}}$$
\textit{\textbf{We get}}
$$ \dfrac{520 - 500}{\dfrac{100}{\sqrt{75}}} = 1.73$$
\nt{
  When $H_0$ : $\mu = 500$ is true, $\bar{x}$ should be around 500
  \bigbreak \noindent
  Is $\bar{x} = 520$ close to or far from 500?
}
\noindent\textbf{Problem 4.} Find the rejection region
\bigbreak \noindent
\textbf{Problem 5.} Give the decision and write a conclusion to the problem

\pagebreak
\noindent 
\q
Buiness and management (in the book)
\bigbreak \noindent 
$$ H_0 : \mu = 51500$$
$$ H_A : \mu < 51500$$
$$ \alpha = .01$$
\textit{\textbf{Big sample and population standard deviation tells us this is a Z test}}
$$ Z = \dfrac{\bar{x} - \mu}{\dfrac{\sigma}{\sqrt{n}}}$$
\textit{\textbf{So,}}

$$ \dfrac{49762 - 51500}{\dfrac{3750}{\sqrt{38}}}$$

$$ = -2.86$$
\textit{\textbf{Looking up the z score we get in the table we get}}
$$ Z = -2.3263$$
\textit{\textbf{Reject $H_0$ if Z $\le$ -2.3263}}
So, we reject $H_0$, there is enough evidence at the 1\% level, to conclude mean salaray has decreased.
\nt{
  if $<$, we use the left end of the tail
  \bigbreak \noindent
  if $>$, we use the right end of the tail
}
\section*{9.4 - $p$ Values}
\subsection*{Example (Pain reliever problem)}
\textit{\textbf{Hypothesis}}
$$ H_0 : \mu = 500$$
$$ H_A : \mu >500$$
\textit{\textbf{Find the area beyond 1.73}}
$$ 1 - .9582$$
$$ = .0418$$
Since the $p$ value is less than alpha, we reject the null
\end{document}

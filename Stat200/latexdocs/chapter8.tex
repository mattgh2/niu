\documentclass{report}

\input{~/latex/template/preamble.tex}
\input{~/latex/template/macros.tex}

\title{\Huge{chapter 8}}
\author{\huge{Matt Warner}}
\date{\huge{}}
\pagestyle{fancy}
\fancyhf{}
\rhead{}
\lhead{\leftmark}
\cfoot{\thepage}
% \usepackage[default]{sourcecodepro} \usepackage[T1]{fontenc}

\pgfpagesdeclarelayout{boxed}
{
  \edef\pgfpageoptionborder{0pt}
}
{
  \pgfpagesphysicalpageoptions
  {%
    logical pages=1,%
  }
  \pgfpageslogicalpageoptions{1}
  {
    border code=\pgfsetlinewidth{1.5pt}\pgfstroke,%
    border shrink=\pgfpageoptionborder,%
    resized width=.95\pgfphysicalwidth,%
    resized height=.95\pgfphysicalheight,%
    center=\pgfpoint{.5\pgfphysicalwidth}{.5\pgfphysicalheight}%
  }%
}

\pgfpagesuselayout{boxed}

\begin{document}
  \maketitle
  \section*{Chapter 8 - Confidence Intervals Based on a Single Sample}
    \textbf{Satistical inference - }using a sample statistic (e.g. $\overline{x}$ = sample mean) to make some statement or some conclusion about a population parameter (e.g. $\mu$ = population mean) There are two types.
    \begin{itemize}
      \item \textbf{Confidence interval} 
        \begin{itemize}[label=$\circ$]
          \item Uses a sample statistic to \textit{estimate} the unknown value of a parameter.
          \item We write the estimate in the form of an interval that we belive captures the actual or true parameter value.
          \item The confidence interval has a specified level of confidence.
        \end{itemize}
      \item \textbf{Hypothesis Test} (or significance test)
    \end{itemize}
    \bigbreak \noindent
    \subsection*{Section 8.2 - A Confidence Interval for a Population Mean when $\sigma$ is Known}
    \begin{itemize}
      \item Provided 
        \begin{itemize}[label=$\circ$]
          \item Underlying population has a normal distribution or $n$ is large $ \left(n \ge 30\right)$
          \item $\sigma$ (\underline{population} standard deviation) has a \underline{known value}
        \end{itemize}
      \item A $100(1 - a)\%$ confidence interval to estimate $\mu$ is: 
        $$ \overline{x}\pm \left(Z_{\frac{a}{z}}\cdot \frac{\sigma}{\sqrt{n}}\right)$$
      \item The \textbf{\underline{critical value}} $ \left(Z_{a}{z}\right)$ is chosen based on the \textbf{\underline{level of confidence}}
        \vspace{1mm}

        \subitem \hspace{-7mm}$\pm{Z\frac{a}{z}}$ = values from a standard normal distribution that capture the middle $100(1-a)\%$
      \item Simplified form
        \begin{itemize}[label=$\circ$]
          \item estimate $\pm${ margin of error}
          \item \textbf{margin of error} = (critical value x \textbf{standard error of $\overline{\mathbf{x}}$})
        \end{itemize}
    \end{itemize}
\q
\textbf{An administrator at a large university wants to estimate $\mu$, the mean GPA of all students on campus. A random sample of $ n = 50$ students is selected and the GPA of each student is recorded. The resulting mean is $\overline{x} = 2.60$. Assume that the GPAs in the population are normally distributed with $\sigma = 0.75$}
\bigbreak \noindent \bigbreak \noindent
\textbf{Problem 1. Calculate a 95\% confidence interval}
\bigbreak \noindent
\textit{\textbf{its asking for 95\%, so our tails are}}

$$ \alpha = .05 \rightarrow \dfrac{\alpha}{2} = 0.25$$
\textit{\textbf{our z scores are then}}

$$ \pm 1.96$$
\textit{\textbf{So, using the formula}}
$$ \overline{x}\pm \left(Z_{\frac{a}{z}}\cdot \frac{\sigma}{\sqrt{n}}\right)$$
\textit{\textbf{We have}}
$$ 2.60\pm(1.96\cdot\dfrac{.75}{\sqrt{50}})$$
\textit{\textbf{our confidence intervals are then}}

$$ 2.60 \pm .21$$

$$ = 2.39, 2.81$$
\bigbreak \noindent \bigbreak \noindent
\hrule
\bigbreak \noindent
\textbf{Problem 2. Calculate a 90\% confidence interval to estimate $\mu$}
\bigbreak \noindent
\textit{\textbf{If its asking for 90\% are tails are}}

$$ 0.05$$
\textit{\textbf{Our z scores are then}}

$$\pm{1.6499}$$
\textit{\textbf{So, using the formula}}

$$ \overline{x}\pm \left(Z_{\frac{a}{z}}\cdot \frac{\sigma}{\sqrt{n}}\right)$$
\textit{\textbf{We have}}

$$ 2.60\pm(1.6449 \cdot\dfrac{.75}{\sqrt{50}})$$
\textit{\textbf{So, our confidence intervals are}}
$$ 2.60\pm{.17}$$
$$ = 2.43, 2.77$$
\bigbreak \noindent \bigbreak \noindent
\hrule
\bigbreak \noindent
\textbf{Problem 3. Calculate a 99\% confidence interval to estimate $\mu$}
\bigbreak \noindent
\textit{\textbf{99\% means are tails are}}

$$.005$$
\textit{\textbf{our z scores would then be}}

$$\pm{2.5758}$$
\textit{\textbf{So}}

$$2.60\pm(2.5758\cdot\dfrac{.75}{\sqrt{50}})$$
\textit{\textbf{So, our confidence intervals are}}

$$ 2.60\pm{.27}$$
$$ 2.33, 2.87$$

\pagebreak
\subsection*{The effect of choosing the confidence level and sample size}
The confidence level (90\%, 95\%, 99\% etc. ) and the sample size ($n$) are chosen by the statistician or researcher. It is important to understand how these choices affect the overall length of the confidence interval.
\bigbreak \noindent
\begin{itemize}
  \item If the confidence level \textbf{increases} then the interval gets \textbf{wider} and \textbf{less} precise
  \item If the confidence level \textbf{decreases}, then the interval gets \textbf{narrower} and \textbf{more }precise
  \item If the sample size $n$ \textbf{increases} then the interval gets \textbf{narrower} and \textbf{more} precise
  \item If the sample size $n$ \textbf{decreases}, then the interval gets \textbf{wider} and \textbf{less} precise
\end{itemize}
\bigbreak \noindent
\q
Which confidence interval would be longer and which would be shorter?
\bigbreak \noindent
(a) 90\% confidence and $n$ = 50 
\vspace{1mm}

\hspace{-4mm}95\% confidence and $n$ = 50
\vspace{1.5mm}

\sol
\vspace{1mm}

\noindent 95\% confidence interval would be longer than 90\%

\bigbreak \noindent
(b) 95\% confidence and $n$ = 50
\vspace{1mm}

\hspace{-5mm} 95\% confidence and $n$ = 100
\vspace{1.5mm}

\sol
\vspace{1.5mm}

\noindent 95\% confidence and $n$ = 50 would be longer
\bigbreak \noindent
\subsection*{Meaning of confidence}
In the previous example
\begin{itemize}
  \item $\mu$ (= mean GPA of the entire campus) is \textbf{unknown} but it has a \textbf{fixed value}
  \item After getting the sample and making our calculation, our 95\% confidence interval is also \textbf{fixed} with endpoints 2.39 to 2.81
  \item But, if we took another sample we would likely get a different $\bar{x}$ and \textbf{different} endpoints than before, and we would still state that we are confident the new interval captures $\mu$.
\end{itemize}
\bigbreak \noindent
\subsection*{Sample size calculation}
\textbf{Before} the sample is picked
\begin{itemize}
  \item we specify the desired 
    \begin{itemize}[label=$\circ$]
      \item confidence level 
      \item bound for the margin of error (B)
    \end{itemize}
  \item we ask: What size sample is needed?
    $$ n = \left(\dfrac{\sigma \cdot Z_{\frac{a}{z}}}{B}\right)^Z$$
\end{itemize}
\nt{
  \textbf{ALways} round $n$ up to a whole number
}
\section*{8-3 A Confidence Interval for a Population Mean when $\sigma$ is Unknown}
\begin{itemize}
  \item Provided 
    \begin{itemize}[label=$\circ$]
      \item Underlying population has a normal distribution 
      \item $\sigma$ (population standard deviation) has an \textbf{unknown value}
      \item $s$ (sample standard deviation) has a \textbf{known value}
    \end{itemize}
  \item A 100(1 - a)\% confidence interval to estimate $\mu$ is
    $$ \bar{x} \pm \left(t_{\frac{a}{z}}\cdot \frac{s}{\sqrt{n}}\right)$$
  \item $\pm{t\frac{a}{z}}$ = critical values from a \textbf{t distribution} (with \textbf{degree of freedom} $df = n-1$) that capture the middle 100(1 - $a$)\%
\end{itemize}
\bigbreak \noindent
\subsection*{Facts about the t distribution}
\begin{itemize}
  \item It is described by a symmetric, bell-shaped curve that is centered at 0. 
  \item There are many t distributions. Each is identified by giving its degree of freedom ($df$).
  \item It is wider and has more spread than the standard normal distribution
  \item As the $df$ increases the t distribution looks more like the standard normal distribution.
\end{itemize}
\begin{figure}[ht]
\centering
\includegraphics[width=0.5\textwidth]{ /home/mattw/niu/Stat200/latexdocs/figures/tdis.png }
\end{figure}
\q In each of the following, find the appropriate t critical value for use in constructing a condidence interval
\bigbreak \noindent
\textbf{Problem 1. } n = 10, 90\% confident
$$ df = n - 1$$
$$ df = 9$$
$$t_{.05,9} = 1.8331$$
\bigbreak \noindent \bigbreak \noindent
\textbf{Problem 2.} n = 15, 95\% confidence
$$ df = 14$$
\textit{\textbf{95\% confidence means our tails are both .025}}
$$ t_{.025,14}= 2.1448 $$
\bigbreak \bigbreak \noindent
\textbf{n = 21, 99\% confidence}

$$ df = 20$$
\textit{\textbf{our tails our .005, so looking up 20df and .005, we get}}

$$ t_{.005,20} = 2.8453$$
\bigbreak
\q
Oil obtained from orange blossoms through distillation is used in perfume. Suppose the oil yield is normally distributed. In a random sample of 11 distillations, the sample mean oil yield was $\bar{x} = 980.2$ g with standard deviation $s = 27.6$ g.
\vspace{1mm}

\noindent \textbf{Find a 95\% confidence interval for the true mean oil yield per batch.}
$$  \overline{x} \pm \left(T \frac{s}{\sqrt{n}}\right)$$

$$ 980.2 \pm \left(T \frac{27.6}{\sqrt{11}}\right)$$
$$ df = 10$$
$$\text{tails} = .025$$
$$ t_{.025,10} = 2.2281$$
\textit{\textbf{So,}}
$$ 980.2 \pm \left(2.2281 \frac{27.6}{\sqrt{11}}\right)$$
$$ 980.2 \pm 18.5$$
\textit{\textbf{We are 95\% confident that $\mu$ is between 961.7, and 998.7}}
\bigbreak \noindent
\q
The earth is structured...
\bigbreak \noindent
\textit{\textbf{using,}}

$$  \overline{x} \pm \left(T \frac{s}{\sqrt{n}}\right)$$

$$  127.5 \pm \left(T \frac{21.3}{\sqrt{26}}\right)$$
$$ df = 25 \ \ \ \ \ \text{ our tails are .05}$$
\textit{\textbf{finding the value in the t-table}}
$$ t_{.05,25} = 1.7081$$
\textit{\textbf{So,}}
$$  127.5 \pm \left(1.7081 \frac{21.3}{\sqrt{26}}\right)$$
$$ 127.5 \pm{1.7081}$$
\bigbreak  \noindent
\textit{\textbf{We are 90\% confident that $\mu$ is between 120.4, and 134.6}}

\end{document}

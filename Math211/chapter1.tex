\documentclass{report}

\input{~/latex/template/preamble.tex}
\input{~/latex/template/macros.tex}

\title{\Huge{Chapter 1 - Differentiation}}
\author{\huge{Matt Warner}}
\date{\huge{}}
\pagestyle{fancy}
\fancyhf{}
\rhead{}
\lhead{\leftmark}
\cfoot{\thepage}
% \usepackage[default]{sourcecodepro}
% \usepackage[T1]{fontenc}


\pgfpagesdeclarelayout{boxed}
{
  \edef\pgfpageoptionborder{0pt}
}
{
  \pgfpagesphysicalpageoptions
  {%
    logical pages=1,%
  }
  \pgfpageslogicalpageoptions{1}
  { 
    border code=\pgfsetlinewidth{1.5pt}\pgfstroke,%
    border shrink=\pgfpageoptionborder,%
    resized width=.95\pgfphysicalwidth,%
    resized height=.95\pgfphysicalheight,%
    center=\pgfpoint{.5\pgfphysicalwidth}{.5\pgfphysicalheight}%
  }%
}

\pgfpagesuselayout{boxed}

\begin{document}
  \maketitle
\section*{Limits: A numerical and Graphical Approach}
\bigbreak \noindent \bigbreak \noindent
\begin{minipage}{0.44\textwidth}
Consider the pattern formed by the following sequence of numbers	
$$ 0.9, 0.99, 0.999, 0.9999, \text{ and so on}$$
\bigbreak \noindent \bigbreak \noindent
The numbers appear to be getting closer to the number 1, yet never equal 1 exactly. Assuming that the sequence continues in the same manner, we could say that the \textit{limit} of this sequence of numbers is 1. Note that this sequence of numbers approaches 1 ``from the left,'' meaning that all numbers in the sequence are less than the limit, 1. We write $x \rightarrow a $, read ``x approaches a from the left'', to represent a sequence of numbers that approaches $a$ from the left.
\vspace{2mm}

Thus, the sequence 0.9, 0.99,0.999, and so on, is written $x \rightarrow 1$
\end{minipage}
\begin{minipage}{0.55\textwidth}
  \vspace{15mm}

    \incfig[1]{figfig}
    \bigbreak \noindent
\end{minipage}
\bigbreak \noindent \bigbreak \noindent
    The following sequence approaches 1 ``from the right'':
    $$1.1, 1.01,1.001,1.0001,1.00001, \text{ and so on}$$
    \bigbreak \noindent
    We write $ x \rightarrow a^+$, read ``x approaches $a$ from the right'' to represent a sequence of numbers that approaches $a$ from the right.
    \bigbreak \noindent
    Thus, the sequence 1.1, 1.01,1.001,1.0001,1.00001, and so on, is written $x \rightarrow 1^+$
    \bigbreak \noindent
    \bigbreak \bigbreak \noindent
    \ex{}{
      For each sequence, determine its limit, and rewrite the sequence in the form $x \rightarrow a^-$ or $ x \rightarrow a+$
      \bigbreak \noindent
      \begin{enumerate}
        \item $2.24, 2.249, 2.2499, 2.24999,\ldots$ 
        \item $5.51,5.501,5.5001,5.50001,\ldots$
        \item $\frac{1}{2}, \frac{3}{4}, \frac{7}{8}, \frac{15}{16},\frac{31}{32},\frac{63}{64}$
      \end{enumerate}
    }
    \sol{}
  \bigbreak \noindent
  a) $ x \rightarrow 2.25^-$
  \bigbreak \noindent
  b) $x \rightarrow 5.5^+$
  \bigbreak \noindent
  $ c) x \rightarrow 1^-$
\end{document}

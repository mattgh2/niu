\documentclass{report}

\input{~/latex/template/preamble.tex}
\input{~/latex/template/macros.tex}
\usepackage{minted}
\usemintedstyle{xcode}
\title{\Huge{References and Pointers}}
\author{\huge{Matt Warner}}
\date{\huge{}}
\pagestyle{fancy}
\fancyhf{}
\rhead{}
\lhead{\leftmark}
\cfoot{\thepage}
\lstset{
  showstringspaces=false,
  % ... (other options)
}
% \usepackage[default]{sourcecodepro} \usepackage[T1]{fontenc}

\pgfpagesdeclarelayout{boxed}
{
  \edef\pgfpageoptionborder{0pt}
}
{
  \pgfpagesphysicalpageoptions
  {%
    logical pages=1,%
  }
  \pgfpageslogicalpageoptions{1}
  {
    border code=\pgfsetlinewidth{1.5pt}\pgfstroke,%
    border shrink=\pgfpageoptionborder,%
    resized width=.95\pgfphysicalwidth,%
    resized height=.95\pgfphysicalheight,%
    center=\pgfpoint{.5\pgfphysicalwidth}{.5\pgfphysicalheight}%
  }%
}

\pgfpagesuselayout{boxed}

\begin{document}
\tableofcontents
\newpage
\vspace{-10mm}\chapter*{References and Pointers}
  \section{Introduction}
A computers memory is a sequency of bytes. We can number the bytes from 0 to the last one. Each number, known as an address, represents a location in the memory.
\bigbreak \noindent
Everything we put into memory has an adresss. For example, when we declare and initilize an \textit{int} variable named \textit{power}:
\bigbreak \noindent
\centerline{int power = 9000;}
\bigbreak \noindent
This will set aside an \textit{int}-size piece of memory for the variable \textit{power} somewhere and put the value $9000$ into that memory.
\bigbreak \noindent
\section{References}
In c++, reference variable is an alias for something else, that is, another name for an already existing variable.
\bigbreak \noindent
So suppose we make Sonny a reference to someone named Mark. You can refer to the person as either Sonnny or Mark
\bigbreak \noindent
Heres how we do that with code:
\bigbreak \noindent
Suppose we have an \textit{int} variable called \textit{Mark}, we can create an alias to it by using the \& sign in the declaration.
\bigbreak \noindent
\centerline{int \&sonny = mark;}
\bigbreak \noindent
So here, we made \textit{sonny} a reference to \textit{mark}
\bigbreak \noindent
Now when we make changes to \textit{sonny} (add 1, subtract 2, etc), \textit{mark} also changes
\bigbreak \noindent
\nt{
Two things to note about references
\begin{itemize}
  \item Anything we do to the reference also happens to the original 
  \item Aliases cannot be changed to alias something else
\end{itemize}
}
\newpage
\section{Pass-By-Reference}
Pass-by-reference referes to passing parameters to a function by using references. When called, the function can modify the value of the arguments by using the reference passed in.
\nt{
  Arrays do this by default. If you wanted to modify an array passed into the function, using pass-by-refernence is not needed.
  \bigbreak \noindent
  You would do this to modify all other data types, like int and such
}
\bigbreak \noindent
This allows us to:
\begin{itemize}
\item Modify the value of the functions arguments
\item Avoid making copies of a variable/object for performance reasons.
\end{itemize}
\bigbreak \noindent
The following code snippet shows an example of pass-by-reference
\bigbreak \noindent
\begin{minted}{C++}

void swap_num(int &i, int &j) {

  int temp = i;
  i = j;
  j = temp;

}

int main() {

  int a = 100;
  int b = 200;

  swap_num(a, b);

  std::cout << "A is " << a << "\n";
  std::cout << "B is " << b << "\n";

}
\end{minted}
\section{Pass-By-Reference with Const}
Sometimes, we use \textit{const} in a function parameter; this is when we know for a fact that we want to write a function where the parameter won't change inside the function. Here's an example:
\bigbreak \noindent
\begin{minted}{c++}
int triple(int const i) {

  return i * 3;

}
\end{minted}
In this example, we are not modifying the \colorbox{lightgray}{i}. If inside the function \colorbox{lightgray}{triple()}, the value of \colorbox{lightgray}{i} is changed, there will be a compiler error.
\bigbreak \noindent
So to save the computational cost for a function that doesn't modify the parameter values(s), we can actually go a step further and use a \colorbox{lightgray}{const} reference.
\bigbreak \noindent
\begin{minted}{c++}
int triple(int const &i) {

  return i * 3;

}
\end{minted}
This will ensure the same thing: the parameter won't be changed.
However, by making \colorbox{lightgray}{i} a reference to the argument, this saves the computational cost of making a copy of the argument
\section{Memory Address}
The \colorbox{lightgray}{\&} symbol can have another meaning. The ``address of'' operator, \colorbox{lightgray}{\&}, is used to get the \textbf{memory address}, the location in the memory of an object.
Suppose we declare a variable called:
\begin{minted}{c++}
int porcupine_count = 3;
\end{minted}
We can find where this variable is stored in memory by printing out \colorbox{lightgray}{\&porcupine\_count}
\begin{minted}{c++}
std::cout << &porcupine_count << "\n";
\end{minted}
It will return something like:
\begin{minted}{c++}
0x7ffd7caa5b54
\end{minted}
This is a memory address represented in \textbf{hexadecimal}. A memory address is usually denoted in hexadecimal instead of binary for readability and concisness.
\nt{
  \begin{itemize}
    \item When \colorbox{lightgray}{\&} is used in a declaration, it is a reference operator.
    \item When \colorbox{lightgray}{\&} is not used in a declaration, it is an address operator
  \end{itemize}
}
\section{Pointers}
In C++, a \textbf{pointer} variable is mostly the same as other variables, which can store a piece of date.
\bigbreak \noindent
Unlike normal variables, which store a value (such as an \colorbox{lightgray}{int}, \colorbox{lightgray}{double}, \colorbox{lightgray}{char}), a pointer stores a memory address.
\bigbreak \noindent
While references are a newer mechanism that originated in C++, pointers are an older mechanism that was inherited from C. It is recommended to avoid pointers as much as possible; usually, a reference will do the trick.
\nt{
  pointers must be delcared before they can be used, just like a normal variable. They are syntactically distinguished by the \colorbox{lightgray}{*}, so that \colorbox{lightgray}{int*} means ``pointer to int'' and \colorbox{lightgray}{double*} means `` pointer to double''
}
\bigbreak \noindent
\begin{minted}{c++}
int* number;
double* decimal;
char* character;
\end{minted}
We can point our pointers to the memory address of variables by writting
\begin{minted}{c++}
int* var = &otherVar;
\end{minted}
\newpage
\section{Dereference}
The asterisk sign \colorbox{lightgray}{*} a.k.a the dereference operator is used to obtain the value pointed to by a variable. This can be done by preceding the name of a pointer variable with \colorbox{lightgray}{*}.
\begin{minted}{c++}
int foo = *ptr;
\end{minted}
\nt{
  \begin{itemize}
    
    \item When \colorbox{lightgray}{*} is used in a declaration, it is creating a pointer.
    \item When \colorbox{lightgray}{*} is not used in a declaration, it is a dereference operator.
  \end{itemize}
}
\section{Null Pointer}
When we declare a pointer variable like so, its content is not initilized:
\begin{minted}{c++}
int* ptr;
\end{minted}
In other words, it contains an address of ``somewhere'', which is of course not a valid location. This is \textbf{dangerous} We need to initilize a pointer by assigning it a valid address.
\bigbreak \noindent
But suppose we don't know where we are pointing to, we can use a \textbf{null pointer}.
\bigbreak \noindent
\colorbox{lightgray}{nullptr} is a new keyword introduced in C++11. It provides a typesafe pointer value representing an empty pointer.
\bigbreak \noindent
We can use \colorbox{lightgray}{nullptr} like so:
\begin{minted}{c++}
int* ptr = nullptr;
\end{minted}
\nt{
  In older C/C++ code, \colorbox{lightgray}{NULL} was used for this purpose. \colorbox{lightgray}{nullptr} is meant as a modern replacement to \colorbox{lightgray}{NULL}
}

\end{document}

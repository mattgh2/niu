\documentclass{report}

\input{~/latex/template/preamble.tex}
\input{~/latex/template/macros.tex}

\title{\Huge{Multidimensional arrays}}
\author{\huge{Matt Warner}}
\date{\huge{}}
\pagestyle{fancy}
\fancyhf{}
\rhead{}
\lhead{\leftmark}
\cfoot{\thepage}
% \usepackage[default]{sourcecodepro} \usepackage[T1]{fontenc}

\pgfpagesdeclarelayout{boxed}
{
  \edef\pgfpageoptionborder{0pt}
}
{
  \pgfpagesphysicalpageoptions
  {%
    logical pages=1,%
  }
  \pgfpageslogicalpageoptions{1}
  {
    border code=\pgfsetlinewidth{1.5pt}\pgfstroke,%
    border shrink=\pgfpageoptionborder,%
    resized width=.95\pgfphysicalwidth,%
    resized height=.95\pgfphysicalheight,%
    center=\pgfpoint{.5\pgfphysicalwidth}{.5\pgfphysicalheight}%
  }%
}

\pgfpagesuselayout{boxed}

\begin{document}
  \maketitle
  \tableofcontents
  \newpage
  \section{Overview}
  Arrays in C++ may have more than one dimension. Two-dimensional arrays are the most common example, but larger numbers of dimensions are possible
  \section{Declaring 2d arrays}
  When a 2d array is declared, we must specify two different dimensions for the array: the number of rows, and the number of columns in each row.
\begin{minted}{c++}
data-type array-name[number-of-rows][number-of-columns];
\end{minted}
\bigbreak \noindent
As with a 1d array, the number of rows and the number of columns must be specified as integers. You may use an integer literal, a symbolic constant, a variable, a function call that returns an integer, or an expression that resoles to an integer. For example:
\begin{mdframed}
\begin{minted}{c++}
int numbers[2][4];  // 2d array of two rows and
                    // four columns in each row
\end{minted}
\end{mdframed}
\section{Initilizing a 2d array}
A 2d array may be Initilized at the time it is declared.
\begin{mdframed}
\begin{minted}{c++}
int numbers[2][4] = {{1,3,5,7}, {2,4,6,8}};
\end{minted}
\end{mdframed}
\section{Accessing an Element of a 2d array}
To access an individual element of a 2d array, we must code two subscripts - one to specify the row and the other to specify the column within the row
\begin{mdframed}
\begin{minted}{c++}
cout << numbers[1][2] << endl; // Prints column 2 of row 1, the
                               // integer value 6
\end{minted}
\end{mdframed} 
\section{Iterating over the Elements of a 2D array}
We usually use two nested for loops to iterate over the elements of a 2D array. For example, to find the sum of all elements of the array \colorbox{lightgray}{numbers}, we might code the following loops:
\begin{mdframed}
\begin{minted}{c++}
int sum = 0;
for (int row = 0; row < 2; row++)
{

  for (int col = 0; col < 4; col++)
  {
    sum+= numbers[row][col];
  }
}
\end{minted}
\end{mdframed}
\nt{
  We typically loop through the rows of the array in the outer loop and the columns of a row in the inner loop, but it's certainly possible to reverse thatpattern and loop through the columns of the array in the outer loop and then the rows of a column in the inner loop.
}
\section{Passing 2D Arrays to Functions}
Like one dimensional arrays, 2D arrays are \textbf{passed to functions by address} by default. This means that the function will be able to modify the elements of the original array in the calling routine. For example, the following program:
\begin{mdframed}
\begin{minted}{c++}
#include <iostream>

using std::cout;
using std::endl;

void print_array(const int[][4]);
void increment_array(int[][4]);

int main()
{
  int numbers[2][4] = {{1,3,5,7}, {2,4,6,8}};

  print_array(numbers);

  increment_array(numbers);

  print_array(numbers);

  return 0;
}

void print_array(const int array[][4])
{
  for (int row = 0; row < 2; row++)
  {
    for (int col = 0; col < 4; col++)
      cout << array[row][col] << ' ';

    cout << endl;
  }
  cout << endl;
}

void increment_array(int array[][4])
{
  for (int row = 0; row < 2; row++)
    for (int col = 0; col < 4; col++)
    {
      array[row][col] += 10;
    }
}
\end{minted}
\end{mdframed}
will produce the following output:
\begin{mdframed}
  1 3 5 7 \\ 2 4 6 8
  \bigbreak \noindent
  11 13 15 17 \\ 12 14 16 18
  
\end{mdframed}
\end{document}

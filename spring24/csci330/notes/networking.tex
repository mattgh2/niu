\documentclass{report}

\input{~/latex/template/preamble.tex}
\input{~/latex/template/macros.tex}

\title{\Huge{Networking Lecture Notes}}
\author{\huge{Matt Warner}}
\date{\huge{}}
\pagestyle{fancy}
\fancyhf{}
\rhead{}
\lhead{\leftmark}
\cfoot{\thepage}
% \usepackage[default]{sourcecodepro} \usepackage[T1]{fontenc}

\pgfpagesdeclarelayout{boxed}
{
  \edef\pgfpageoptionborder{0pt}
}
{
  \pgfpagesphysicalpageoptions
  {%
    logical pages=1,%
  }
  \pgfpageslogicalpageoptions{1}
  {
    border code=\pgfsetlinewidth{1.5pt}\pgfstroke,%
    border shrink=\pgfpageoptionborder,%
    resized width=.95\pgfphysicalwidth,%
    resized height=.95\pgfphysicalheight,%
    center=\pgfpoint{.5\pgfphysicalwidth}{.5\pgfphysicalheight}%
  }%
}

\pgfpagesuselayout{boxed}

\begin{document}
  \maketitle
  \section{Unit Overview}
  \begin{itemize}
    \item Network concepts \& terminology
    \item OSI reference model for protocols
      \begin{itemize}[label=$\circ$]
        \item Physical layer
        \item Data Link layer
        \item Network layer
        \item Transport layer
      \end{itemize}
  \end{itemize}
  \section{Network Terminology}
\begin{itemize}
  \item Connected graph constructed from
    \begin{itemize}[label=$\circ$]
      \item node
      \item link
    \end{itemize}
  \item Nodes can reach others via path:
    \begin{itemize}[label=$\circ$]
      \item sequence of nodes and links
    \end{itemize}
\end{itemize}
\nt{
  In a network of computers, each computer becomes a node. The connection between nodes are called links.
}
\section{Internet Terminology}
\begin{itemize}
  \item node
    \begin{itemize}[label=$\circ$]
      \item host or intermediary
    \end{itemize}
  \item link
    \begin{itemize}[label=$\circ$]
      \item point-to-point or broadcast
    \end{itemize}
  \item link medium
    \begin{itemize}[label=$\circ$]
      \item wired or wireless
    \end{itemize}
  \item path
    \begin{itemize}[label=$\circ$]
      \item routed or switched
    \end{itemize}
\end{itemize}
\section{Networking Protocol}
\begin{itemize}
  \item communication in a network is governed by rules and conventions
  \item information is exchanged between nodes via messages
  \item messages use well-defined format
  \item each message has an exact meaning intended to provoke a defined response of the receiver
\end{itemize}
\nt{
  A protocol describes the syntax, semantics, and synchronization of communication
}
\newpage
\section{OSI Model}
\begin{itemize}
  \item The OSI model divides rules of networking into 7 layers
\begin{itemize}[label=$\circ$]
  \item Each layer serves a specific function
  \item If all layers are functioning, hosts can share data
\end{itemize}
\end{itemize}
\subsection{OSI reference model}
\begin{figure}[ht]
\centering
\includegraphics[width=0.5\textwidth]{ ~/Documents/figures/OpenSystemIntercommunicationOSI.PNG }
\end{figure}
\section{Layered protocols}
\begin{itemize}
  \item complexities of communication organized into successive layers of protocols
    \begin{itemize}[label=$\circ$]
      \item lower-level layers: specific to medium
      \item higher-level layers: specific to application
    \end{itemize}
  \item standards achieve inter operability
\end{itemize}
\section{OSI reference model layers}
Each of the seven protocol layers are responsible for a share of the communications task between two nodes in the network.
\begin{itemize}
  \item \textbf{Application}: provides services directly to user applications
  \item \textbf{Presentation}: performs data transformations to provide common interface for user applications
  \item \textbf{Session}: establishes, manages and ends user connection
  \item \textbf{Transport}: provides functions to guarantee reliable network link
  \item \textbf{Network}: establishes, maintains and terminates network connections
  \item \textbf{Data link}: ensures the reliability of link
  \item \textbf{Physical}: controls tranmission of the raw bit stream over the medium
\end{itemize}
\subsection{More on Layers}
\textbf{\textit{Layer 1 - Physical - Transporting Bits}} \vspace{1.5mm}

\noindent       Computer data exists in the form of Bits (1's and 0's)
Anything that contributes to moving bits from one computer to another, is considered layer 1 technology. \vspace{1.5mm}

\noindent L1 Technologies: \vspace{1.5mm}

\noindent\textbf{Cables}: Ethernet, Coaxial, Fiber. \vspace{1.5mm}

\noindent \textbf{Repeaters}  \vspace{1.5mm}

\noindent \textbf{Hubs} \vspace{1.5mm}

\noindent \textbf{Wi-Fi} is also considered to be a L1 technology. Wi-Fi solely exisits to cary 1's and 0's from one computer to the next. \vspace{1.5mm}
\bigbreak \noindent
\textit{\textbf{Layer 2 - Data Link}} \vspace{1.5mm}

\noindent Interacts with the Wire (i.e., Physical layer). Puts bits on the wire, and retrieves bits from the wire. \vspace{2.5mm}

\noindent NIC - Network Interface Cards / Wi-FI Access Cards \vspace{2.5mm}

\noindent Layer 2 uses an Addressing Scheme, known as a MAC address
\begin{itemize}
  \item MAC addresses
  \begin{itemize}[label=$\circ$]
    \item 48 bits, represented as 12 hex digits
    \item \textbf{94-65-9C-3B-8A-E5} (windows representation)
    \item \textbf{94:65:9C:3B:8A:E5} (linux representation)
    \item \textbf{9465.9C3B.8AE5} (cisco, routers, and switches)
    \item Every NIC has a unique MAC address
       
  \end{itemize}
\item L2 Technologies: NICs, Switches
\end{itemize}

\bigbreak \noindent
\textit{\textbf{Layer 3 - Network - End to End delivery}} \vspace{2.5mm}

\noindent Layer 3 uses its own Addressing Scheme, IP addresses \vspace{2mm}

\noindent L3 Technologies: Routers, Hosts, (anything with an IP)

\begin{figure}[ht]
\centering
\includegraphics[width=0.5\textwidth]{ ~/Documents/figures/OSIModel.jpg}
\caption*{ \textit{\textbf{Figure 4.2}} OSI reference model layers}
\end{figure}
\section{Physical Layer: Wired Media}
\begin{itemize}
  \item Ethernet (grades below)
    \begin{itemize}[label=$\circ$]
      \item 10BASE-T, 100BASE-TX, 100BASE-T
      \item 10Gbe, 40GbE, 100GbE
    \end{itemize}
  \item Business/backbone
    \begin{itemize}[label=$\circ$]
      \item DS1(T1): 1.54Mbs to DS5: 400Mbs
      \item \textbf{optical circuits:} OC-1: 50Mbs to OC-768: 40Gbs
    \end{itemize}
  \item Last mile:
    \begin{itemize}[label=$\circ$]
      \item Modem
      \item DSL
      \item cable: DOCSIS
      \item FiOS
    \end{itemize}
\end{itemize}
\subsection{Physical Layer: Wireless Media}
\begin{minipage}{0.5\textwidth}
\begin{itemize}
  \item \textbf{Cellphone Data}
    \begin{itemize}[label=$\circ$]
      \item EDGE, GPRS, HSPA+
      \item 4G LTE up to 100MBs
      \item 5G over 100Mbs
    \end{itemize}
  \item \textbf{Satellite}
    \begin{itemize}[label=$\circ$]
      \item Wildblue: 12Mbs
      \item HughesNet: 15Mbs
      \item Starlink: 200Mbs
    \end{itemize}
\end{itemize}
\end{minipage}
\begin{minipage}{0.5\textwidth}
\begin{itemize}
  \item WiFi: 802.11
    \begin{itemize}[label=$\circ$]
      \item up to 150Mbs \& MIMO
      \item new: ``ac'' up to 1Gbs
    \end{itemize}
  \item WiMax: 802.16
    \begin{itemize}[label=$\circ$]
      \item up to 40Mbs
    \end{itemize}
  \item WPAN
    \begin{itemize}[label=$\circ$]
      \item BlueTooth up to 2Mbs
      \item NFC up to 423Kbs
      \item ZigBee up to 256Kbs
    \end{itemize}
\end{itemize}	
\end{minipage}
\newpage
\section{Data Link Layer: functionality}
\begin{itemize}
\item Medium access control
 \begin{itemize}[label=$\circ$]
   \item arbitrate who transmits
 \end{itemize} 
\item Addressing
  \begin{itemize}[label=$\circ$]
    \item address of recevier, address of sender
  \end{itemize}
  \item Framing
    \begin{itemize}[label=$\circ$]
      \item delimited unit of transmission for data \& control
    \end{itemize}
  \item Error control and reliability
  \item Flow control
\end{itemize}
\section{Network Layer}
\begin{itemize}
  \item also called: Internet Protocol Layer
    \begin{itemize}[label=$\circ$]
      \item provides host to host transmission service,
        \subitem where hosts are not necessarily adjacent
    \end{itemize}
  \item layer provides services
    \begin{itemize}[label=$\circ$]
      \item addressing
        \begin{itemize}[label=$\bullet$]
        \item hosts have global addresses: IPv4, IPv6 
        \item uses data link layer protocol to translate address: ARP
        \end{itemize}
      \item routing and forwarding
        \begin{itemize}[label=$\bulllet$]
          \item find path from host to host
        \end{itemize}
    \end{itemize}
\end{itemize}
\section{IPv4 Address}
\begin{itemize}
  \item IP address \hspace{5mm} $\bullet$ 127.0.0.1
    \begin{itemize}[label=$\circ$] 
      \item 32bit unique identifier, written as quad \hspace{5mm} $\bullet$ 131.156.145.90
      \end{itemize}
      \item network
        \begin{itemize}[label=$\circ$]
          \item first n bits of IP number, written as "\textbackslash n" \hspace{5mm}$\bullet$ 131.156.0.0\16
          \item 8 -class A, 16 - class B, 24 - class C \hspace{5mm} $\bullet$ 131.156.145.0/24
          \item more than 24 - class D
        \end{itemize}
      \item netmask
        \begin{itemize}[label=$\circ$]
          \item 32 bit number with first n bits all 1, rest 0 \hspace{5mm} $\bullet$ 255.255.255.0
        \end{itemize}
      \item broadcast
        \begin{itemize}[label=$\circ$]
          \item network number (first n bits), rest all 1  \hspace{5mm} $\bullet$ 131.156.145.255
        \end{itemize}
        \bigbreak \noindent
      \item gateway IP  \ \ \  \ \ \ \ \ \ \ \ \ \ $\bullet$ 131.156.145.1
      \item name server IP \ \ \ \ \ \ \ \ $\bullet$ 131.156.145.2
\end{itemize}
\section{IPv6 Address}
\section{Transport Layer}
\begin{itemize}
  \item Provides end-to-end communication services for applications
  \item byte format as abstraction on underlying system format
  \item raises reliability
  \item enables multiplexing
    \begin{itemize}[label=$\circ$]
      \item provides multiple endpoints on a single node: \underline{port}
    \end{itemize}
\end{itemize}
\end{document}

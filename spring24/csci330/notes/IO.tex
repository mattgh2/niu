\documentclass{report}

\input{~/latex/template/preamble.tex}
\input{~/latex/template/macros.tex}

\title{\Huge{Input \& Output}}
\author{\huge{Matt Warner}}
\date{\huge{}}
\pagestyle{fancy}
\fancyhf{}
\rhead{}
\lhead{\leftmark}
\cfoot{\thepage}
% \usepackage[default]{sourcecodepro} \usepackage[T1]{fontenc}

\pgfpagesdeclarelayout{boxed}
{
  \edef\pgfpageoptionborder{0pt}
}
{
  \pgfpagesphysicalpageoptions
  {%
    logical pages=1,%
  }
  \pgfpageslogicalpageoptions{1}
  {
    border code=\pgfsetlinewidth{1.5pt}\pgfstroke,%
    border shrink=\pgfpageoptionborder,%
    resized width=.95\pgfphysicalwidth,%
    resized height=.95\pgfphysicalheight,%
    center=\pgfpoint{.5\pgfphysicalwidth}{.5\pgfphysicalheight}%
  }%
}

\pgfpagesuselayout{boxed}

\begin{document}
  \maketitle
  \tableofcontents
  \newpage
  \section{UNIX System Call} 
  \begin{itemize}
    \item a system call is how a program prequest a service from the operating system, i.e. UNIX kernal
    \item System call executes code in the kernal and makes direct use of facilities provided by the kernal.
  \end{itemize}
  \bigbreak \noindent
  \textit{\textbf{Versus:}}
  \begin{itemize}
    \item libray function is linked to executable, becomes part of the executable.
  \end{itemize}
  \subsection{System Call Categories}
  \begin{itemize}
    \item \textbf{File management}
      \begin{itemize}[label=$\circ$]
        \item create/delete file, open/close, read/write, get/set attributes.
      \end{itemize}
    \item \textbf{Process control}
      \begin{itemize}[label=$\circ$]
        \item create/terminate process, wait/signal event, allocate/free memory
      \end{itemize}
    \item \textbf{Communication}
      \begin{itemize}[label=$\circ$]
        \item create/delete connection, send/recieve messages, remote devices
      \end{itemize}
    \item \textbf{Information management}
      \begin{itemize}[label=$\circ$]
        \item get/set system data and attributes, e.g. time
      \end{itemize}
    \item \textbf{Device management}
      \begin{itemize}[label=$\circ$]
        \item attach/request/release/detach device, read/write/position.
      \end{itemize}
  \end{itemize}
  \subsection{System Call Invocation} 
  \begin{itemize}
    \item Declare system call via appropriate C header file
      \begin{itemize}[label=$\circ$]
        \item Read man page carefully (section 2)
      \end{itemize}
    \item Prepare parameters using basic C data types
      \begin{itemize}[label=$\circ$]
        \item No c++ classes, i.e. string
      \end{itemize}
    \item Prepare suitable return value variable
  \end{itemize}
  \textit{\textbf{Note: Call like any other function}}
  \section{File Management}
  \begin{itemize}
    \item \textbf{open} \hspace{6.5mm} open a file
    \item \textbf{read} \hspace{6.5mm} read data from a file
    \item \textbf{write} \hspace{5.5mm} write data to a file
    \item \textbf{close} \hspace{6.5mm} close a file
    \item \textbf{creat} \hspace{6.5mm} make a new file
    \item \textbf{unlink} \hspace{5mm} remove file
    \item \textbf{stat} \hspace{10mm} remove file
    \item \textbf{chmod} \hspace{5mm} change permissions
    \item \textbf{dup} \hspace{10mm} duplicate file descriptor
  \end{itemize}
  \textit{\textbf{All calls share \underline{file descriptor}, i.e. number, to identify file}}
  \newpage
  \subsection{System Call: open}
  \bigbreak \noindent
  \textbf{\textit{Syntax:}}
  \begin{minted}{bash}
  int open(const char* pathname, int flags)
  \end{minted} 
  \begin{itemize}
    \item opens file specified as \textbf{pathname} for access
    \item \textbf{flags} determine access types
      \subitem\textbf{O\_RDONLY} \ \ read only
      \subitem \textbf{O\_WRONLY} \ \ write only
      \subitem \textbf{O\_RDWR} \ \ read and write
    \item returns file descriptor, to be used in read/write/close
    \item returns -1 on error
      \bigbreak \noindent
    \item additional \textbf{flags,} used with \textbf{O_WRONLY}
        \subitem \textbf{O\_APPEND} \ \ to append to an existing file
        \subitem \textbf{O\_TRUNC} \ \ existing file will overwritten (default)
  \end{itemize}
  \textit{\textbf{Ex:}}

  \textbf{O\_WRONLY} | \textbf{O\_TRUNC} \vspace{1.5mm}
    
    \textbf{O\_WRONLY | O\_APPEND}
    \begin{itemize}
      \item additional flag: \textbf{O\_CREAT}
        \begin{itemize}[label=$\circ$]
          \item creates file, if file does not exist
        \end{itemize}
    \end{itemize}
\subsection{System Call: read}
\bigbreak \noindent
\textbf{\textit{Syntax:}}
\begin{minted}{bash}
ssize_t read(int fd, void *buf, size_t count)
\end{minted}
\begin{itemize}
  \item attempts to read \textbf{count} bytes from file descriptor \textbf{fd} into the buffer starting at \textbf{buf}
    \begin{itemize}[label=$\circ$]
      \item \textbf{ssize\_t} and \textbf{size\_t} are system specific integer types
    \end{itemize}
  \item returns the number of bytes read
    \begin{itemize}[label=$\circ$]
      \item maybe smaller than count, zero indicates end of file
      \item file position is advanced by this number
    \end{itemize}
  \item returns -1 on error
\end{itemize}
\subsection{System Call: close}
\bigbreak \noindent
\textbf{\textit{Syntax:}}
\begin{minted}{bash}
int close(int fd)
\end{minted}
\begin{itemize}
  Closes file specified by \textbf{fd} file descriptor
  \begin{itemize}[label=$\circ$]
  Makes file descriptor available
  \end{itemize}
  \item returns zero on success
\end{itemize}
\subsection{System Call: write}
\bigbreak \noindent
\textbf{\textit{Syntax:}}
\begin{minted}{bash}
ssize_t write(int fd, const void *buf, size_t count)
\end{minted}
\begin{itemize}
  \item Writes up to count bytes from buffer starting at \textbf{buf} to the file referred to by file descriptor \textbf{fd}
    \begin{itemize}[label=$\circ$]
      \item \textbf{ssize\_t} and \textbf{size\_t} are system specific integer types
    \end{itemize}
  \item Returns the number of bytes written
    \begin{itemize}[label=$\circ$]
      \item may be smaller than count
    \end{itemize}
  \item returns -1 on error
\end{itemize}
\newpage
\section{System Call: open with O\_CREAT}
\begin{itemize}
\item must specify file mode, i.e. permissions, of type \textbf{mode\_t}
  \subitem \textbf{S\_IRWXU} \ \ 00800 \ \ user has read, write and execute permission
  \subitem \textbf{S\_IRUSR} \ \ 00400 \ \ user has read permission
  \subitem \textbf{S\_IWUSR} \ \ 00200 \ \ user has write permission
  \subitem \textbf{S\_IXUSR} \ \ 00100 \ \ user has execute permission
  \subitem \textbf{S\_IRWXG} \ \ 00070 \ \ group has read, write and execute permission
  \subitem \textbf{S\_IRWXO} \ \ 00007 \ \ others have read, write and execute permission
\item C++ requires leading '0' for octal numbers
\item actual file mode is further modified by umask
\end{itemize}
\bigbreak \noindent
\textbf{\underline{Example:}}
\begin{minted}{c++}
open("ex.txt", O_WRONLY | O_APPEND | O_CREAT, 00666)
\end{minted}
\subsection{System Call: creat}
\bigbreak \noindent
\textbf{\textit{Syntax:}}
\begin{minted}{bash}
int creat(const char *pathname, mode_t mode)
\end{minted}
\begin{itemize}
  \item creates new file specified as \textbf{pathname} and opens file for write access
  \item returns file descriptor
  \item returns -1 on error
\end{itemize}
\begin{minted}{c++}
#include <sys/types.h>
#include <sys/stat.h>
#include <fcntl.h>
#include <iostream>

int main(int argc, char** argv) {

  int ifd, ofd, count, sum=0;
  char buffer[1024];

  // check command line args
  if (argc < 3) {
    std::cerr << "Usage: copy fromFile toFile\n";
    exit(EXIT_FAILURE);
  }

  // open file to read
  ifd = open(argv[1], O_RDONLY);
  if(ifd == -1) {
    perror(argv[1]);
    exit(EXIT_FAILURE);
  }

  // open file to write
  ofd = open(argv[2], O_WRONLY | O_TRUNC | O_CREAT, 00666);
  
  // loop to read/write buffer
  while ((count = read(ifd, buffer, sizeof(buffer))) != 0) {
    if ((count == -1)) {
      perror(argv[1]);
      exit(EXIT_FAILURE)
    }
    count = write(ofd, buffer, count);
    if (count == -1) {
      perror(argv[2]);
      exit(EXIT_FAILURE)
    }
  }
  cout << "copied " << sum << "bytes from " << argv[1] << " to " << argv[2] << endl;
  // close all files
  close(ifd);
  close(ofd);
}
\end{minted}
\end{document}

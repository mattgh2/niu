\documentclass{report}

\input{~/latex/template/preamble.tex}
\input{~/latex/template/macros.tex}

\title{\Huge{Midterm Review}}
\author{\huge{Matt Warner}}
\date{\huge{}}
\pagestyle{fancy}
\fancyhf{}
\rhead{}
\lhead{\leftmark}
\cfoot{\thepage}
% \usepackage[default]{sourcecodepro} \usepackage[T1]{fontenc}

\pgfpagesdeclarelayout{boxed}
{
  \edef\pgfpageoptionborder{0pt}
}
{
  \pgfpagesphysicalpageoptions
  {%
    logical pages=1,%
  }
  \pgfpageslogicalpageoptions{1}
  {
    border code=\pgfsetlinewidth{1.5pt}\pgfstroke,%
    border shrink=\pgfpageoptionborder,%
    resized width=.95\pgfphysicalwidth,%
    resized height=.95\pgfphysicalheight,%
    center=\pgfpoint{.5\pgfphysicalwidth}{.5\pgfphysicalheight}%
  }%
}

\pgfpagesuselayout{boxed}

\begin{document}
  \maketitle
\section{Topics covered so far}
\begin{itemize}
  \item Introduction 
  \item File system
  \item Editors
  \item Net Ultilities
  \item Permissions
  \item Shell
  \item Shell scripts
  \item awk report generator
  \item sed stream editor
\end{itemize}
\section{History of UNIX}
\begin{itemize}
  \item Invented by Ken Thomson at Bell Labs in 1969 
    \begin{itemize}[label=$\bullet$]
      \item First version written in assembly language 
      \item single user system, no network capability
    \end{itemize}
  \item Thompson, Dennis Ritchie, Brian Kernighan
    \begin{itemize}[label=$\bullet$]
      \item Rewrote Unix in C 
    \end{itemize}
  \item Unix evolution:
    \begin{itemize}[label=$\bullet$]
      \item Bell Labs, USL, Novell, SCO  \hspace{30mm}$\bullet$ Newest:
        \begin{itemize}[label=$\bullet$]
          \item BSD, FreeBSD, Mach, OS X \hspace{30mm} $\bullet$ Linux on portables: Android
          \item AIX, Ultrix, Irix, Solaris, \ldots
        \end{itemize}
      \item Linux: Linus Torvalds
    \end{itemize}
\end{itemize}
\section{Command Line Structure}
\bigbreak \noindent
\textbf{\textit{Syntax:}}
\$ command [-options] [arguments]
\begin{itemize}
  \item UNIX is case sensitive
  \item Must be a space between command, options and arguments
  \item Fields enclosed in [ ] are optional
\end{itemize}
\nt{
  The brackets as used to show that they are optional
}
\newpage
\section{RTFM: The man Command}
  Shows pages from system manual 
\bigbreak \noindent
\begin{minipage}{0.5\textwidth}
\textbf{\textit{Syntax:}}
\begin{minted}{bash}
$ man date
$ man -k date
$ man crontab
$ man -S 5 crontab
\end{minted}
\end{minipage}
\begin{table}[h!]
  \centering
\begin{tabular}{|m{2cm}|m{6cm}|}
\hline
\textbf{Section} & \textbf{Description} \\ \hline
1                & User commands        \\ \hline
2                & System calls         \\ \hline
3                & C library functions  \\ \hline
4                & Special system files \\ \hline
5                & File formats         \\ \hline
6                & Games                \\ \hline
7                & Misc. features       \\ \hline
8                & System admin utilities \\ \hline
\end{tabular}
\end{table}
\bigbreak \noindent
\textit{\textbf{Caveats:}} \vspace{1.5mm}

\noindent Some commands are aliases \vspace{1.5mm}

\noindent Some commands are part of the shell
\section{Path}
\begin{itemize}
  \item path: list of names seperated  by ``/''
  \item \textbf{Absolute Path}
    \begin{itemize}[label=$\circ$]
      \item Traces a path from root to a file or a directory
      \item Always begins with the root (/) directory \vspace{1mm}
        
       \noindent \textbf{\underline{Example:}} /home/student/Desktop/assign1.txt
    \end{itemize}
  \item \textbf{Relative Path}
    \begin{itemize}[label=$\circ$]
      \item Traces a path from the current directory
      \item No inital forward slash (/)
        \begin{itemize}[label=$\circ$]
          \item dot (.) refers to current directory
          \item (..) referes to one level up in the directory hierarchy \vspace{1.5mm}

       \noindent \textbf{\underline{Example:}} Desktop/assign1.txt
        \end{itemize}
    \end{itemize}
\end{itemize}
\section{Linking Files}
\begin{itemize}
  \item Allows one file to be known by different names 
  \item Link is a reference to a file stored elsewhere
    \bigbreak \noindent
  \item 2 types:
    \begin{itemize}[label=$\circ$]
      \item Hard link (default) 
      \item Symbolic link (a.k.a. ``soft link'')
    \end{itemize}
\end{itemize}
\bigbreak \noindent
\textbf{\textit{Syntax:}}
\begin{minted}{bash}
ln [-s] target local
\end{minted}
\newpage
\section{User's Disk Quota}
\begin{itemize}
  \item Quota is upper limit of \hspace{20mm} $\bullet$ 2 Kinds of limits:
    \begin{itemize}[label=$\circ$]
      \item amount of disk space  \hspace{20mm} $\circ$ Soft limit: ex. 100MB (system will only complain)
      \item number of files \hspace{30mm} $\circ$ Hard limit ex. 120MB (Cannot be exceeded)
    \end{itemize}
    For each user account
    \bigbreak \noindent
  \item \textbf{command:} quota -v
    \begin{itemize}[label=$\circ$]
      \item displays the user's disk usage and limits
    \end{itemize}
\end{itemize}
\section{Unix Text editors}
\begin{itemize}
  \item Vi or vim 
  \item pico, nano
  \bigbreak \noindent
\item GUI editors
  \begin{itemize}[label=$\circ$]
    \item gedit or ``text Editor'' 
    \item geany
    \item emacs
    \item VS code
  \end{itemize}
\end{itemize}
\section{Networking Utilities}
\begin{itemize}
  \item Most network protocols were developed on UNIX 
  \item Most (not all) UNIX systems provided them:
    \bigbreak \noindent
  \item login to another computer
    \begin{itemize}[label=$\circ$]
    \item telnet, rlogin, rsh, ssh 
  \end{itemize}
\item copy files to another computer
  \begin{itemize}[label=$\circ$]
    \item rcp, spc 
    \item ftp, sftp
  \end{itemize}
\item Linux desktop provides GUI-enabled tools
  \begin{itemize}[label=$\circ$]
    \item file manager 
  \end{itemize}
\end{itemize}
\section{Special Permissions}
\begin{itemize}
  \item The regular file permissions (rwx) are used to assign security to files and directories
  \item 3 additional special permissions can be optionally used on files and directories
    \begin{itemize}[label=$\circ$]
    \item Set User Id (SUID) (allows programs to run with the permissions of the owner of the file)
    \item Set Group ID (SGID) (allows programs to run with the permissions of the group of the file)
    \item Sticky bit (makes sure you can only delete files that you actually own)
    \end{itemize}
\end{itemize}
\newpage
\section{Permissions}
\section{File mode creation mask}
\begin{itemize}
  \item umask (user mask)
    \begin{itemize}[label=$\circ$]
      \item governs default permissions for files and directories
      \item sequence of 9 bits: 3 times 3 bits of rwx
      \item default 000 010 010 (022)
    \end{itemize}
  \item in octal form its bits are removed from:
    \begin{itemize}[label=$\circ$]
      \item for a file: \hspace{10mm} 110 110 110 (666)
      \item for a directory: \hspace{1mm} 111 111 111 (777)
    \end{itemize}
  \item new permissions
    \begin{itemize}[label=$\circ$]
      \item file: 110 100 100 (644)
      \item directory: 111 101 101 (755)
    \end{itemize}
\end{itemize}
\section{UNIX Command Interpreters}
\textbf{Common Term: Shell}
\begin{itemize}
  \item Standard:  
    \begin{itemize}[label=$\circ$]
      \item every UNIX system has a ``Bourne shell compatible'' shell
    \end{itemize}
  \item history:
    \begin{itemize}[label=$\circ$]
      \item sh: Original Bourne shell, written in 1978 by Steve Bourne
      \item ash: Almquist shell, BSD-licensed replacement of sh
    \end{itemize}
  \item Today:
    \begin{itemize}[label=$\circ$]
      \item bash: Bourne again shell, GNU replacement of sh
      \item dash: Debian Almquist shell, small scripting shell
    \end{itemize}
\end{itemize}
\section{Bash shell basics}
\begin{itemize}
  \item Customization
\begin{itemize}[label=$\circ$]
  \item startup and initilization
  \item variables, prompt and aliases
\end{itemize}
\item Command line behavior
  \begin{itemize}[label=$\circ$]
    \item history
    \item sequence  (;)
    \item substitution (\`\`)
    \item redirections and pipe
  \end{itemize}
\end{itemize}
\newpage
\section{Redirections and Pipe}
\begin{table}[h]
\centering
\begin{tabular}{@{}ll@{}}
\toprule
Command Syntax       & Meaning                                                   \\ \midrule
\texttt{command < file}         & redirect input from \texttt{file} to \texttt{command}                 \\
\texttt{command > file}         & redirect output from \texttt{command} to \texttt{file}                \\
\texttt{command >> file}        & redirect output of \texttt{command} and appends it to \texttt{file}   \\
\texttt{command 2> file}        & add error output to standard output, redirect both into \texttt{file} \\
\texttt{command 2>\&1}          & redirect both standard output and standard error to \texttt{file}     \\
\texttt{command1 | command2}    & take/pipe output of \texttt{command1} as input to \texttt{command2}   \\
\texttt{command <\& label}      & take input from current source until \texttt{LABEL} line              \\ \bottomrule
\end{tabular}
\end{table}
\section{Wildcards: * ? [ ] \{ \}}
A pattern of special characters used to match file names on the command line
\bigbreak \noindent
\begin{minipage}{0.5\textwidth}
* \ \ zero or more characters	\\ 
\$ rm *  \\ 
\$ ls *.txt \\
\$ wc -l assign.* \\
\$ cp a*.txt docs
\end{minipage}
\begin{minipage}{0.5\textwidth}
? \ exactly one character (can be chained together)	 \\
\$ ls assign?.cc \\ 
\$ wc assign?.?? \\
\$ rm junk.???
\end{minipage}
\section{Regular Expressions}
\begin{itemize}
  \item A pattern of special characters to match strings in a search 
  \item Typically made up from special characters called meta-characters: . * + ? [ ] \{ \} ( )
  \item Regular expressions are used throughout UNIX:
    \begin{itemize}[label=$\circ$]
      \item utilities: grep, awk, sed, \ldots
    \end{itemize}
  \item 2 types of regular expressions: basic vs. extended
\end{itemize}
\section{Metacharacters}
. \hspace{10mm} Any one character, except new line \vspace{2mm}

\noindent [a-z]\hspace{6mm} Any one of the enclosed characters (e.g. a-z) \vspace{2mm}

\noindent * \hspace{10mm} Zero or more preceding characters \vspace{2mm}

\noindent ? \ also \ \textbackslash ? \hspace{2mm} Zero or one of the preceding characters \vspace{2mm}

\noindent + \ also: \textbackslash +  \hspace{2mm} One or more of the preceding characters \vspace{2mm}

\noindent \textasciicircum  \ or  \$ \hspace{10mm} Beginning or end of line \vspace{2mm}

\noindent \textbackslash $<$ or \textbackslash $>$ \hspace{10mm} Beginning or end of word \vspace{2mm}

\noindent ( ) \ also: \ \textbackslash( \textbackslash) \hspace{3mm} Groups matched characters to be used later \vspace{2mm}

\noindent $|$ also: \textbackslash$|$ \hspace{10mm} Alternate\vspace{2mm}

\noindent x\{m,n\} also: \ x\textbackslash\{m,n\textbackslash\} \hspace{2mm} Repetition of character x between m and n times
\newpage
\section{Quoting and Escaping}
Allows to distinguish between the literal value of a symbol and the symbols used as meta-characters
\bigbreak \noindent
Done via the following symbols:
\begin{itemize}
  \item Backslash (\textbackslash)
  \item Single quote (')
  \item Double quote (")
\end{itemize}
\section{Shell Script: the basics}
\begin{itemize}
  \item 1. line for shell script
    \subitem \#! /bin/bash
    \subitem or: \#! /bin/sh
  \item to run:
    \subitem \$ bash script
  \item or:
    \begin{itemize}[label=$\circ$]
      \item make executable: \hspace{5mm} \$ chmod +x script
      \item invoke via: \hspace{15mm} \$ ./script
    \end{itemize}
\end{itemize}
\section{bash shell Programming Features}
\begin{itemize}
  \item Variables
    \begin{itemize}[label=$\circ$]
      \item string, number, array
    \end{itemize}
  \item Input/Output
    \begin{itemize}[label=$\circ$]
      \item echo, printf
      \item command line arguments, read from user
    \end{itemize}
  \item Decision 
    \begin{itemize}[label=$\circ$]
      \item conditional execution, if-then else
    \end{itemize}
  \item Repetition
    \begin{itemize}[label=$\circ$]
      \item while, until, for
    \end{itemize}
  \item Functions
\end{itemize}
\section{Command line arguments}
\hspace{15mm} \underline{Meaning} \vspace{1.5mm}

\noindent \$1 \ \ \ \ \ \ \ \ \ \ first parameter \vspace{1.5mm}

\noindent \$2 \ \ \ \ \ \ \ \ \ \ second parameter \vspace{1.5mm}

\noindent \${10} \ \ \ \ \ \ \ \ 10th parameter
\bigbreak \noindent
\$0 \ \ \ \ \ \ \ \ \ \ Name of the script \vspace{1.5mm}

\noindent \$* \ \ \ \ \ \ \ \ \ \ all positional parameters \vspace{1.5mm}

\noindent \$\# \ \ \ \ \ \ \ \ \ The number of arguments
\newpage
\section{User input: read command}
\bigbreak \noindent
\textbf{\textit{Syntax:}}
\begin{minted}{bash}
read [-p "prompt"] varname [more vars] 
\end{minted}
\nt{
Once again the brackets are for stuff that is optional
}
\begin{itemize}
  \item words entered by user are assigned to
    \begin{minted}{bash}
    varname and "more vars"
    \end{minted}
  \item last variable gets rest of input line
\end{itemize}
\section{test command}
\bigbreak \noindent
\textbf{\textit{Syntax:}}
\begin{minted}{bash}
test expression
[ expression ]
\end{minted}
\begin{itemize}
  \item evaluates 'expression' and returns true or false
\end{itemize}
\bigbreak \noindent
\textbf{\underline{Example:}}
\begin{mdframed}
\begin{minted}{bash}
if test $name = "Joe"
then
  echo "Hello Joe"

if [ $name = "Joe" ]; then
  echo "Hello Joe"
\end{minted}
\end{mdframed}
\section{Shell scripting features}
\begin{itemize}
  \item How to debug?
  \item Decision
    \begin{itemize}[label=$\circ$]
      \item case
    \end{itemize}
  \item Repetition
    \begin{itemize}[label=$\circ$]
      \item while, until
      \item for
    \end{itemize}
  \item Functions
\end{itemize}
\section{What can you do with awk?}
\begin{itemize}
  \item awk operation: 
    \begin{itemize}[label=$\circ$]
      \item reads a file line by line 
      \item splits each input line into fields
      \item compares input line/fields to pattern
        performs action(s) on matched lines
    \end{itemize}
  \item Useful for:
    \begin{itemize}[label=$\circ$]
      \item transforming data files 
      \item producing formatted reports
    \end{itemize}
  \item Programming constructs
    \begin{itemize}[label=$\circ$]
      \item format output lines 
      \item arithmetic and string operations
      \item conditionals and loops
    \end{itemize}
\end{itemize}
\section{Basic awk script}
\begin{itemize}
  \item consists of patterns and actions
\end{itemize}
\bigbreak \noindent
\textbf{\textit{Syntax:}}
\begin{minted}{bash}
pattern {action}
\end{minted}
\begin{itemize}
  \item if pattern is missing, action is applied to all lines
  \item if action is missing, the matched line is printed
  \item must have either pattern or action
\end{itemize}
\bigbreak \noindent
\textbf{\underline{Example:}}
\begin{mdframed}
\begin{minted}{bash}
awk '/for/ { print }' testfile
\end{minted}
\end{mdframed}
\begin{itemize}
\item prints all lines containing string ``for'' in testfile
\end{itemize}
\section{awk variables}
awk reads input line into buffer: \underline{record} and \underline{fields}
\begin{itemize}
  \item field buffer:
    \begin{itemize}[label=$\circ$]
      \item one for each field in the current record
      \item variable names: \$1, \$2, \ldots
    \end{itemize}
  \item record buffer
    \begin{itemize}[label=$\circ$]
      \item \$0 holds the entire record
    \end{itemize}
\end{itemize}
\bigbreak \noindent
\begin{itemize}
  \item NR \ \ \ Number of the current record
  \item NF \ \ \ Number of fields in the current record
  \item FS \ \ \ Field separator (default=whitespace)
\end{itemize}
\section{awk Patterns}
\begin{itemize}
  \item simple patterns
    \begin{itemize}[label=$\circ$]
      \item BEGIN, END
      \item expression patterns: whole line vs. explicit field match
       \begin{itemize}[label=$\circ$]
         \item whole line \ \ \ \ /regExp/
         \item field match \ \ \ \ \$2 \textasciitilde /regExp
       \end{itemize} 
    \end{itemize}
  \item range patterns
  \begin{itemize}[label=$\circ$]
    \item specified as from and to:
      \subitem Example:  \ \ \ \ \ \ /regExp/,/regExp/
  \end{itemize}
\end{itemize}
\newline
\section{awk Patterns}
\begin{itemize}
  \item basic expressions
    \bigbreak \noindent
  \item output: \ \ \ \ print, printf
  \item decisions: \ \ \ \ if
  \item loops: \ \ \ \ for, while
\end{itemize}
\section{How Does sed Work?}
\begin{itemize}
  \item sed reads file line by line
    \begin{itemize}[label=$\circ$]
      \item line of input is copied into a temporary buffer called \underline{pattern} space
      \item editing instructions are applied to line in the pattern space
      \item line is sent to output (unless "-n" option was used)
      \item line is removed from pattern space
    \end{itemize}
  \item sed reads next line of input, until end of file
\end{itemize}
\nt{
  input file is unchanged unless ``-i'' option is used
}
\section{sed instruction format}
\begin{itemize}
  \text address determines which lines in the input file are to be processed by the commands(s)
  \begin{itemize}[label=$\circ$]
    \item if no address is given, then the command is applied to each input line
  \end{itemize}
\item address types:
  \begin{itemize}[label=$\circ$]
    \item Single-Line address
    \item Set-of-Lines address
    \item Range address
  \end{itemize}
\end{itemize}
\end{document} 

\documentclass{report}

\input{~/latex/template/preamble.tex}
\input{~/latex/template/macros.tex}

\title{\Huge{Introduction to awk}}
\author{\huge{Matt Warner}}
\date{\huge{}}
\pagestyle{fancy}
\fancyhf{}
\rhead{}
\lhead{\leftmark}
\cfoot{\thepage}
% \usepackage[default]{sourcecodepro} \usepackage[T1]{fontenc}
\usetikzlibrary{matrix,shapes,arrows,positioning,chains}
\pgfpagesdeclarelayout{boxed}
{
  \edef\pgfpageoptionborder{0pt}
}
{
  \pgfpagesphysicalpageoptions
  {%
    logical pages=1,%
  }
  \pgfpageslogicalpageoptions{1}
  {
    border code=\pgfsetlinewidth{1.5pt}\pgfstroke,%
    border shrink=\pgfpageoptionborder,%
    resized width=.95\pgfphysicalwidth,%
    resized height=.95\pgfphysicalheight,%
    center=\pgfpoint{.5\pgfphysicalwidth}{.5\pgfphysicalheight}%
  }%
}

\pgfpagesuselayout{boxed}

\begin{document}
  \maketitle
  \tableofcontents
  \clearpage
  \section{What is awk?}  
  \begin{itemize}
    \item Created by: 
      \subitem Aho, Weinberger and Kernighan.
    \item awk is a scripting language used for manipulating data and generating reports
    \item Versions of awk:
      \subitem awk, nawk, mawk, pgawk, \ldots
    \item GNU awk: gawk
  \end{itemize}
\section{What can you do with awk?}
\begin{itemize}
  \item awk operation: 
    \begin{itemize}[label=$\circ$]
      \item reads a file line by line 
      \item splits each input line into fields
      \item compares input line/fields to pattern
        performs action(s) on matched lines
    \end{itemize}
  \item Useful for:
    \begin{itemize}[label=$\circ$]
      \item transforming data files 
      \item producing formatted reports
    \end{itemize}
  \item Programming constructs
    \begin{itemize}[label=$\circ$]
      \item format output lines 
      \item arithmetic and string operations
      \item conditionals and loops
    \end{itemize}
\end{itemize}
\section{Basic awk invocation}
\bigbreak \noindent
\textbf{\textit{Syntax:}}
\begin{minted}{bash}
awk 'script' file(s)

awk -f scriptfile file(s) 
\end{minted}
\bigbreak \noindent
   Common option: -F (to change the field seperator)
   \section{Basic awk script}
   \begin{itemize}
     \item consists of patterns \& actions: 
   \end{itemize}
   \bigbreak \noindent
   \textbf{\textit{Syntax:}}
   \begin{minted}{bash}
   pattern {action}
   \end{minted}
   \bigbreak \noindent
   \begin{itemize}
     \item If pattern is missing, action is applied to all lines 
     \item if action is missing, the matched line is printed
     \item Must have either pattern or action
   \end{itemize}
   \bigbreak \noindent
   \textbf{\underline{Example:}}
   \begin{minted}{bash}
    # print all lines containing string "for" in testfile

    awk '/for/ { print }' testfile
   \end{minted}
   \newpage \noindent
   \section{awk variables}
   awk reads input line into buffers \textit{\textbf{record}} and \textit{\textbf{fields}}
   \begin{itemize}
     \item field buffer: 
       \begin{itemize}[label=$\bullet$]
        \item One for each field in the current record 
        \item Variable names: \$1, \$2, \ldots
       \end{itemize}
     \item record buffer:
       \begin{itemize}[label=$\bullet$]
         \item \$0 holds the entire record
       \end{itemize}
   \end{itemize}
   \subsection{More variables}
  NR \ \ \ Number of the current record \vspace{1.5mm}

  \noindent  NF \ \ \ Number of fields in the current record \vspace{1.5mm}
  
 \noindent \textit{\textbf{Also:}} \vspace{2mm}

  \noindent FS \ \ \ Field sperator (default=whitespace)
\bigbreak \noindent
\textbf{\underline{Example:}}
\bigbreak \noindent
Say we had a file called emps:
\begin{minted}{bash}
~$ cat emps
Tom Jones     4424     5/12/66    543354
Mary Adams    5346     11/4/63    28765
Sally Chang   1654     7/22/54    6500000
Billy Black   1683     9/23/44    336500

$ awk '/Tom/ { print $0 }' emps
Tom Jones     4424     5/12/66    543354

$ awk '{ print NR, $0 }' emps
1 Tom Jones     4424     5/12/66    543354
2 Mary Adams    5346     11/4/63    28765
3 Sally Chang   1654     7/22/54    6500000
4 Billy Black   1683     9/23/44    336500
\end{minted}
\subsection*{Example: Space as Field Separator}
\begin{mdframed}
  \begin{minted}{bash}
~$ cat emps
Tom Jones     4424     5/12/66    543354
Mary Adams    5346     11/4/63    28765
Sally Chang   1654     7/22/54    6500000
Billy Black   1683     9/23/44    336500

awk '{ print NR, $1, $2, $5 }' emps

1 Tom Jones     543354
2 Mary Adams    28765
3 Sally Chang   6500000
4 Billy Black   336500
\end{minted}
\end{mdframed}
\newpage
\subsection*{Example: Colon as Field Seperator}
\begin{minted}{bash}
~$ cat emps2
Tom Jones:4424:5/12/66:543354
Mary Adams:5346:11/4/63:28765
Sally Chang:1654:7/22/54:650000
Billy Black:1683:9/23/44:336500

awk -F: '/Jones/{ print $1, $2 }' emps2
Tom Jones 4424

awk -F: '/Jones/{print $1, $3}' emps2
Tom Jones 5/12/66
\end{minted}
\subsection*{Example: File Processing}
\$ cat input
\bigbreak \noindent
Jan & 13 & 25 & 15 & 115 \\
Feb & 15 & 32 & 24 & 22 \\
Mar & 15 & 24 & 34 & 228 \\
Apr & 31 & 52 & 63 & 420 \\
May & 16 & 34 & 29 & 208 \\
Jun & 31 & 42 & 75 & 492 \\
Jul & 24 & 34 & 67 & 436 \\
Aug & 15 & 34 & 47 & 316 \\
Sep & 15 & 53 & 67 & 277 \\
Oct & 29 & 54 & 68 & 525 \\
Nov & 20 & 87 & 82 & 577 \\
Dec & 17 & 35 & 61 & 401 \\
Jan & 21 & 36 & 64 & 620 \\
Feb & 26 & 58 & 80 & 652 \\
Mar & 24 & 75 & 70 & 495 \\
Apr & 21 & 70 & 74 & 514 \\
\begin{minted}{bash}
# prints all records
awk '{print}' input

# prints only first field of each record
awk '{print $1}' input
\end{minted}
\section{Simple Patterns}
\begin{itemize}
  \item BEGIN 
    \begin{itemize}[label=$\bullet$]
      \item Matches before the first line of input 
      \item Used to create header for report
    \end{itemize}
  \item END
    \begin{itemize}[label=$\bullet$]
      \item Matches after the last line of input 
      \item Used to create footer for report
    \end{itemize}
\end{itemize}
\subsection{More Patterns}
\begin{itemize}
  \item expression patterns: whole line vs. explicit field match
    \begin{itemize}[label=$\bullet$]
      \item whoe line \ \ \ \ \ \ \ \ /regExp/ 
      \item field match \ \ \ \$2 \textasciitilde /regExp/
    \end{itemize}
  \item range patterns
    \begin{itemize}[label=$\bullet$]
      \item specified as from and to: 
        \subitem example: \ \ \ \ /regExp/,/regExp/
    \end{itemize}
\end{itemize}
\bigbreak \noindent
\textbf{\underline{Example:}}
\begin{mdframed}
\$ cat input \\
Jan & 13 & 25 & 15 & 115 \\
Feb & 15 & 32 & 24 & 22 \\
Mar & 15 & 24 & 34 & 228 \\
Apr & 31 & 52 & 63 & 420 \\
May & 16 & 34 & 29 & 208 \\
Jun & 31 & 42 & 75 & 492 \\
Jul & 24 & 34 & 67 & 436 \\
Aug & 15 & 34 & 47 & 316 \\
Sep & 15 & 53 & 67 & 277 \\
Oct & 29 & 54 & 68 & 525 \\
Nov & 20 & 87 & 82 & 577 \\
Dec & 17 & 35 & 61 & 401 \\
Jan & 21 & 36 & 64 & 620 \\
Feb & 26 & 58 & 80 & 652 \\
Mar & 24 & 75 & 70 & 495 \\
Apr & 21 & 70 & 74 & 514 \\
\begin{minted}{bash}
$ cat script
BEGIN {
  print "January Sales Revenue"
}
$1 ~ /Jan/ {
  print $1, $2+$3+$4, $5
}
END {
  print NR, " records processed"
}
# invoking script
awk -f script input

#OUTPUT
January Sales Revenue
Jan 53 115
Jan 121 620
16 records processed
\end{minted}
\end{mdframed}
\newpage
\section{awk actions}
\begin{itemize}
  \item basic expressions 
  \item output: print, printf
  \item decisions: if
  \item loops: for, while
\end{itemize}
\section{awk Expressions}
Consists of: \textbf{operands} and \textbf{operators} 
\begin{itemize}
  \item operands:
    \begin{itemize}[label=$\bullet$]
      \item numeric and string constants 
      \item variables
      \item functions and regular expressions
    \end{itemize}
  \item operators:
    \begin{itemize}[label=$\bullet$]
      \item assignment: = \ ++ \ -- \ += \ -= \ *= \ /=
      \item arithmetic: + \ - \ * \ \% \ \textasciicircum
      \item logical: \&\& \ $\|$ \ !
      \item relational: $>$ \ $<$ \ $>=$ \ $<=$ \ $==$ \ $!=$
      \item match: \textasciitilde \ \ \ !\textasciitilde
      \item string concatenation: space
    \end{itemize}
\end{itemize}
\section{awk Variables}
\begin{itemize}
  \item Created via assignment 
\end{itemize}
\bigbreak \noindent
\textbf{\textit{Syntax:}}
\begin{minted}{bash}
var = expression
\end{minted}
\begin{itemize}
  \item types: number (not limited to integer) 
  \item variables come into existence when first used
  \item type of variable depends on its use
  \item variables are initilized to either o or ``''
\end{itemize}
\bigbreak \noindent
\textbf{\underline{Example:}}
\begin{minted}{bash}
cat input 
\end{minted}
Jan & 13 & 25 & 15 & 115 \\
Feb & 15 & 32 & 24 & 22 \\
Mar & 15 & 24 & 34 & 228 \\
Apr & 31 & 52 & 63 & 420 \\
May & 16 & 34 & 29 & 208 \\
Jun & 31 & 42 & 75 & 492 \\
Jul & 24 & 34 & 67 & 436 \\
Aug & 15 & 34 & 47 & 316 \\
Jan & 21 & 36 & 64 & 620 \\
\newpage
\begin{mdframed}
  \begin{minted}{bash}
cat script
BEGIN {
  print "January Sales Revenue"
  count = 0
  sum = 0
}
$1 ~ /Jan/ && NF == 5 {
  print $1, $2+$3+$4, $5
  count++
  sum+= $5
}
END {
  print count, " records produce: ", sum
}
# invoke
awk -f script input

# OUTPUT
January Sales Revenue
Jan 53 115
Jan 121 620
2 records produce: 735
\end{minted}
\end{mdframed}
\section{awk output: print}
\begin{itemize}
  \item Writes to standard output 
    \begin{itemize}
      \item Output is terminated by newline 
    \end{itemize}
  \item If called with no parameter, it will print \$0
  \item Printed parameters are seperated by blank   
  \item Print control characters are allowed:
    \begin{itemize}
      \item \textbackslash n \textbackslash a \textbackslash t \textbackslash b \ldots
    \end{itemize}
\end{itemize}
\bigbreak \noindent
\textbf{\underline{Print examples:}}
\begin{mdframed}
\begin{minted}{bash}
$ cat grades
john 85
andrea 89
jasper 84

# Normal Output
$ awk '{print $1, $2}' grades

# values seperated by comma
$ awk '{print $1 "," $2}' grades

$ awk '{print $1 "," $2, " some more"}' grades
\end{minted}
\end{mdframed}
\newpage
\section{printf: Formatting output}
\bigbreak \noindent
\textbf{\textit{Syntax:}}
\begin{minted}{bash}
printf(format-string, var1, var2, ...)
\end{minted}
\begin{itemize}
  \item Each format specifier within ``format-string'' requires additional argument of matching type 
    \subitem \%d, \%i \ \ \ decimal integer
    \subitem \%c \ \ \ single character
    \subitem \%s \ \ \ string of characters
    \subitem \%f \ \ \ floating point number
\end{itemize}
\subsection{Format specifier modifers}
\begin{itemize}
  \item between ``\%'' and letter 
    \subitem \%10s
    \subitem \%7d
    \subitem \%10.4f
    \subitem \%-20s
  \item Meaning:
    \begin{itemize}[
      label = $\circ$
      ]
      \item  width of field, field is printed right justified (``-'' for left justify)
      \item  Precision: number of digits after decimal point
    \end{itemize}
\end{itemize}
\section{awk Example: list of products}
101: & propeller & 97.95 \\
102: & trailer hitch & 97.95 \\
103: & sway bar & 49.99 \\
104: & fishing line & 0.99 \\
105: & mirror & 4.99 \\
106: & cup holder & 2.49 \\
107: & cooler & 14.89 \\
108: & wheel & 49.99 \\
109: & transom & 199.00 \\
110: & pulley & 9.88 \\
111: & lock & 31.00 \\
112: & boat cover & 120.00 \\
113: & premium fish bait & 1.00 \\
  \begin{minted}{bash}
BEGIN {
  FS= ":"
  print "Marine Parts R Us"
  print "Main catalog"
  print "Part-id\tname\t\t\t price"
  print "=================================="
}
{
  printf("%3d\t%-20s\t%6.2f\n", $1, $2, $3)
}
END {
  print "=================================="
  print "Catalog has", NR, "parts"
}
\end{minted}
\newpage
\section{Typical awk script}
Divided into three major parts: \vspace{3mm}

\begin{figure}[ht]
    \centering
    \incfig[1]{awkparts}
    %\caption{awkparts}
    %\label{fig:awkparts}
\end{figure}
\nt{
  Comment lines start with \#
}
\section{awk Array}
\begin{itemize}
  \item awk allows one-dimensional arrays 
    \begin{itemize}[label=$\circ$]
      \item index can be number or string 
      \item elements can be string or number
\end{itemize}
\item array need not be declared
  \begin{itemize}[label=$\circ$]
    \item its size 
    \item its element type
  \end{itemize}
\item array elements are created  when first used
  \begin{itemize}[label=$\circ$]
    \item initalized to 0 or ``'' 
  \end{itemize}
\end{itemize}
\subsection{Arrays}
\bigbreak \noindent
\textbf{\textit{Syntax:}}
\begin{minted}{bash}
arrayName[index] = value
\end{minted}
\bigbreak \noindent
\textbf{\underline{Examples:}}
\begin{mdframed}
\begin{minted}{bash}
list[1] = "some value"
list[2] = 123

list["other"] = "oh my !"
\end{minted}
\end{mdframed}
\subsection*{Array as Map}
\begin{minted}{awk}
Age["Robert"] = 46
Age["George"] = 22
Age["Juan"] = 19
\end{minted}
\newpage
\subsection{Array Examples}
Input file: \vspace{1.5mm}

\noindent 1 & clothing & 3141 \\
1 & computers & 9161 \\
1 & textbooks & 21321 \\
2 & clothing & 3252 \\
2 & computers & 12321 \\
2 & supplies & 2242 \\
2 & textbooks & 15462 \vspace{4mm}

\noindent Desired output: \textbf{summary of department sales}
\bigbreak \noindent
\textbf{\underline{Complete program:}}
\begin{mdframed}
\begin{minted}{bash}
{
  deptSales[$2] += $3
}
END {
  for (x in deptSales)
    print x, deptSales[x]
}
\end{minted}
\end{mdframed}
\section{awk built-in functions}
\begin{itemize}
  \item \textbf{arithmetic} 
    \subitem \underline{ex:} sqrt, rand
  \item \textbf{string}
    \subitem \underline{ex:} index, length, split, substr, sprintf, tolower, toupper
  \item \textbf{misc.}
    \subitem \underline{ex:} system, systime
\end{itemize}
\subsection*{The split function}
\bigbreak \noindent
\textbf{\textit{Syntax:}}
\begin{minted}{bash}
split(string, array, fieldsep)
\end{minted}
This divides \textbf{string} into pieces separated by \textbf{fieldsep} \vspace{1mm}

\noindent It then stores the pieces in \textbf{array} \vspace{1mm}

\noindent if \textbf{fieldsep} is omitted, the value of FS is used
\bigbreak \noindent
\textbf{\underline{Example:}}
\begin{mdframed}
\begin{minted}{bash}
split("26:Miller:Comedian", fields, ":")

# Results
fields[1] = "26"
fields[2] = "Miller"
fields[3] = "Comedian"
\end{minted}
\end{mdframed}
\newpage
\section{awk control structures}
\begin{itemize}
  \item \textbf{Conditional} 
    \begin{itemize}[label=$\circ$]
      \item if-else 
    \end{itemize}
  \item \textbf{Repetition}
    \begin{itemize}[label=$\circ$]
      \item for 
        \begin{itemize}
          \item with counter 
          \item with array index
        \end{itemize}
      \item while
        \subitem \underline{also:} break, continue
    \end{itemize}
\end{itemize}
\section{If Statement}
\bigbreak \noindent
\textbf{\textit{Syntax:}}
\begin{minted}{bash}
if (conditional expression)
  statement-1
else
  statement-2
\end{minted}
\bigbreak \noindent
\textbf{\underline{Example:}}
\begin{minted}{bash}
if ( NR < 3 )
  print $2
else
  print $3
\end{minted}
\nt{
  Use compound \{ \} for more than one statement. \\ 
  \{ \vspace{1.5mm}

    ...
    ... \vspace{1.5mm}

  \}
}
\subsection*{If Statement for arrays}
\bigbreak \noindent
\textbf{\textit{Syntax:}}
\begin{minted}{bash}
if (value in array)
  statement-1
else 
  statement-2
\end{minted}
\bigbreak \noindent
\textbf{\underline{Example:}}
\begin{mdframed}
\begin{minted}{bash}
if ("clothing" in deptSales)
  print deptSales["clothing"]
else
  print "not found"
\end{minted}
\end{mdframed}
\section{for Loop}
\bigbreak \noindent
\textbf{\textit{Syntax:}}
\begin{minted}{bash}
for (initialization; limit-test; update)
  statement
\end{minted}
\bigbreak \noindent
\textbf{\underline{Example:}}
\begin{mdframed}
\begin{minted}{awk}
for (i=1; i <= 10; i++)
  print "The square of ", i, " is ", i*i
\end{minted}
\end{mdframed}
\subsection*{for Loop for arrays}
\bigbreak \noindent
\textbf{\textit{Syntax:}}
\begin{minted}{bash}
for (var in array)
  statement
\end{minted}
\bigbreak \noindent
\textbf{\underline{Example:}}
\begin{mdframed}
\begin{minted}{bash}
for (x in deptSales) {
  print x
  print deptSales[x]
}
\end{minted}
\end{mdframed}
\section{while Loop}
\bigbreak \noindent
\textbf{\textit{Syntax:}}
\begin{minted}{awk}
while (logical expression)
  statement
\end{minted}
\bigbreak \noindent
\textbf{\underline{Example:}}
\begin{mdframed}
\begin{minted}{awk}
i=1 
while (i <= 10) {
  print "The square of ", i, " is ", i*i
  i=i+1
}
\end{minted}
\end{mdframed}
\section{loop control statements}
\begin{itemize}
  \item \textbf{break} 
    \subitem exits loop
  \item \textbf{continue}
    \subitem skips the rest of current iteration, continues with next iteration
\end{itemize}
\newpage

\end{document}

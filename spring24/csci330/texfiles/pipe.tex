\documentclass{report}

\input{~/latex/template/preamble.tex}
\input{~/latex/template/macros.tex}

\title{\Huge{Process \& Pipe}}
\author{\huge{Matt Warner}}
\date{\huge{}}
\pagestyle{fancy}
\fancyhf{}
\rhead{}
\lhead{\leftmark}
\cfoot{\thepage}
% \usepackage[default]{sourcecodepro} \usepackage[T1]{fontenc}

\pgfpagesdeclarelayout{boxed}
{
  \edef\pgfpageoptionborder{0pt}
}
{
  \pgfpagesphysicalpageoptions
  {%
    logical pages=1,%
  }
  \pgfpageslogicalpageoptions{1}
  {
    border code=\pgfsetlinewidth{1.5pt}\pgfstroke,%
    border shrink=\pgfpageoptionborder,%
    resized width=.95\pgfphysicalwidth,%
    resized height=.95\pgfphysicalheight,%
    center=\pgfpoint{.5\pgfphysicalwidth}{.5\pgfphysicalheight}%
  }%
}

\pgfpagesuselayout{boxed}

\begin{document}
  \maketitle
\section{Unit Overview}
\begin{itemize}
  \item Process Management
    \begin{itemize}[label=$\circ$]
      \item create new process
      \item change what process is doing
    \end{itemize}
  \item Pipe concept
    \begin{itemize}[label=$\circ$]
      \item enables inter-process communication
    \end{itemize}
\end{itemize}
\section{Process Management System Calls}
\begin{itemize}
  \item \textbf{fork}
    \begin{itemize}[label=$\circ$]
      \item create a new process
    \end{itemize}
  \item \textbf{wait}
    \begin{itemize}[label=$\circ$]
    wait for a process to terminate
    \end{itemize}
    \item \textbf{exec}
      \begin{itemize}[label=$\circ$]
        \item execute a program
      \end{itemize}
\end{itemize}
\subsection{System Call: fork}
\begin{itemize}
  \item creates new process that is duplicate of current process
  \item new process is \textit{\textbf{almost}} the same as current process
    \bigbreak \noindent
  \item new process is \textit{\textbf{child}} of current process
  \item old process is \textit{\textbf{parent of new process}}
    \bigbreak \noindent
  \item after call to fork, both processes run concurrently
\end{itemize}
\bigbreak \noindent
\textbf{\underline{Example:}}
\begin{minted}{c++}
#include <sys/types.h>
#include <unistd.h>
#include <iostream>

int main() {

  std::cout << "Before fork\n";
  
  fork();

  std::cout << "After fork\n";

  return 0;
 
 /*
    ***Output***
    Before fork
    After fork
    After fork
 */
}
\end{minted}
\textit{\textbf{Note:}} ``After fork'' gets printed twice because fork() duplicates the current process. Also, fork() child process will execute the remaining part of the code after the call to fork() so ``Before fork'' only gets printed once. \vspace{1.5mm}
\begin{figure}[ht]
\centering
\includegraphics[width=0.6\textwidth]{ $HOME/Documents/figures/ksnip_20240416-183834.png }
\end{figure}

\noindent Its also worth mentioning that the pid of the child process is 0, indicating a newly created child process, so if we had:
\begin{minted}{c++}
pid_t p = fork;

std::cout << p << std::end;
\end{minted}
The output would be something like: \vspace{1mm}

\noindent 4412 \\ 0

\section{Fork() in depth}
The Fork system call is used for creating a new process in Linux, and Unix systems, which is called the \textit{\textbf{child process}}, which runs concurrently with the process that makes the fork() call (parent process). After a new child process is created, both processes will execute the next instruction following the fork() system call.
\bigbreak \noindent
The child process uses the same pc(program counter), same CPU registers, and same open files which use in the parent process. It takes no parameters and returns an integer value.
\bigbreak \noindent
Below are different values returned by fork().
\begin{itemize}
  \item \textit{\textbf{Negative Value:}} The creation of a child process was unsuccessful.
  \item \textit{\textbf{Zero:}} Returned to the newly created child process.
  \item \textit{\textbf{Positive value:}} Returned to parent or caller. The value contains the process ID (PID) of the newly created child process
\end{itemize}

\begin{figure}[ht]
\centering
\includegraphics[width=0.5\textwidth]{ $HOME/Documents/figures/Fork_in_C.jpg }
\end{figure}
\nt{
  fork() is threading based function, to get the correct output run the program on a local system
}
\bigbreak \noindent
\textbf{\underline{Example:}}
\begin{minted}{c++}
pid_t p = fork();
if (p < 0) {
  perror("fork fail");
  exit(1);
}

std::cout << "Hellow world!" << "process_id(pid) = " <<  getpid() << std::endl;
\end{minted}
There are a couple things to note here. For starters, the fork system call is included in the unistd.h lib.\vspace{1.5mm}

\noindent Secondly, we call the fork() function and set its return value equal to a variable of type \textit{\textbf{pid\_t}}, which is used to store the process id (PID) returned from the function call. \vspace{1.5mm}
\nt{
  The type of \textit{\textbf{pid\_t}} is a \textit{\textbf{signed int}}
  \bigbreak \noindent
  The header file which is needed to to used \textit{\textbf{pid\_t}} is \textit{\textbf{sys/types.h}}
}
\noindent 
After obtaining the process id, we check for failure (if the pid is negative value) \vspace{1.5mm}

\noindent We then print out the pid using the getpid function found in \textit{\textbf{sys/types.h}}
\subsection{PID functions}
\begin{itemize}
  \item \textit{\textbf{getpid()}} - this function returns the process id of the calling process.
  \item \textit{\textbf{getppid()}} - this function returns the process id of the parent process.
\end{itemize}
\bigbreak \noindent
Since we know that the returned value of fork() for the child process is 0, we can implement a section of code that will only execute within the child process by comparing pid to 0. \vspace{1.5mm}

\noindent That would look something like this:

\begin{minted}{c++}
{
    pid_t p;
    p = fork();
    if(p<0)
    {
      perror("fork fail");
      exit(1);
    }
    // child process because return value zero
    else if ( p == 0)
        printf("Hello from Child!\n");
 
    // parent process because return value non-zero.
    else
        printf("Hello from Parent!\n");
}
int main()
{
    forkexample();
    return 0;
}
\end{minted}
\section{Systel Call: wait}
\bigbreak \noindent
\textbf{\textit{Syntax:}}
\begin{minted}{cpp}
pid_t wait(int *status)
\end{minted}
\begin{itemize}
  \item lets parent process wait untill a child process terminates
    \begin{itemize}[label=$\circ$]
      \item parent is resumed once a child process terminates
    \end{itemize}
  \item returns process id of terminated child
    \begin{itemize}[label=$\circ$]
      \item return -1 if there is no child to wait for
    \end{itemize}
  \item \textbf{status} holds exit status of child
    \begin{itemize}[label=$\circ$]
      \item can be examined with \textbf{WEXITSTATUS(status)}
    \end{itemize}
\end{itemize}
\bigbreak \noindent
\textbf{\underline{Example:}}
\begin{minted}{cpp}
int pid, status;

std::cout << "Before fork\n" << std::endl;

fork();

pid = wait(&status);

if (pid == -1) {
  std::cout << "nothing to wait for\n";
  return 255;
} else {
  std::cout << "done waiting for " << pid << ". ended with: " << WEXITSTATUS(status) << std::endl;
}

std::cout << "After fork\n";

/*
  *** Output ***
  Before fork
  nothing to wait for
  done waiting for: 23550, ended with: 255
  After fork
*/
\end{minted}
\section{System Call: exec}
\begin{itemize}
  \item family of functions that replace current process image with a new process image
    \begin{itemize}[label=$\circ$]
      \item actual system call: \textbf{execve}
      \item library functions
        \begin{itemize}[label=$\circ$]
          \item \textbf{execl, execlp, execle}
          \item \textbf{execv, execvp}
        \end{itemize}
    \end{itemize}
      \item arguements specify new executable to run, plus its arguments and enviroment.
\end{itemize}
\subsection{C Library Function: execl}
\bigbreak \noindent
\textbf{\textit{Syntax:}}
\begin{minted}{cpp}
int execl(const char *path, const char *arg, ...)
\end{minted}
\begin{itemize}
  \item starts executable for command spefified in \textbf{path}
  \item new executable runs in current process
  \item \textbf{path} is specified as absolute path
  \item arguments are specified as list, starting at argv[0],
    \subitem terminated with (char *NULL)
  \item new executable keeps same enviroment
  \item returns -1 on error
\end{itemize}
\bigbreak \noindent
\textbf{\underline{Example:}}
\begin{mdframed}
\begin{minted}{cpp}
#include <unistd.h>
#include <sys/types.h>
#include <cstdio>
#include <cstdlib>
#include <iostream>

int main() {
  int rs;

  cout << "Program started in process " << getpid() << endl;

  rs = execl("/bin/ps", "ps", (char *)NULL);

  if (rs == -1) {
    perror("execl");
    exit(rs);
  }
  cout << "Maybe we see this ?\n";
  return 0;
}
\end{minted}
\end{mdframed}
\subsection{C Library Functions: exec family}
\begin{itemize}
  \item \textbf{execl, execlp, execle}
    \begin{itemize}[label=$\circ$]
      \item specify arguments and enviroment as list
    \end{itemize}
  \item \textbf{execv, execvp}
    \begin{itemize}[label=$\circ$]
      \item specify arguments and enviroment as array of string values
    \end{itemize}
  \item \textbf{execlp, execvp}
    \begin{itemize}[label=$\circ$]
      \item look for new executable via PATH
    \end{itemize}
\end{itemize}
\bigbreak \noindent
\textbf{\underline{Example:}}
\begin{minted}{cpp}
#include <sys/types.h>
#include <unistd>

int rs; // return status

cout << "program started in process: " << getpid();

rs = execvp("ls", argv);
if (rs == -1) {
  perror("execvp");
  exit(rs);
}

cout << "Maybe we see this ?\n";
\end{minted}
\section{Together: fork and exec}
\begin{itemize}
  \item UNIX does not have a \underline{single} system call to spawn a new additional process with a new executable
  \item Instead, there are two steps that must be used:
    \begin{itemize}[label=$\circ$]
      \item fork to duplicate current process
      \item execo to morph child process into new executable
    \end{itemize}
\end{itemize}
\bigbreak \noindent
\textbf{\underline{Example:}}
\begin{minted}{cpp}
#include <unistd>
#include <sys/types.h>
#include <sys/wait.h>

#include <cstdio>
#include <cstdlib>
#include <iostream>

int main(int argc, char **argv) {
  int rs, pid, status;

  // Create child process
  pid = fork();

  // Error checking
  if (pid == -1) {
    perror("fork");
    exit(pid);
  }
  if (pid == 0) { // child process
    rs = execvp("echo", argv)
    if (rs == -1) {
      perror ("execvp");
      exit(rs);
    }
  } else {
    cout << "done waiting for: " << wait(&status) << endl;
  }
  return 0;
}
\end{minted}
\section{UNIX Pipe}
\begin{itemize}
  \item Can create a software \textit{\textbf{pipline:}}
    \subitem set of processes chained by their standard IO
  \item Output of one process becomes input of second process
\end{itemize}
\bigbreak \noindent
\textbf{\textit{command line example:}}
\begin{minted}{bash}
  ls | wc
\end{minted}
\section{System Call: pipe}
\bigbreak \noindent
\begin{minted}{cpp}
int pipe(int pipefd[2])
\end{minted}
\begin{itemize}
  \item Creates a \textit{\textbf{channel}} to transport data
  \item Has direction: one side to write, one side to read
    \begin{itemize}[label=$\circ$]
      \item available via 2 file descriptors \textbf{pipefd[2]}
      \item read side \textbf{pipefd[0]}
      \item write side \textbf{pipefd[1]}
    \end{itemize}
\end{itemize}
\bigbreak \noindent
\textbf{\underline{Example:}}
\begin{minted}{cpp}
#include <unistd.h>
#include <iostream>
#include <cstdlib>
int main() {
  cout << "Before pipe\n";

  int pipefd[2], rs;
  
  rs = pipe(pipefd);
  if (rs == -1) {
    perror("pipe");
    exit(EXIT_FAILURE);
  }

  write(pipefd[1], "Hello", 6);
   
  char buffer[256];
  read(pipefd[0], buffer, sizeof(buffer));

  cout << "pip contained: "  << buffer << endl;

  return 0;

}
\end{minted}
\end{document}

\documentclass{report}

\input{~/latex/template/preamble.tex}
\input{~/latex/template/macros.tex}

\title{\Huge{Chapter 4 and 5 notes}}
\author{\huge{Matt Warner}}
\date{\huge{}}
\pagestyle{fancy}
\fancyhf{}
\rhead{}
\lhead{\leftmark}
\cfoot{\thepage}
% \usepackage[default]{sourcecodepro}
% \usepackage[T1]{fontenc}


\pgfpagesdeclarelayout{boxed}
{
  \edef\pgfpageoptionborder{0pt}
}
{
  \pgfpagesphysicalpageoptions
  {%
    logical pages=1,%
  }
  \pgfpageslogicalpageoptions{1}
  {
    border code=\pgfsetlinewidth{1.5pt}\pgfstroke,%
    border shrink=\pgfpageoptionborder,%
    resized width=.95\pgfphysicalwidth,%
    resized height=.95\pgfphysicalheight,%
    center=\pgfpoint{.5\pgfphysicalwidth}{.5\pgfphysicalheight}%
  }%
}

\pgfpagesuselayout{boxed}

\begin{document}
  \maketitle

  \section*{4.1 Experiments, Sample Spaces, and Events}
  \bigbreak \noindent
  \subsection*{Experiments}
  \begin{mdframed}
    \vspace{3mm}

\begin{minipage}{0.4\textwidth}
  To understand probability concepts, we need to think carefully about \textbf{experiments}	. Consider the activity, or act, of tossing a coin, selecting a card from a standard poker deck, counting the number of people standing on a city bus, or even testing a cell phone for defects before shipment. In each of these activites, the outcome is uncertain. For example, when we test a new cell phone, we do not know (for sure) wheter it will be defect-free. This idea of uncertainty leads to the definiton of an experiment.
\end{minipage}
\hspace{7mm}\begin{minipage}{0.5\textwidth}
\thmcon{
  \textbf{\underline{Defintion}}
  \vspace{3mm}

  An experiment is an activity in which there are at least two possible outcomes and the result of the activity cannot be predicted with absolute certainty.
}	
\end{minipage}
\vspace{3mm}
\end{mdframed}
\bigbreak \noindent \bigbreak \noindent
Here are some examples of experiements
\begin{itemize}
  \item Roll a six-sided die and record the number that lands face up. 
    \vspace{2mm}

  \subitem We cannot say with certainty that the number face up will be a 1, a 2, etc., so this activity is an experiment.
  \item Using a radar gun, record the speed of a pitch at a Red Soxs baseball game.
    \vspace{2mm}

    \subitem We're not sure whether the pitch will be a fastball, curveball, slider, etc. And even if we steal the signal from the catcher, we cannot predict the speed of the pitch with certainty.
  \item Count the number of patients who arrive at the emergency room of a city hospital during a 24-hour period.
  \vspace{.5mm}

    \subitem Although past records might help us estimate the patient volume, there is no way of predicting the exact number of patients who visit the emergency room during a 24-hour period.
\end{itemize}
\bigbreak \noindnet
\subsection*{Outcomes}
Because we don't know for sure what will happen when we conduct an experiment, we need to consider all possible outcomes. This sounds easy (just think about all the things that can happen), but it can be tricky. Sometimes it involves a lot of counting, but often outcomes can be visualized using a tree diagram. Consider the following examples.
\bigbreak \noindnet
\ex{}{
  Suppose an outgoing letter at a New York City post office is selected at random and the first digit in the address zip code is recorded. How many possible outcomes are there, and what are they.
  \bigbreak \noindent
  \nt{
  This is an experiment because we cannot predict the first digit in a zip code with certainty.
  } 
}
\pagebreak
\section*{4.3 Counting Techniques}
\bigbreak \noindent
\begin{large}
 The Multiplication Rule 
\end{large}
In an equally likely outcome experiment, computing probabilities means counting. To find the probability of an event A, count the number of outcomes in the event A and divide by the number of outcomes in the entire sample space. 
\ex{}{
  A little league Louisville Slugger bat can be customized on the barrel, handle, grip, and the end cap. There are 14 barrel colors, 16 handle colors, 20 grip designs, and 5 end cap styles. How many possible Louisville sluggers can be manufactured.
\bigbreak \noindent
This is a counting problem, and there are four slots to fill: barrel, handle, grip and end cap. We'll assume that all choices are compatible, and that the choice of any one item does not depend on any other item.
\bigbreak \noindent
Here's how to apply the Multiplication Rule.
$$\frac{14}{\text{Barrel}} \cdot \frac{16}{\text{handle}} \cdot \frac{20}{\text{Grip}}\cdot \frac{5}{\text{End Cap}} = 22,400$$
}
\bigbreak \noindent \bigbreak \noindent
\ex{}{
  Texas drives can purchase a special Adopt-a-Beach license plate so that a portion of the fee goes to beach cleanup efforts. This license plate consists of two letters, two numbers, and a letter.
  \bigbreak \noindent
  A. How many different Adopt-a-Beach license plates are possible?
  \bigbreak \noindent
  B. How many Adopt-a-Beach plates begin with BB?
  \bigbreak \noindent
  \sol
  \bigbreak \noindent
  A. There are 26 possible letters for the first, second and fifth slots, and 10 possible numbers for the third and fourth slots.
  \bigbreak \noindent
  Using the multiplication rule, we get
  $$ 26 \cdot 26 \cdot 10 \cdot 10 \cdot 26$$
  \bigbreak \noindent
  B. Starting with BB gives us
  $$ 1 \cdot 1 \cdot 10 \cdot 10 \cdot 26$$
}
\pagebreak
\section*{4.4 Conditional Probability}
\bigbreak \noindent
consider a banker who commutes 30 miles to work every day. Because of several factors (e.g, weather, road construction, family obligations), the probabiltiy that she makes it to work on time on any random day is 0.5. If the event T is
\bigbreak \noindent
\centerline{T = the banker makes it to work on time},
\bigbreak \noindent
then P(T) = 0.5. This is an unconditional probability statement: No extra information related to the event T is know or given.
\bigbreak \noindent
Suppose a random day is selected, and the road conditions are terrible because of a snowstorm. The probability that the banker arrives at work on time is surely lower, perhaps around 0.1. Knowing the extra information (a snowstorm) changes the probabilty assignment for T.
\bigbreak \noindent
The statement "What is the probability that the banker arrives at work on time if it is snowing?" is a conditional probability question. The extra information is that it's snowing outside. If the event F is defined as
\bigbreak \noindent
\centerline{F = a snowstorm}
\bigbreak \noindent
then this conditional probability is written as P(T|F) = 0.1; the probability that the banker arrives at work on time, given that it is snowing, is 0.1.
\bigbreak \noindent \bigbreak \noindent
Consider an experiment in which a fair, six-sided die is rolled and the number that lands face up is recorded. The sample space is S = \{1,2,3,4,5,6\}. Consider the following events.
\bigbreak\noindent
\centerline{A = \{1\}  = roll a 1 \hspace{5mm} and B = \{1,3,5\} = roll an odd number}
\bigbreak \noindent
\pagebreak
\section*{4.5 Independence}


\end{document}


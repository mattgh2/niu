\documentclass{report}

\input{~/latex/template/preamble.tex}
\input{~/latex/template/macros.tex}

\title{\Huge{Chapter 3 Notes - Numerical Summary Measures}}
\author{\huge{Matt Warner}}
\date{\huge{}}
\pagestyle{fancy}
\fancyhf{}
\rhead{Stats 200}
\lhead{\leftmark}
\cfoot{\thepage}
% \usepackage[default]{sourcecodepro}
% \usepackage[T1]{fontenc}
\usepackage{tikz}
\usepackage{pgfplots}
\pgfplotsset{compat=1.18}

\pgfpagesdeclarelayout{boxed}
{
  \edef\pgfpageoptionborder{0pt}
}
{
  \pgfpagesphysicalpageoptions
  {%
    logical pages=1,%
  }
  \pgfpageslogicalpageoptions{1}
  {
    border code=\pgfsetlinewidth{1.5pt}\pgfstroke,%
    border shrink=\pgfpageoptionborder,%
    resized width=.95\pgfphysicalwidth,%
    resized height=.95\pgfphysicalheight,%
    center=\pgfpoint{.5\pgfphysicalwidth}{.5\pgfphysicalheight}%
  }%
}

\pgfpagesuselayout{boxed}

\begin{document}
  \maketitle

\begin{large}
  \noindent\textbf{\section{Important features of a numerical data set}}
\end{large}
\hrule
\bigbreak \noindent
\begin{itemize}
  \item Shape of the distribution $\rightarrow$ see Chapter 2 
  \item Center of the distribution
  \item Spread of the distribution
\end{itemize}
\bigbreak \noindent
\begin{large}
  \textbf{Measure of Center}
\end{large}
\bigbreak \noindent
Goal - calculate a number that identifies the \underline{middle of a data set} or identifies a \underline{typical data value}
\bigbreak \noindent
\begin{itemize}
  \item \textbf{Sample mean}
    \begin{itemize}
      \item Let $ x_1, x_2, x_3, \ldots, x_n$ denote the $n$ data values in a sample. 
      \item $\bar{x} = \frac{x_1+x_2+x_3+\ldots+ x_n}{n} = \frac{\sum{x_i}}{n}$
    \end{itemize}
\end{itemize}
\begin{itemize}
  \item \textbf{Sample median} 
    \begin{itemize}
      \item Sort the data values from smallest to largest 
      \item $\tilde{x}= \begin{cases}\text { middle value } & \text { if } n \text { is odd } \\ \text { average of middle pair } & \text { if } n \text { is even }\end{cases}$ 
    \end{itemize}
\end{itemize}
\begin{itemize}
  \item \textbf{Sample mode}
    \begin{itemize}
  \item $M =$ data value occurring most often.
  \item if two (or three) values each occur most often the dist'n is bimodal (or trimodal).
  \item If every value occurs equally often then the mode does not exist.
  \end{itemize}
\end{itemize}




\bigbreak \noindent \bigbreak \noindent
\ex{}{

A random sample of $n=12$ tractor trailers was selected from a particular stretch of highway and their speeds recorded. Here are the data.
\vspace{2mm}

$$65,79, 60,67,71,83,61,67,64,77,69,74$$
}
\bigbreak \noindent
\prob{a} Calculate the sample mean $(\bar{x})$, sample median$(\tilde{x})$, and the sample mode (M).
\bigbreak \noindent

Sample mean =  $(65+79+60+67+71+83+61+67+64+77+69+74) / 2 $ 
\vspace{1.5mm}

\centerline{ = 69.75}
\vspace{2mm}

Sample median = $68$
\vspace{4mm}

Sample mode = $67$
\vspace{5mm}

\noindent\prob{b} Remove 83 and re-calculate $\tilde{x}$
\vspace{3mm}

$\tilde{x} = 67$
\bigbreak \noindent
\noindent\prob{c} Change 60 to 61 and re-calculate $M$.
\vspace{3mm}

$M = 67 \text{ and } 61$
\bigbreak \noindent
\prob{d} Use the original data and change one copy of 67 to 68 and re-calculate $M$.
\vspace{2mm}

$M$ = \textit{No Mode}
\nt{
If there are outliers, the mean  should not be used to obtain a measure of center since it can pull the mean up or down. Instead, consider using the median as it is less sensitive to outliers.

}

\bigbreak \noindent \bigbreak \noindent
  What does the shape of a distribution possibly tell us about the mean and median? 
  \vspace{1.5mm}

    \begin{itemize}
      \item When a distribution is \textbf{left skewed} the mean is likely \textbf{less} than the median. 
      \item When a distribution is \textbf{right skewed} the mean is likely \textbf{greater} than the median. 
      \item When a distribution is \textbf{symmetric} the mean is likely \textbf{roughly the same} as the median.
\end{itemize}
\bigbreak \noindent \bigbreak \noindent
\textbf{Question - } The distribution of tractor trailer speeds has possibly what shape?
\vspace{3mm}

\sol{The data creates a shape on the histogram that is skewed to the right.}
\bigbreak \noindent \bigbreak \noindent
\nt{
- $\bar{x}=$ mean of a sample $\quad$ ( $\bar{x}$ is an example of a samplestatistic $)$
\vspace{2mm}

\noindent \hspace{2mm}$\mu=$ mean of the entire population ( $\mu$ is an example of a population parameter)
\vspace{2mm}

\noindent \hspace{1mm}When $\mu$ is unknown we often use $\bar{x}$ as an estimate (or prediction) of $\mu$.
}
\bigbreak \noindent \bigbreak \noindent
\hrule
\bigbreak \noindent
\begin{large}
  \noindent \textbf{\subsection{Measures of Spread}}
\end{large}
\bigbreak \noindent
Goal - calculate a number that measures the variability among the data values. or measures how far apart the data values are from each other.
\bigbreak \noindent
\begin{itemize}
  \item For measures of spread, if the answer is:
  \begin{itemize}
  \item large, then the data is more spread out (has more variability) 
  \item small, then the data is less spread out (has less variability)
  \end{itemize}
\end{itemize}
\bigbreak \noindent
\begin{itemize}
  \item \textbf{Sample range} \hspace{10mm} $ R = $ max - min 
    \vspace{2mm}

  \item \textbf{sample variance}
  \begin{itemize}
  \item Definition \hspace{10mm} $S^2 = \frac{(x_1 - \bar{x})^2 + (x_2 - \bar{x})^2 + \ldots + (x_n - \bar{x})^2}{n-1} = \frac{\sum(x_i -\bar{x})^2}{n-1}$
    \vspace{2mm}

  \item (Computational or short cut) \hspace{10mm} $s^2=\frac{\sum x_i^2-\frac{\left(\sum x_i\right)^2}{n}}{n-1}$
  \end{itemize}
  \vspace{3mm}

\item \textbf{Sample standard deviation} \hspace{10mm} $s = \sqrt{s^2}$
\end{itemize}
\vspace{5mm}
\hrule
\vspace{3mm}
\begin{large}
  \textbf{\subsection{Interquartile Range}} 
\end{large}

  \centerline{IQR = $Q_3 - Q_1$}
  \vspace{2mm}

\begin{itemize}
  \item $Q_3 = 3^{rd} \text{ quartile} = 75^{th} \text{percentile}$
  \item the number that has 75\% of the data set at or below it
  \item the median of the upper half of the data.
\end{itemize}
\vspace{3mm}

\begin{itemize}
  \item $Q_1 = 1^{st} \text{ quartile} = 25^{th} \text{ percentile}$
  \item the number that has 25\% of the data set at or below it
  \item the median of the lower half of the data.
\end{itemize}

\vspace{5mm}

\ex{}{
%% Left off here in class. at the example problem

  A random sample of $n=12$ tractor trailers was selected froma a particular strectch of highway and their speeds recorded. Here are the data.
  $$60, 61, 64, 65, 67, 67, 69, 71, 74, 77, 79, 83$$
}
\bigbreak \noindent
\prob{a} Find the sample range  $$83 - 60 = 23$$
\vspace{3mm}

\noindent\prob{b} $Q_1$ = median of the first half $=(64+ 62) / 2 \rightarrow 64.5$  

\vspace{2.5mm}
\hspace{-5mm}$M$ $ = (67 + 69) / 2 = 68$  
\vspace{2.5mm}

\hspace{-6mm} $Q_3$ = median of second half = $75.5$
\vspace{2.5mm}

\noindent \hspace{5mm}IQR = $11$
\bigbreak \noindent
\prob{c} Suppose ther was a $13^{th}$ tractor trailer in the sample with $x_{13} = 85$. Re-do part \prob{(b}
\vspace{3mm}

$Q_1 = 65$ \hspace{10mm} $M = 69$ \hspace{10mm} $Q_3 = 77$ \hspace{10mm} IQR = 12

\bigbreak \noindent

\ex{}{
  find $Q_1$ and $Q_3$ for the following data values: 8, 3,7, 4, 1, 10, 5, 2, 11, 9, 6
}
\vspace{3mm}

\prob{a} $Q_1 = (3+4) / 2 = 3.5$
\vspace{3mm}

\prob{b} $Q_3 = (8+9) / 2 = 8.5$
\bigbreak \noindent \bigbreak \noindent
\nt{

  When finding range, do not forget to order the data from smallest to largest. 
}
\pagebreak
\ex{}{
  Consider the following data
  \vspace{2mm}

  \centerline{$2,5,6,9,13$ (in inches)}
}
\bigbreak \noindent

\prob{a} Find the sample variance using the definition
\bigbreak \noindent

First we need to find the mean.

$$(2+5+6+9+13) / 5 = 7$$

\centerline{Mean = 7}
\vspace{2mm}

So, 

$$ S^2 = \frac{(2-7)^2 + (5-7)^2 + (6-7)^2 + (9 -7)^2 + (13-7)^2}{5-1}$$

  $$ S^2 = 17.5$$
\vspace{3mm}

\prob{b} Find the sample variance using the computational (or short cut) formula.
\bigbreak \noindent

Computational Formula:

$$s^2=\frac{\sum x_i^2-\frac{\left(\sum x_i\right)^2}{n}}{n-1}$$
\vspace{3mm}

So,
$$s^2  = \frac{\sum{(2^2+5^2+6^2+9^2+13^2)} - \frac{35}{5}}{4}$$

$$ s^2 = \frac{315 = 245}{4}$$

$$ s^2 = 17.5$$
\vspace{3mm}

\prob{c} $S = \sqrt{S^2}$ = $\sqrt{17.5} = 4.18$
\vspace{3mm}

\prob{d} The units are measured in inches.

\bigbreak \noindent
\hrule
\bigbreak \noindent
\begin{large}
  \textbf{\subsection{Important Facts}} 
\end{large}
\bigbreak \noindent
\begin{itemize}
  \item Standard deviation is often thought of as measuring the typical distance that an observation is from the mean.
  \item Standard deviation is often used as a ruler for judging distances.
    \vspace{3mm}

  \item $s^2$ = variance of a \textbf{sample} \hspace{27mm} ($s^2$ is an example of a sample statistic)
    \vspace{2mm}

    $\sigma^2$ = variance of the \textbf{entire population} \hspace{5mm}  ($\sigma^2$ is an example of a population parameter)
    \vspace{3mm}

    When $\sigma^2$ is an unknown we often use $s^2$ as an estimate (or prediction) of $\sigma^2$
    \vspace{3mm}

    The mean and the standard deviation can be used together to describe the distrobution of a data set more precisely.
\end{itemize}
    \bigbreak \noindent
    %first jump
    \ex{}{
    Suppose that Bob took exams in his English and Math classes. His scores together with the class summaries are below. In which class did Bob do better, relative to his classmates?
    \bigbreak \noindent
    \hspace{25mm}\begin{tabular}{r|c|c|} 
& \multicolumn{1}{c}{ English } & Math \\
\cline { 2 - 3 } Bob & 80 & 85 \\
\cline { 2 - 3 } Mean & 75 & 80 \\
\cline { 2 - 3 } Standard Deviation & 2.5 & 5 \\
\end{tabular}
}
\sol{English is the correct answer due to the standard deviation being lower.
  \vspace{2mm}

\noindent  Bob scored 2 standard deviations above the mean in english, in math he was only 1 standard deviation above the mean
}

\bigbreak \noindent \bigbreak \noindent
\hrule
\bigbreak \noindent
\begin{large}
  \textbf{\subsection{Empirical Rule}} 
\end{large}
\begin{itemize}
  \item Applies \textbf{only} when the shape of the distribution is \textbf{approximately normal} (bell shaped)
  \item Approximately 68\% of the observations are within 1 standard deviation of the mean
  \item Approximately 95\% of the observations are within 2 standard deviations of the mean
  \item Approximately 99.7\% of the observations are within 3 standard devations of the mean
\end{itemize}
\bigbreak \noindent \bigbreak \noindent
\ex{}{
In a random sample of people, the mean height was \( \bar{x} = 66 \) inches with a standard deviation of \( s = 4 \) inches. Assuming the distribution of heights is normal, answer the following questions:
}
\bigbreak \noindent
%second jump
\begin{enumerate}
  \item[(a)] 68\% of the people have heights between \underline{62} \& \underline{70}.
    
    \vspace{1em}
    
    \item[(b)] \underline{99.7}\% of the people have heights between 54 and 78.
    
    \vspace{1em}
    
    \item[(c)] 95\% of the people have heights between \underline{58} \& \underline{74}.
    
    \vspace{1em}
    
    \item[(d)] What percent of the heights are between 66 and 70?
      \vspace{2mm}

   34\% of the data  is between 66 and 70. This is because it is only the right side of the middle 68\% 

    \vspace{1em}
    
    \item[(e)] What percent of the heights are between 58 and 66?
      \vspace{2mm} 

      47.5\% of the data. rhis is because it is only the left side of the 2 standard deviation range (95\%)
    \vspace{1em}
    
    \item[(f)] What percent of the heights are greater than 74?
      \vspace{2mm}

      100 - 95 = both tails. Here we only want 1 tail. So, 5 / 2 = 2.5\%
    
    \vspace{1em}
    
    \item[(g)] What percent of the heights are either less than 54 or greater than 78?
      \vspace{2mm} 

      100 - 99.7 = .3\%
    \vspace{1em}
\end{enumerate}
\bigbreak \noindent
\hrule
\bigbreak \noindent
\subsection{Chebyshev's Rule}
\bigbreak \noindent
\begin{itemize}
    \item Applies to any data set, regardless of shape.
    
    \vspace{1em}
    
    \item Provided \( k > 1 \), at least \( 100 \left(1 - \frac{1}{k^2}\right) \% \) of the data set is within \( k \) standard deviations of the mean. In other words, the data lie in the interval from \( \bar{x} - (k \cdot s) \) to \( \bar{x} + (k \cdot s) \).
    
    \vspace{1em}
\end{itemize}
\ex{}{
  – In a random sample of people the mean height was $\bar{x}$ = 66 inches with a standard
deviation of $s$ = 4 inches. Without assuming anything about the shape of the distribution of
heights, answer the following.
}
\bigbreak \noindent
\begin{enumerate} % third jump
    \item[(a)] What percent of heights are between 58 and 74? 
      \vspace{1mm}
    How does this compare to using the Empirical Rule?
    \vspace{2mm} 

    $ k = 2 $  \hspace{10mm} at least $100(1-\frac{1}{2^2})\% \rightarrow \text{at least 75\%}$
    \vspace{1em}
    
    \item[(b)] What percent of heights are either less than 58 or greater than 74?
    
    \vspace{1em}
    
    \item[(c)] What percent of heights are between 54 and 78?
    
    \vspace{1em}
    
    \item[(d)] What percent of heights are between 60 and 72?
    
    \vspace{1em}
    
    \item[(e)] At least 84\% of the heights are between \underline{\hspace{1cm}} \& \underline{\hspace{1cm}}.
    
    \vspace{1em}
\end{enumerate}

\end{document}

\documentclass{report}
\input{~/latex/template/preamble.tex}
\input{~/latex/template/macros.tex}

\title{\Huge{Chapter 4 Notes}}
\author{\huge{Matt Warner}}
\date{\huge{}}
\pagestyle{fancy}
\fancyhf{}
\rhead{}
\lhead{\leftmark}
\cfoot{\thepage}
% \usepackage[default]{sourcecodepro} \usepackage[T1]{fontenc}
\usepackage[utf8]{inputenc}
\pgfpagesdeclarelayout{boxed}
{
  \edef\pgfpageoptionborder{0pt}
}
{
  \pgfpagesphysicalpageoptions
  {%
    logical pages=1,%
  }
  \pgfpageslogicalpageoptions{1}
  {
    border code=\pgfsetlinewidth{1.5pt}\pgfstroke,%
    border shrink=\pgfpageoptionborder,%
    resized width=.95\pgfphysicalwidth,%
    resized height=.95\pgfphysicalheight,%
    center=\pgfpoint{.5\pgfphysicalwidth}{.5\pgfphysicalheight}%
  }%
}

\pgfpagesuselayout{boxed}
\begin{document}
\maketitle
\chapter*{Integration}
\addcontentsline{toc}{chapter}{Chapter 4: Integral Calculus}

\section{Antidifferentiation}
\addcontentsline{toc}{section}{4.1 Antidifferentiation}

\section{Antiderivatives as Areas}
\addcontentsline{toc}{section}{4.2 Antiderivatives as Areas}

\section{Area and Definite Integrals}
\addcontentsline{toc}{section}{4.3 Area and Definite Integrals}

\section{Properties of Definite Integrals}
\addcontentsline{toc}{section}{4.4 Properties of Definite Integrals}
\subsection{Additive Property}
\subsection{Average Value}
\subsection{Moving Average}

\section{Integration Techniques: Substitution}
\addcontentsline{toc}{section}{4.5 Integration Techniques: Substitution}

\section{Integration Techniques: Integration by Parts}
\addcontentsline{toc}{section}{4.6 Integration Techniques: Integration by Parts}
\newpage
\section*{1 - Antidifferentiation}
\begin{mdframed}
  
In this chapter, we explore \textit{integration}, which is one of the two principal branches of calculus (\textit{differential calculus}, which we have studied in Chapters 1-3, being the other).
\bigbreak \noindent
With \textit{integral calculus}, we are able to determine accumulation of quantity based on a given rate function, for example:
\begin{itemize}[label=$\circ$]
  \item Given a function $ y = v(t)$, the velocity of an object at time $t$, we can determine an object's \textit{distance traveled} over an interval of time.
  \item Given a function $y=P'(x)$, the marginal profit of a business after $x$ units have been sold, we can determine the business's \textit{total profit} after $x$ units have been sold
  \item Given a function $y=f(t)$, the rate of change of a population after $t$ years, we can determine the \textit{total population growth} after $t$ years
\end{itemize}
\end{mdframed}
\bigbreak \noindent
\section*{Finding Antiderivatives}
One aspect of integral calculus is \textbf{Antidifferentiation}, which is the process of differentiation performed in reverse. Given a function $f$, we find another function $F$ such that
$$ \frac{d}{dx}F(x) = f(x)$$
The function $F$ is an \textbf{antiderivative} of $f$. For example, if $f(x) = 2x$, then $F(x) = x^2$ is an antiderivative of $f$ since
$$ \frac{d}{dx}(x^2) = 2x$$
Note that functions like $F(x) = x^2 +5$ and $F(x) = x^2 - 17$ are also antiderivatives of $f(x) = 2x$ since
$$ \frac{d}{dx}(x^2+5)=2x+0 =2x; \ \ \text{and} \ \frac{d}{dx}(x^2-17) = 2x+0=2x$$
Thus, an antiderivative of $f(x) =2x$ is any function that can be written in the form $F(x)=x^2+C$, where $C$ is any constant. This leads us to the following theorem.
\bigbreak \noindent
\thm{}{
  The antiderivatives of $f(x)$ is the set of functions $F(x) + C$ such that
  $$ \frac{d}{dx}\left[F(x)+C = f(x)\right]$$
  The constant $C$ is called the constant of intergration
}
\bigbreak \noindent \bigbreak \noindent
If $F$ is an antiderivative of $f$, we write
$$ \int f(x)dx =F(x)+C$$
This equation is read as ``the antiderivative of $f(x)$, with respect to $x$, is $F(x) +C$ " \\
or as `` the integral of $f(x)$, with respect to $x$ if $F(x) +C$''. The expression on the left side is called an\textbf{indefinite integral}. They symbol $\int$ is the \textit{integral sign}, and $f(x)$ is the \textit{integrand}. The symbol $dx$ can be regarded as indicating that $x$ is the variable of integration, similar to $ \frac{d}{dx}$ indicating that the expression that follows it is to be differentiated with respect to $x$.
\newpage
\q
Determine these indefinite integrals. That is, find the antiderivative of each integrand.
$$ a) \int{8}\ dx; \ \ \ \ b) \int{3x^2}\ dx; \ \ \ \int{e^x}\ dx'; \ \ \ \int{\dfrac{1}{x}}\ dx, x \neq 0$$
\bigbreak \noindent
\textbf{Problem 1.} 
$$\int{8}\ dx$$
$$ = 8x +C$$
\bigbreak \noindent
\textbf{Problem 2.}
$$ \int{3x^2}\ dx$$
$$ = x^3+C$$
\bigbreak \noindent
\textbf{Problem 3.}
$$ \int{e^x}\ dx$$
$$ = e^x +C$$
\bigbreak \noindent
\textbf{Problem 4.}
$$\int{\dfrac{1}{x}}\ dx$$
$$ = \ln{\lvert x\rvert} + C$$
\bigbreak \noindent
\nt{
  Every antiderivative can be checked by differentiation
}
\bigbreak \noindent \bigbreak \noindent
The results of Question 1 suggest some rules for antiderivatives, which are summarized in Theorem 2.
\bigbreak \noindent
\thm{Rules for Antiderivatives}{
  A1. Constant Rule:
$$
\int k d x=k x+C
$$

A2. Power Rule (where $n \neq-1$ ):
$$
\int x^n d x=\frac{1}{n+1} x^{n+1}+C .
$$

A3. Natural Logarithm Rule:
$$
\int \frac{1}{x} d x=\ln |x|+C \text {, and for } x>0, \int \frac{1}{x} d x=\ln x+C .
$$

A4. Exponential Rule (base $e$):
$$
\int e^{a x} d x=\frac{1}{a} e^{a x}+C, \quad a \neq 0 .
$$
}
\newpage
\noindent The Power Rule for Antiderivaties can be viewed as a two step process:
\bigbreak
\begin{minipage}{0.5\textwidth}
  \hspace{10mm}\incfig[1]{integ}
\end{minipage}
\begin{minipage}{0.5\textwidth}
  \hspace{-10mm}\vspace{-20mm}\begin{enumerate}
  \item Increase the exponent by 1. 
  \item Divide the term by the new power
\end{enumerate}	
\end{minipage}
\vspace{-20mm}\begin{mdframed}
  \q
Find the following indefinite integrals:
$$ a) \int{x^7}\ dx; \ \ \ \ b) \int{\sqrt{x}}\ dx; \ \ \ \ c) \int{\dfrac{1}{x^3}}\ dx;$$
\end{mdframed}
\bigbreak \noindent
\textbf{Problem 1.}
$$ \int{x^7}\ dx$$
$$ = \dfrac{x^{7+1}}{7+1} + C$$
$$ = \dfrac{1}{8}x^8 + C$$
\bigbreak \noindent
\hrule
\bigbreak \noindent
\textbf{Problem 2.}
$$ \int{\sqrt{x}}\ dx$$
$$ = \int{x^{\frac{1}{2}}}\ dx = \dfrac{x^{(1/2)+1}}{\frac{1}{2}+1}+C = \dfrac{x^{3/2}}{\frac{3}{2}}+C$$ 
$$ = \dfrac{2}{3}x^{\frac{3}{2}}+C, \ \ \ \text{or } \ \ \dfrac{2}{3}x\sqrt{x}+C$$
\bigbreak \noindent
\hrule
\bigbreak \noindent
\textbf{Problem 3.}
$$ \int{\dfrac{1}{x^3}}\ dx$$
$$ = \int{x^{-3}} = \dfrac{x^{-3+1}}{-3+1}+C = -\dfrac{1}{2}x^{-2}+C$$
$$ = -\dfrac{1}{2x^2} +C$$
\bigbreak \noindent
\hrule
\bigbreak \noindent
The Power Rule for Antiderivatives is valid for all real numbers $n$, except for $ n = -1$. for $n=-1$, we have $x^{-1} = \dfrac{1}{x}$, which is the derivative of the natural logarithm function, $ y = \ln{\lvert x\rvert}$. Therefore, 
$$ \int{\dfrac{1}{x}}\ dx = \ln{\lvert x\rvert} +C, \text{ and for } x > 0, \int{\dfrac{1}{x}}\ dx = \ln{x}+C$$
In Question 3, we explore the case of $f(x) = e^{ax}$
\newpage
\begin{mdframed}
\q
Find $$\int{e^{4x}}\ dx$$
\end{mdframed}
\sol
$$\int{e^{4x}}\ dx = \dfrac{1}{4}e^{4x}+C$$
Two useful properties of antiderivatives are presented in Theorem 6.3.
\bigbreak \noindent
\thm{}{
P1. A constant multiplier can be factored to the front of the indefinite integral:
$$
\int[c \cdot f(x)] d x=c \cdot \int f(x) d x .
$$

P2. The antiderivative of a sum or difference is the sum or difference of the antiderivatives:
$$
\int[f(x) \pm g(x)] d x=\int f(x) d x \pm \int g(x) d x
$$
}
\q
Find each antiderivative. Assume $x>0$
$$ a) \int{(3x^5+7x^2+8)}\ dx \ \ \ \ \ \ b) \int{\dfrac{4+3x+2x^4}{x}}\ dx$$
\textbf{Problem 1.}
$$ \int{3x^5}\ dx + \int{7x^2}\ dx + \int{8}\ dx$$
$$ {3(\frac{1}{6}x^6)} + 7(\frac{1}{3}x^3)+8x+C$$
$$ \dfrac{1}{2}x^6 + \dfrac{7}{3}x^3 + 8x+C$$
\bigbreak \noindent
\hrule
\bigbreak \noindent
\textbf{Problem 2.}
$$ \int{\dfrac{4+3x+2x^4}{x}}\ dx$$
$$ = \dfrac{4}{x} + \dfrac{3x}{x} + \dfrac{2x^4}{x} = \dfrac{4}{x} + 3 + 2x^3$$
\textit{\textbf{Therefore,}}
$$\int{\dfrac{4+3x+2x^4}{x}}\ dx = \int{\left(\dfrac{4}{x}+3+2x^3}\right)\ dx$$
$$ = 4\ln{x} + 3x+\dfrac{1}{2}x^4 + C$$
\newpage
\section*{Initial Conditions}
When a point that is a solution of an antiderivative is given, it is possible to solve for $C$. The given point is called an inital condition
\begin{mdframed}
  \vspace{2mm}

 Find 

 $$ \int{(2x+3)}\ dx$$
 Given that $F(1) = -2$
\end{mdframed}
\sol
\textit{\textbf{If we specify that $F'(x)  =2x+3$, then we have}}
$$ F(x) = \int{(2x+3)}\ dx = x^2 +3x+C$$
Since $F(1) =-2$, we can substiute and solve for $C$.
$$ -2 = (1)^2+3(1) +C$$
Simplifiying, we have $-2 = 4+C$, or $C = -6$
\bigbreak \noindent
Therefore, the specific antiderivative that satisfies the inital conditions is
$$ F(x) = x^2+3x-6$$
\bigbreak \noindent
\hrule
\bigbreak \noindent
\vspace{-5mm}
\section*{Applications}
\bigbreak \noindent
\vspace{-1em}
\begin{mdframed}
\q
A rock is thrown upward with the inital velocity 50 ft/sec from 10 ft above the ground has a velocity modeled by $v(t) = -32t+50$, where $t$ is the number of seconds after the rock is released and $v(t)$ is in feet per second
\bigbreak \noindent
a) Determine a distance function $h$ as a function of $t$ (in this case, ``distance'' is the height of the rock)
\bigbreak \noindent
b) Find the height and the velocity of the rock after 3 sec.
\end{mdframed}
\bigbreak \noindent
\sol 
\bigbreak \noindent
\textbf{\texit{For part a) Since distance (height) is the antiderivative of velocity, we have}}
$$ h(t) = \int{(-32t +50)}\ dt = -16t^2 +50t +C$$
The inital height, 10 ft, gives us the ordered pair (0,10) as an initial condition. We substitute $0$ for $t$ and 10 for $h(t)$, and solve for $C$

$$ 10 = -16(0)^2 + 50(0) +C$$
$$ 10=C$$
Therefore, the distance function is given by
$$ h(t) = -16t^2 +50t +10$$
\bigbreak \noindent
\textit{\textbf{For part b) to find the height of the rock after 3 sec, we substitute 3 for $t$ in the distance function}}
$$ h(3) = -16(3)^2 +50(3) + 10 = 16 \text{ft.}$$
\textit{\textbf{The velocity is}}
$$v(3) = -32(3)+50 = -46\text{ ft/sec.}$$
\newpage
\section*{2 Antiderivaties as Areas}
Integral calculus studies the \textit{accumulation} of units as the input variables increases. For example, suppose a jogger maintains a constant velocity of 5 mi/hr. As she runs, she ``accumulates'' distance. After 1 hr, she has run 5 mi. Between the first hour and the second hour, she has run 5 mi, so that for the first 2 hr, she has accumulated a distance of 10 mi.
\bigbreak \noindent
We can view accumulations graphically, as shown in the following example.
\begin{mdframed}
\q
Emma drives her motor scooter at 15 mi/hr for an extended period of time
\begin{enumerate}
  \item How far has she traveled after 1 hr. 
  \item How far has she traveled between the first hour and the second hour?
  \item How far has she traveled cumulatively over the first 2 hr?
  \item What function $f(t)$ gives Emma's total distance traveled after $t$ hours?
    \vspace{1em}
\end{enumerate}
\end{mdframed}
\sol
Emma's velocity after $t$ hours is given by $v(t)$ = 15, where $v(t)$ is in miles per hour. We graph $v$, noting that its graph is a horizontal line.
\bigbreak \noindent
\begin{minipage}{0.5\textwidth}
	
a) On the graph, we shade a rectangular area between $t=0$ and $t=1$.  \\ This area represents (1hr)$\left(15\dfrac{\text{mi}}{\text{hr}}\right) = 15 \text{ mi}$. \\ Thus, Emma has traveled 15 mi after 1 hr.
\end{minipage}
\hspace{10mm}\begin{minipage}{0.5\textwidth}	
    \incfig[1]{area7}
\end{minipage}
\begin{figure}[ht]
    \centering
    %\caption{area7}
    %\label{fig:area7}
\end{figure}
\begin{minipage}{0.5\textwidth}
b) On the graph, we shade a rectangular area between $ t =1$ and $t=2$. \\ This rectangular angle has width 1 and height 15. \\ Thus, in this 1-hr period, Emma has traveled another 15 mi	
\end{minipage}
\hspace{10mm}\begin{minipage}{0.5\textwidth}
    \incfig[1]{area9}
\end{minipage}
\begin{figure}[ht]
    \centering
    %\caption{area9}
    %\label{fig:area9}
\end{figure}
\begin{minipage}{0.5\textwidth}
c) Emma has accumulated a total of 15 + 15 = 30 mi traveled over the first 2 hr	\\ Graphically, her total distance traveled after 2 hr is the area of the rectangle between $t=0$ and $t=2$, with width 2 and height 15
\end{minipage}
\hspace{13mm}\begin{minipage}{0.5\textwidth}
	
    \incfig[1]{area12}
\end{minipage}

\pagebreak
\noindent d) We set up an input-output table for $v$, including a third column showing distance traveled in each hour and a fourth column showing total \textit{accumulated} (or \textit{cumulative}) distance traveled
\bigbreak \noindent
\begin{center}
\begin{tabular}{|c|c|c|c|}
\hline \begin{tabular}{c} 
Time $t$ \\
(in hours)
\end{tabular} & \begin{tabular}{c} 
Velocity, $v(t)$ \\
(in miles per hour)
\end{tabular} & \begin{tabular}{c} 
Distance Traveled \\
in Each Hour
\end{tabular} & \begin{tabular}{c} 
Accumulated \\
Distance Traveled
\end{tabular} \\
\hline 1 & 15 & 15 & 15 \\
2 & 15 & 15 & 30 \\
3 & 15 & 15 & 45 \\
4 & 15 & 15 & 60 \\
\hline
\end{tabular}
\end{center}
\bigbreak \noindent
The accumulated distances traveled suggest that $f(t)= 15t$ gives Emma's total distance traveled, in miles, after $t$ hours. Note that $f(t) = 15t$ is an antiderivative of $v(t) = 15$
\section*{Geometry and Areas}
Example 1 suggests that the antiderivative plays a role in determining area under a graph. For linear functions, we can use geometry to find the area under the graph of a function. Two formulas, where $b$ = base and $h$ = height, are useful:
\bigbreak \noindent
\hspace{15mm}\begin{minipage}{0.5\textwidth}
	
    \incfig[1]{square3}
\end{minipage}
\begin{minipage}{0.5\textwidth}
    \incfig[1]{triangle}
\end{minipage}
\begin{figure}[ht]
    \centering
    %\caption{square3}
    %\label{fig:square3}
\end{figure}
\begin{figure}[ht]
    \centering
    %\caption{triangle}
    %\label{fig:triangle}
\end{figure}
\vspace{-7mm}
\hrule
\bigbreak \noindent
\begin{mdframed}
\q
A toy drone flies in a straight line, and its velocity $t$ seconds after takeoff is given by $v(t) = 2t$, where $v(t)$ is in meters per second.
\begin{enumerate}
  \item Find the distance the drone has flown after 1 sec.
  \item Find the distance the drone has flown between $t=1$ sec and $t=2$ sec.
  \item Find the cumulative distance the drone has flown over the first 2 sec.
  
\end{enumerate}
\end{mdframed}
\sol
We graph $v(t) = 2t$, noting that it is a linear function. Thus, we can use geometry to find the areas under its graph
\bigbreak \noindent 
\begin{minipage}{0.5\textwidth}
  \textbf{Problem 1.} To find the distance flown after 1 sec, we form a triangular area under $v$ from $t= 0$ to $t=1$ \\ The area is given by $A =\dfrac{1}{2}bh$. Here, $b=1$ and $h=2$, so after 1 sec, the drone has flown
  $$ A = \dfrac{1}{2}\left(1 \text{ sec}\right)\left(2\dfrac{\text{m}}{\text{sec}}\right) = 1 \text{ m}$$
\end{minipage}
\hspace{10mm}\begin{minipage}{0.5\textwidth}
	
    \incfig[1]{trainfig}
\end{minipage}
\begin{figure}[ht]
    \centering
    %\caption{trainfig}
    %\label{fig:trainfig}
\end{figure}

\pagebreak
\noindent
\begin{minipage}{0.5\textwidth}
  \noindent\textbf{Problem 2.} To find the distance traveled between $t=1$ sec and $t=2$ sec, we form a trapezoidal area under v over $[1,2]$ \\ This can be regarded as a rectangle and a triangle, as shown in the figure to the right. \\ The rectangle has area (1)(2) = 2, and the triangle has area $\dfrac{1}{2}(1)(2) = 1$.\\
  Thus, the drone has flown a distance of $ 2 + 1  = 3 \text{ m}$ between $t=1$ sec and $ t=2$ sec.
\end{minipage}
\hspace{10mm}\begin{minipage}{0.5\textwidth}
    \incfig[1]{trap}
\end{minipage}
\begin{figure}[ht]
    \centering
    %\caption{trap}
    %\label{fig:trap}
\end{figure}
\bigbreak \noindent
\begin{minipage}{0.5\textwidth}
  \textbf{Problem 3.}	The cummulative distance flown after 2 sec is 1 m [from part problem 1] plus 3 m [from problem 2], or a total of 4 m. \\ We can also veiw the cummulative distance as the area of the entire triangle over [0,2], or
  $$ \dfrac{1}{2}(2)(4) = 4 \text{ m}$$
\end{minipage}
\hspace{10mm}\begin{minipage}{0.5\textwidth}
    \incfig[1]{fulltri}
\end{minipage}
\begin{figure}[ht]
    \centering
    %\caption{fulltri}
    %\label{fig:fulltri}
\end{figure}
\bigbreak \noindent
\hrule
\bigbreak \noindent
\textbf{Examples 1 and 2 in this section suggest a pattern}
\bigbreak \noindent \bigbreak \noindent
\begin{minipage}{0.5\textwidth}
The graph of $f(x) = k$, where $k$ is a constant, is a horizontal line of height $k$. The region under this graph over the interval [0,$x$] is a rectangle, and its area is
    $$ A = kx \text{ (height times base)}$$
\end{minipage}
\hspace{10mm}\begin{minipage}{0.5\textwidth}
    \incfig[1]{bigsqare}
\end{minipage}
\begin{figure}[ht]
    \centering
    %\caption{bigsqare}
    %\label{fig:bigsqare}
\end{figure}
\begin{minipage}{0.5\textwidth}
  The graph of $f(x) = mx$ is a line with slope $m$, passing through the origin. \\ The region under this graph over the interval [0,x] is a triangle, and its area is
  $$ A = \dfrac{1}{2}(x)(mx) = \dfrac{1}{2}mx^2.$$
\end{minipage}
\hspace{1mm}\begin{minipage}{0.5\textwidth}
    \incfig[1]{anotherfig4}
\end{minipage}
\begin{figure}[ht]
    \centering
    %\caption{anotherfig4}
    %\label{fig:anotherfig4}
\end{figure}
\vspace{-5em}\begin{mdframed}
\q
Green Leaf Skateboards has the following marginal-cost function for producing skateboards: For up to 50 skateboards, the cost is $\$40$ per skateboard. For quantities from 51 through 125 skateboards, the cost drops to $\$30$ per skateboard. After 125 skateboard, it drops to $\$25$ per skateboard. If $x$ represents the number of skateboards produced, we have
 $$C^{\prime}(x)=\left\{\begin{array}{lc}40, & \text { for } 0 \leq x \leq 50, \\ 30, & \text { for } 50<x \leq 125, \\ 25, & \text { for } 125<x\end{array}\right.$$
 \vspace{1mm}

\noindent Where $C'(x)$ is the cost per skateboard, in dollars. Find the total cost of producing 150 skateboards.
\end{mdframed}
\bigbreak \noindent
\sol We calculate the areas of the rectangles formed under the graph of $C'$ over the intervals 
$$[0,50],[50,125], \text{ and } [125,150]$$
\bigbreak \noindent
\begin{figure}[ht]
  \hspace{17em}    \incfig[1]{biggbox}
    %\caption{biggbox}
    %\label{fig:biggbox}
\end{figure}
\section*{Riemann Summation}
In \textbf{Riemann summation}, rectangles can be used to approximate the area under the graph of a continuous function.
\bigbreak \noindent
In the following figure, [a,b] is divided into four subintervales, each having width
$$ \Delta{x} = \dfrac{(b-a)}{4}$$
\hspace{17em}\incfig[1]{summation}
\begin{figure}[ht]
    \centering
    %\caption{summation}
    %\label{fig:summation}
\end{figure}

\pagebreak \noindent
The heights of the rectangles shown are $f(x_1), f(x_2), f(x_3)$, and $ f(x_4)$ and the area of the region under the curve is approximately the sum of the areas of the four rectangles
$$ f(x_1) \Delta{x} + f(x_2) \Delta{x} + f(x_3) \Delta{x} + f(x_4) \Delta{x} \hspace{10mm} \text{(This is the riemann sum)}$$
We can denote this sum with \textbf{summation}, or \textbf{sigma, notation}, which uses the Greek capital letter sigma, $\sum$:
$$\sum\limits_{i=1}^{4}f(x_i)\Delta{x}$$
\begin{mdframed}
\q
Express $\sum\limits_{i=1}^5h(x_i)\Delta{x}$ without using summation notation
\vspace{1mm}
\end{mdframed}
\sol We have
$$ \sum_{i=1}^5 h\left(x_i\right) \Delta x=h\left(x_1\right) \Delta x+h\left(x_2\right) \Delta x+h\left(x_3\right) \Delta x+h\left(x_4\right) \Delta x+h\left(x_5\right) \Delta x $$
\begin{mdframed}
\q
Consider the graph of $f(x) = \sqrt{4-x^2}$ over the interval [0,2]. This is a quarter-circle radius 2. USe a Riemann sum to approximate the area under the graph using 4 equally sized subintervals and then 8 equally sized subintervals. Then use geometry to find the area under $f$ over [0,2], and compare this value to your approximations
\vspace{1mm}
\end{mdframed}
\sol 
\bigbreak \noindent
\textit{\textbf{Dividing [0,2] into 4 subintervals of equal width, we have}}
$$\Delta{x} = \dfrac{2-0}{4} = \dfrac{1}{2}$$
\textit{\textbf{So, we have}}
$$ \sum\limit_{i=1}^4f(x_i)\Delta{x}$$
$$ = f(0)\cdot{\frac{1}{2}} + f\left(\frac{1}{2}\right) \cdot \frac{1}{2} + f(1) f\left(1\right) \cdot \frac{1}{2} + f\left(\frac{3}{2}\right)\cdot \frac{1}{2}$$
$$ = 3.49571$$
Thus, the area under $f$ over [0,2] is approximately 3.49571 square units. Note that this is approximation is greater than the actual area of the quarter-circle
\bigbreak \noindent
\textit{\textbf{Divding [0,2]}} into 8 subintervals of equal width, we have
$$\Delta{x} = \dfrac{2-0}{8} = \dfrac{1}{4}$$
\textit{\textbf{So, we have}}
$$\sum\limit_{i=1}^8f(x_i)\Delta{x}$$
$$ = 3.33982$$
Using 8 subintervals, we have refined the estimate of the area under $f$ over [0,2] to 3.33982 square units.

\pagebreak \noindent
\section*{Definite Integrals}
\thmcon{
  \textbf{\underline{Defintion}}
  \vspace{3mm}

  Let y = $f(x)$ be continuous and nonnegative over an interval [a,b]. A \textbf{definite integral} is the limit as $n\to\infty$ (equivalently $\Delta{x}\to 0$) of the Riemann sum of the areas of rectangles under the graph of $ y=f(x)$ over [a,b]
  $$ \text{Exact area} = \displaystyle\lim_{\Delta{x}\to 0} \ \sum\limit_{i=1}^nf(x_i)\Delta{x}$$
$$ = \int_a^bf(x)\ dx$$
}
\begin{mdframed}
\q
Find the value of
$$\int_0^2(3x+2)\ dx$$
\end{mdframed}
\sol
\bigbreak \noindent
\textit{\textbf{We sketch the graph over the interval [0,2] and note that the region is a trapezoid. Thus, we can use geometry to determine this area.}}
\begin{figure}[ht]
    \centering
    \incfig[1]{definite}
    %\caption{definite}
    %\label{fig:definite}
\end{figure}
\textit{\textbf{So,}}
$$ \int_0^2(3x+2)\ dx = 2\cdot 2 + \dfrac{1}{2}\cdot2\cdot 6$$
$$ = 10$$
\bigbreak \noindent
\hrule
\section*{Area and Definite integrals}
\q
Find the area under the graph of
$$f(x) = \dfrac{1}{5}x^2 + 3 \text{ over } [2,5]$$
\sol
\bigbreak \noindent
\textit{\textbf{Although making a drawing is not required, doing so helps visualize the problem. The interval is [2,5], so we have $a=2$ and $b=5$}}
$$\int\left(\dfrac{1}{5}x^2+3\right) = F(x)$$
$$F(x) = \dfrac{1}{15}x^3+3x+C$$ 

\pagebreak \noindent
\textit{\textbf{For simplicity, we set C = 0, so that $F(x) = \dfrac{1}{15}x^3 +3x$}}
\bigbreak \noindent
Area over $[2,5] = F(5) - F(2)$
$$ = \dfrac{1}{15}(5)^3 + 3(5) -  \left[\dfrac{1}{15}(2)^3+3(2)\right]$$
$$ = \left(\dfrac{125}{15}+15\right) - \left(\dfrac{8}{15} + 6\right)$$
$$ = 16\dfrac{4}{5}$$
\bigbreak \noindent
\q
Find the area under the graph of
$$ y=x^2+1 \text{ over } [-1,2]$$
\sol
\bigbreak \noindent
\textit{\textbf{Find the antiderivative of $f$}}
$$ F(x) = \dfrac{x^3}{3}+x $$
\nt{
  For simplicity, we set C = 0
}
\noindent
\textit{\textbf{Now we substitute the endpoints, 2 and -1, and find the difference F(2) - F(-1)}}
$$ F(2) - F(-1) = \left[\dfrac{2^3}{3} + 2\right] - \left[\dfrac{(-1)^3}{3} + (-1)\right]$$
$$ = \dfrac{8}{3} + 2 - \left[\dfrac{-1^3}{3}+(-1)\right]$$
$$ = \dfrac{8}{3}+2+\dfrac{1}{3}+1$$
$$ = 6$$
\bigbreak \noindent
\thmcon{
  \textbf{\underline{Defintion}}
  \vspace{3mm}

  Let $f$ be any continuous function over [a,b] and $F$ be any antiderivative of $f$.
  \bigbreak \noindent
  Then the \textbf{definite integral} of $f$ from $a$ to $b$ is
  $$ \int_a^bf(x)\ dx = F(b) - F(a)$$
  \bigbreak \noindent
  Where $F(x)$ is an antiderivative of $f(x)$
}

\pagebreak \noindent
\begin{mdframed}
\q
Evalulate each of the following
$$ \text{a)} \int_{-1}^4(x^2-x)\ dx; \ \ \ \text{b)} \int_0^2e^x\ dx; \ \ \ \text{c)} \int_2^5\dfrac{1}{x}\ dx;$$
\vspace{1mm}
\end{mdframed}
\bigbreka \noindent
\textbf{Problem 1.}
$$ \int_{-1}^4(x^2-x)\ dx$$
$$ =\left[\dfrac{x^3}{3} -\dfrac{x^2}{2}\right]_{-1}^4$$
\textit{\textbf{Remember that we dont care about C (C = 0)}}
$$ \left(\dfrac{(4)^3}{3} - \dfrac{(4)^2}{2}\right) - \left(\dfrac{(-1)^3}{3} - \dfrac{(-1)^3}{2}\right)$$
$$ = \left(\dfrac{64}{3} - \dfrac{16}{2}\right) - \left(\dfrac{-1}{3} - \dfrac{1}{2}\right)$$
$$ = \dfrac{64}{3} - 8 + \dfrac{1}{3} + \dfrac{1}{2} = 14\dfrac{1}{6}$$
\bigbreak \noindent
\textbf{Problem 2.}
$$ \int_0^2e^x\ dx$$
$$ = \left[e^x\right]_0^2$$
$$ e^2 - e^0$$
$$ \approx 6.389$$
\bigbreak \noindent
\textbf{Problem 3.}
$$ \int_2^5\dfrac{1}{x}\ dx$$
$$ = \left[\ln{\lvert{x}\rvert}\right]_2^5$$
$$ = \ln{\lvert{5}\rvert} - \ln{\lvert{2}\rvert}$$
$$ \approx 0.916$$

\pagebreak
\noindent
\section*{More on Area}
When we evalulate the definite integral of a nonnegative function $f$ over [a,b], we get the area under the graph $f$ over that interval
\begin{mdframed}
\q
Find the area under the graph of
$$ y = \dfrac{1}{x^2} \text{ over } [1,10]$$
\vspace{1mm}
\end{mdframed}
\begin{minipage}{0.5\textwidth}
	\hspace{10mm}$\begin{aligned} \int_1^{10} \frac{d x}{x^2} & =\int_1^{10} x^{-2} d x \\ & =\left[\frac{x^{-2+1}}{-2+1}\right]_1^{10} \\ & =\left[\frac{x^{-1}}{-1}\right]_1^{10}=\left[-\frac{1}{x}\right]_1^{10} \\ & =\left(-\frac{1}{10}\right)-\left(-\frac{1}{1}\right) \\ & =1-\frac{1}{10}=0.9\end{aligned}$
\end{minipage}
\begin{minipage}{0.5\textwidth}
    \incfig[1]{gillles}
\end{minipage}
\begin{figure}[ht]
    \centering
    %\caption{gillles}
    %\label{fig:gillles}
\end{figure}
\bigbreak \noindent
\hrule
\bigbreak \noindent
\section*{4.5 - Integration Techniques: Substitution}
The following formulas provide a basis for an integration technique called \textbf{substitution}, a process that is, as we will see, the reverse of differentiation using the Chain Rule.
\bigbreak \noindent
\begin{mdframed}
A. $\int u^r d u=\frac{u^{r+1}}{r+1}+C$, assuming $r \neq-1$
\bigbreak \noindent
B. $\int e^u d u=e^u+C$
\bigbreak \noindent
C. $\int \frac{1}{u} d u=\ln |u|+C ; \quad$ and $\quad \int \frac{1}{u} d u=\ln u+C, \quad u>0$
\end{mdframed}
\bigbreak \noindent
In the above formulas, the variable $u$ represents some function of $x$ and $du$ is the derivative of $u$ with respect to $x$. Recall that we solve $\int_x^7\ dx$ using the Power Rule for Antiderivaties:
$$ \int{x^7}\ dx = \dfrac{x^7+1}{7+1} +C = \dfrac{x^8}{8}+C. \ \ \text{or } \ \dfrac{1}{8}x^8 +C$$
But, what about an integral like $\int(3x-4)^7\ dx$? Suppose we thought the antiderivative was
$$ \dfrac{(3x-4)^8}{8}+C$$
If we check by differentiating, we get
$$ 8\cdot\dfrac{1}{8}\cdot(3x-4)^7\cdot3\cdot dx$$
This simplifies to
$$ 3(3x-4)^7, \ \ \text{not } (3x-4)^7$$

\pagebreak
\noindent
To correct our antiderivative, let's make this substitution:
$$ u = 3x-4$$
Then $ \dfrac{du}{dx} = 3$, and recalling our work with differentials, we have
$$ du = 3 \cdot dx, \text{ and } \ \dfrac{du}{3} = dx$$
With \textit{substitution}, our original integral, $\int(3x-4)^7\ dx$, takes the form
$$ \int(3x-4)^7\ dx = \int{u^7} \cdot \dfrac{du}{3}$$
$$ = \dfrac{1}{3} \cdot \int{u^7}\ du$$
$$ = \dfrac{1}{3} \cdot\dfrac{u^8}{8} + C$$
$$ = \dfrac{1}{3\cdot 8}\cdot (3x-4)^8 +C = \dfrac{1}{24}(3x-4)^8+C$$
\begin{mdframed}
\q
Find $dy$ for each function
\bigbreak \noindent
a) $y=f(x)=x^3$
\vspace{2mm}\noindent

\noindent
b) $y=f(x)=x^{2 / 3}$;
\vspace{2mm}\noindent

\noindent
c) $y=g(x)=\ln x$;
\vspace{2mm}\noindent

\noindent 
d) $y=f(x)=e^{x^2}$
\vspace{1mm}
\end{mdframed}
\bigbreak \noindent
\textbf{Problem 1.}
$$ y=f(x) = x^3$$
$$ = \dfrac{dy}{dx}x^3$$
$$ = 3x^2$$
\bigbreak \noindent
\textbf{Problem 2.}
$$ y = f(x) = x^{\frac{2}{3}}$$
$$ dy = f'(x)\ dx = \dfrac{2}{3}x^{\frac{-1}{3}}\ dx$$
\bigbreak \noindent
\textbf{Problem 3.}
\bigbreak \noindent
\textit{\textbf{We have}}
$$ \dfrac{dy}{dx} = g'(x) = \dfrac{1}{x}$$
\textit{\textbf{So,}}
$$ dy = g'(x)\ dx = \dfrac{1}{x}\ dx, \ \ \text{ or } \dfrac{dx}{x}$$

\pagebreak
\noindent
\begin{mdframed}
\q
Evalulate
$$ \int 3x^2(x^3 +1)^{10}\ dx$$
\end{mdframed}
\bigbreak \noindent
\sol
\bigbreak \noindent
\textit{\textbf{We let}}
$$ u = x^3+1$$
\textit{\textbf{So, we have}}
$$ u = x^3 +1$$
$$ du = 3x^2\ dx$$
\textit{\textbf{We now solve for dx}}
$$ dx = \dfrac{du}{3x^2}$$
\textit{\textbf{Now we have}}
$$ \int 3x^2(u)^{10}\cdot \dfrac{du}{3x^2}$$
\textit{\textbf{$3x^2$ cancels so we are left with}}
$$\int{u^{10}}\ du$$
\textit{\textbf{Now integrate}}
$$ \int{u^{10}} = \dfrac{u^{11}}{11} + C$$
\textit{\textbf{Reversing the substitution we get}}
$$ \dfrac{1}{11} \left(x^3+1\right)^{11} + C$$
\nt{
  to find du, we take the derivative of u
}
\bigbreak \noindent
\newpage
\section*{Integration Techniques: Integration by Parts}
Let $y=u(x)$ and $y=v(x)$ be two functions. Applying the Product Rule, we have
$$ \frac{d}{dx}(u(x)) \cdot v(x)) = u(x) \cdot \frac{d}{dx}v(x) +v(x) \cdot \frac{d}{dx}u(x)$$
Integrating both sides with respect to x, we have
$$
\int\left[\frac{d}{d x}(u(x) \cdot v(x))\right] d x=\int\left[u(x) \cdot \frac{d}{d x} v(x)\right] d x+\int\left[v(x) \cdot \frac{d}{d x} u(x)\right] d x,
$$
Note that 
$$\int\left[ \frac{d}{dx}(u(x)\cdot v(x))\right]\ dx = u(x) \cdot v(x)$$
We simplify by writing $u$ for $u(x)$
$v$ for $v(x), du$ for $ \frac{d}{dx}u(x)$ and $dv$ for $ \frac{d}{dx}v(x)\ dx$:
$$ uv = \int{u\ dv} + \int{v\ du}$$
Solving for $\int{u\ dv}$, we obtain the following theorem
\thm{}{
  $$\int{u\ dv} = uv - \int{v\ du}$$
}
\bigbreak \noindent
\q
Evaluate:
$$ \int{xe^x}\ dx$$
\sol
\textit{\textbf{We let}}
$$ u = x \ \ \ \text{and } \ dv = e^x\ dx$$
\textit{\textbf{In this case, differentiating $u$ gives}}
$$ du = dx$$
\textit{\textbf{and integrating dv gives}}
$$ v = e^x$$
\textit{\textbf{The integrating-by-Parts Formula gives us}}
$$ \int (x)(e^x\ dx) = (x)(e^x) - \int (e^x)(dx)$$
$$ = xe^x - e^x +C$$

\end{document}




% --- LaTeX Homework Template - S. Venkatraman ---

% --- Set document class and font size ---

\documentclass[letterpaper, 11pt]{article}

% --- Package imports ---

\usepackage[]{mdframed}

\usepackage{
  booktabs, amsmath, amsthm, amssymb, mathtools,	  % Math typesetting
  graphicx, wrapfig, subfig, float,                  % Figures and graphics formatting
  listings, color,  pythonhighlight,     % Code formatting
  fancyhdr, sectsty, hyperref, enumerate, enumitem } % Headers/footers, section fonts, links, lists

% --- Page layout settings ---

% Set page margins
\usepackage[left=1in, right=1in, bottom=1in, top=1.1in, headsep=0.2in]{geometry}

% Anchor footnotes to the bottom of the page
\usepackage[bottom]{footmisc}

% Set line spacing
\renewcommand{\baselinestretch}{1.2}

% Set spacing between paragraphs
\setlength{\parskip}{1.5mm}

% Allow multi-line equations to break onto the next page
\allowdisplaybreaks

% Enumerated lists: make numbers flush left, with parentheses around them
\setlist[enumerate]{wide=0pt, leftmargin=21pt, labelwidth=0pt, align=left}
\setenumerate[1]{label={(\arabic*)}}

% --- Page formatting settings ---

% Set link colors for labeled items (blue) and citations (red)
\hypersetup{colorlinks=true, linkcolor=blue, citecolor=red}

% Make reference section title font smaller
\renewcommand{\refname}{\large\bf{References}}

% --- Settings for printing computer code ---

% Define colors for green text (comments), grey text (line numbers),
% and green frame around code
\definecolor{greenText}{rgb}{0.5, 0.7, 0.5}
\definecolor{greyText}{rgb}{0.5, 0.5, 0.5}
\definecolor{codeFrame}{rgb}{0.5, 0.7, 0.5}

% Define code settings
\lstdefinestyle{code} {
  frame=single, rulecolor=\color{codeFrame},            % Include a green frame around the code
  numbers=left,                                         % Include line numbers
  numbersep=8pt,                                        % Add space between line numbers and frame
  numberstyle=\tiny\color{greyText},                    % Line number font size (tiny) and color (grey)
  commentstyle=\color{greenText},                       % Put comments in green text
  basicstyle=\linespread{1.1}\ttfamily\footnotesize,    % Set code line spacing
  keywordstyle=\ttfamily\footnotesize,                  % No special formatting for keywords
  showstringspaces=false,                               % No marks for spaces
  xleftmargin=1.95em,                                   % Align code frame with main text
  framexleftmargin=1.6em,                               % Extend frame left margin to include line numbers
  breaklines=true,                                      % Wrap long lines of code
  postbreak=\mbox{\textcolor{greenText}{$\hookrightarrow$}\space} % Mark wrapped lines with an arrow
}

% Set all code listings to be styled with the above settings
\lstset{style=code}

% --- Math/Statistics commands ---

% Add a reference number to a single line of a multi-line equation
% Usage: "\numberthis\label{labelNameHere}" in an align or gather environment
\newcommand\numberthis{\addtocounter{equation}{1}\tag{\theequation}}

% Shortcut for bold text in math mode, e.g. $\b{X}$
\let\b\mathbf

% Shortcut for bold Greek letters, e.g. $\bg{\beta}$
\let\bg\boldsymbol

% Shortcut for calligraphic script, e.g. %\mc{M}$
\let\mc\mathcal

% \mathscr{(letter here)} is sometimes used to denote vector spaces
\usepackage[mathscr]{euscript}

% Convergence: right arrow with optional text on top
% E.g. $\converge[w]$ for weak convergence
\newcommand{\converge}[1][]{\xrightarrow{#1}}

% Normal distribution: arguments are the mean and variance
% E.g. $\normal{\mu}{\sigma}$
\newcommand{\normal}[2]{\mathcal{N}\left(#1,#2\right)}

% Uniform distribution: arguments are the left and right endpoints
% E.g. $\unif{0}{1}$
\newcommand{\unif}[2]{\text{Uniform}(#1,#2)}

% Independent and identically distributed random variables
% E.g. $ X_1,...,X_n \iid \normal{0}{1}$
\newcommand{\iid}{\stackrel{\smash{\text{iid}}}{\sim}}

% Equality: equals sign with optional text on top
% E.g. $X \equals[d] Y$ for equality in distribution
\newcommand{\equals}[1][]{\stackrel{\smash{#1}}{=}}

% Math mode symbols for common sets and spaces. Example usage: $\R$
\newcommand{\R}{\mathbb{R}}   % Real numbers
\newcommand{\C}{\mathbb{C}}   % Complex numbers
\newcommand{\Q}{\mathbb{Q}}   % Rational numbers
\newcommand{\Z}{\mathbb{Z}}   % Integers
\newcommand{\N}{\mathbb{N}}   % Natural numbers
\newcommand{\F}{\mathcal{F}}  % Calligraphic F for a sigma algebra
\newcommand{\El}{\mathcal{L}} % Calligraphic L, e.g. for L^p spaces

% Math mode symbols for probability
\newcommand{\pr}{\mathbb{P}}    % Probability measure
\newcommand{\E}{\mathbb{E}}     % Expectation, e.g. $\E(X)$
\newcommand{\var}{\text{Var}}   % Variance, e.g. $\var(X)$
\newcommand{\cov}{\text{Cov}}   % Covariance, e.g. $\cov(X,Y)$
\newcommand{\corr}{\text{Corr}} % Correlation, e.g. $\corr(X,Y)$
\newcommand{\B}{\mathcal{B}}    % Borel sigma-algebra

% Other miscellaneous symbols
\newcommand{\tth}{\text{th}}	% Non-italicized 'th', e.g. $n^\tth$
\newcommand{\Oh}{\mathcal{O}}	% Big-O notation, e.g. $\O(n)$
\newcommand{\1}{\mathds{1}}	% Indicator function, e.g. $\1_A$

% Additional commands for math mode
\DeclareMathOperator*{\argmax}{argmax}    % Argmax, e.g. $\argmax_{x\in[0,1]} f(x)$
\DeclareMathOperator*{\argmin}{argmin}    % Argmin, e.g. $\argmin_{x\in[0,1]} f(x)$
\DeclareMathOperator*{\spann}{Span}       % Span, e.g. $\spann\{X_1,...,X_n\}$
\DeclareMathOperator*{\bias}{Bias}        % Bias, e.g. $\bias(\hat\theta)$
\DeclareMathOperator*{\ran}{ran}          % Range of an operator, e.g. $\ran(T) 
\DeclareMathOperator*{\dv}{d\!}           % Non-italicized 'with respect to', e.g. $\int f(x) \dv x$
\DeclareMathOperator*{\diag}{diag}        % Diagonal of a matrix, e.g. $\diag(M)$
\DeclareMathOperator*{\trace}{trace}      % Trace of a matrix, e.g. $\trace(M)$

% Numbered theorem, lemma, etc. settings - e.g., a definition, lemma, and theorem appearing in that 
% order in Section 2 will be numbered Definition 2.1, Lemma 2.2, Theorem 2.3. 
% Example usage: \begin{theorem}[Name of theorem] Theorem statement \end{theorem}
\theoremstyle{definition}
\newtheorem{theorem}{Theorem}[section]
\newtheorem{proposition}[theorem]{Proposition}
\newtheorem{lemma}[theorem]{Lemma}
\newtheorem{corollary}[theorem]{Corollary}
\newtheorem{definition}[theorem]{Definition}
\newtheorem{example}[theorem]{Example}
\newtheorem{remark}[theorem]{Remark}

% Un-numbered theorem, lemma, etc. settings
% Example usage: \begin{lemma*}[Name of lemma] Lemma statement \end{lemma*}
\newtheorem*{theorem*}{Theorem}
\newtheorem*{proposition*}{Proposition}
\newtheorem*{lemma*}{Lemma}
\newtheorem*{corollary*}{Corollary}
\newtheorem*{definition*}{Definition}
\newtheorem*{example*}{Example}
\newtheorem*{remark*}{Remark}
\newtheorem*{claim}{Claim}

% --- Left/right header text (to appear on every page) ---

% Include a line underneath the header, no footer line
\pagestyle{fancy}
\renewcommand{\footrulewidth}{0pt}
\renewcommand{\headrulewidth}{0.4pt}

% Left header text: course name/assignment number
\lhead{CSCI 466 (Databases) -- Assignment 3 - Normalization}

% Right header text: your name
\rhead{Matt Warner}

% --- Document starts here ---

\usepackage{pifont}

\begin{document}
\begin{mdframed}
  
\subsection*{Question 1}
  
R(A, B, C, D, E, F, G, H) \\
\textit{\textbf{Functional Dependencies:}}
\begin{itemize}
    \item[\ding{228}] A $\rightarrow$ D, E
    \item[\ding{228}] C $\rightarrow$ G
    \item[\ding{228}] A, C $\rightarrow$ H, F
\end{itemize}
\end{mdframed}
\underline{\texttt{Part 1:}} is this relation in $_1NF$? \\
Yes, This relation is in $_1NF$. There are no repeating groups here.
\bigbreak \noindent
\underline{\texttt{Part 2:}} is this relation in $_2NF$? \\
Since we have yet to establish a primary key, we can't determine if our relation is in $_2NF$. Our only real candidate key here is \{\underline{A}, \underline{C}, \underline{B}\}. So we make that our primary key.
\bigbreak \noindent
Now that we have our primary key, we can clearly see that we our relation is not in $_2NF$ since two subsets of our primary key (\underline{A} and \underline{C}) are  determinates of non-prime attributes. \vspace{2mm} \\
$A \rightarrow D, \ E$ (violates $_2NF$). \\
$C  \rightarrow G$ (violates $_2NF$).
\bigbreak \noindent
By performing decomposition, we can move to $_2NF$. We need to perform decomposition on all functional dependencies that violate $_2NF$. There are three cases that violates $_2NF$, so after decomposition we will be left with four relations instead of one. 
\bigbreak \noindent
\textbf{Schema in $_2NF$:}
 \begin{itemize}
     \item[\ding{221}] R$_1$(\underline{A}, \underline{B}, \underline{C},)
     \item[\ding{221}] R$_2$(\underline{A}, D, E)
     \item[\ding{221}] R$_3$(\underline{C}, G)
     \item[\ding{221}] R$_4$(\underline{A}, \underline{C},H,F)

 \end{itemize}

\bigbreak \noindent
\underline{\texttt{Part 3:}} is this relation in $_3NF$? \\
Yes, we are already in $_3NF$
\begin{mdframed}
  
\subsection*{Question 2}
Property(id, county, lotNum, lotArea, price, taxRate, (dataPaid, amount)) \vspace{1.5mm} \\
\textit{\textbf{Functional Dependencies:}}
\begin{itemize}
    \item[\ding{228}]  id $\rightarrow$ county, lotNum, lotArea, price, taxRate
    \item[\ding{228}] lotArea $\rightarrow$ price
    \item[\ding{228}] county $\rightarrow$ taxRate
    \item[\ding{228}] id, dataPaid $\rightarrow$ amount
\end{itemize}
\end{mdframed}
\underline{\texttt{Part 1:}} is this relation in $_1NF$? \\
No, there are repeating groups here, so we are not in $_1NF$. To fix this,
we need to select a primary key that makes every value atomic.
\bigbreak \noindent
Making the primary key \texttt{\{\underline{id}, \underline{dataPaid}\}} brings us to $_1NF$. Our new schema and table now look like this. \\
\textit{\textbf{Property}}(\underline{id}, county, lotNum, lotArea, price, taxRate, \underline{dataPaid}, amount).
    \begin{figure}[H]
    \centering
     % \setlength{\tabcolsep}{30}
    \begin{tabular}{l l l l l l l l}
        \hline
        \underline{id}& county& lotNum& lotArea& price& taxRate& \underline{datePaid}& amount. \\
        \hline
        001& Will & 59 & G5 & \$3500 &1\%& 02-05-2012& \$1200 \\
        001& Will & 59 & G5 & \$3500 &1\%& 02-20-2012& \$2300 \\
        \hline
    \end{tabular}
    \end{figure}
\bigbreak \noindent
\underline{\texttt{Part 2:}} is this relation in $_2NF$? \\
No, this relation is not in $_2NF$. The current state of our relation does not have full dependency. Looking at our functional dependencies, we see that a subset of our primary key is the determinent of some non-prime attributes. This violates the rules of $_2NF$. \vspace{2mm}\\
id $\rightarrow$ county, lotNum, lotArea, price, taxRate. (violates $_2NF$).
\bigbreak \noindent
To fix this, we should use decomposition, which splits our relation into two tables: \\
\textit{\textbf{Property}}(\underline{id}, county, lotNum, lotArea, price, taxRate)
    \begin{figure}[H]
    \centering
     % \setlength{\tabcolsep}{30}
    \begin{tabular}{l l l l l l}
   \hline 
        \underline{id}& county& lotNum& lotArea& price& taxRate \\
        \hline
    \end{tabular}
    \end{figure}
    \noindent \textit{\textbf{Payments}}(\underline{id}, datePaid, amount)
        \begin{figure}[H]
        \centering
         \setlength{\tabcolsep}{45}
        \begin{tabular}{l l l}
            \hline
            \underline{id} & \underline{datePaid}&amount \\
        \hline
        \end{tabular}
        \end{figure}


        \newpage \noindent
\underline{\texttt{Part 3:}} is this relation in $_3NF$? \\
No, We are not in $_3NF$ because there are some transitive dependencies in our list of functional dependencies. county functionally determining taxRate, and lotArea functionally determining price are both transitive dependencies, which violates $_3NF$. \vspace{1mm} \\
lotArea $\rightarrow$ price (violates $_3NF$) \\
county $\rightarrow$ taxRate (violates $_3NF$)
\bigbreak \noindent
To fix this, we need to perform decomposition on all cases that violate $_3NF$. That leaves us with: \vspace{1mm}\\
\textit{\textbf{Property}}(\underline{id}, lotNum). \\
\noindent \textit{\textbf{Payments} }(\underline{id}, \underline{datePaid}, amount) \\
\textit{\textbf{LandInfo}}(\underline{lotArea}, price) \\ 
\textit{\textbf{TaxInfo}}(\underline{county}, taxRate)
\subsection*{Question 3}
\begin{mdframed}
Pharmacy(patient\_id, patient\_name, address, (Rx\_num, trademark\_name, generic\_name, (filldate, num\_refills\_left), num\_refills)) \vspace{1.5mm}\\
\textit{\textbf{Functional Dependencies:}}
\begin{itemize}
    \item[\ding{228}] patient\_id $\rightarrow$ patient\_name, address
    \item[\ding{228}] patient\_id, Rx\_num $\rightarrow$ trademark\_name, generic\_name
        \item[\ding{228}] Rx\_num $\rightarrow$ num\_refills
        \item[\ding{228}] Rx\_num, filldate $\rightarrow$ num\_refills\_left
\end{itemize}
\end{mdframed}
\underline{\texttt{Part 1:}} is this relation in $_1NF$? \\
No, This relation is not in $_1NF$. There are many repeating groups here, so not all values are atomic. We can move to $_1NF$ by selecting a primary key that makes all values atomic.
\bigbreak \noindent
The only real candidate key here is \{patient\_id, Rx\_num, filldate\}. Since it is the only candidate, we will make it the primary key. 
\bigbreak \noindent
\textit{Our new schema is:} \vspace{2mm}\\
Pharmacy(\underline{patient\_id}, patient\_name, address, \underline{Rx\_num}, trademark\_name, generic\_name, \underline{filldate}, num\_refills\_left, num\_refills)
\bigbreak \noindent
\underline{\texttt{Part 2:}} is this relation in $_2NF$? \\
This relation is not in $_2NF$ since we have subsets of the primary key that are able to functionally determine non-prime attributes.  \vspace{2mm} \\
patient\_id $\rightarrow$ patient\_name, address (violates $_2NF$) \\ 
patient\_id, Rx\_num $\rightarrow$ trademark\_name, generic\_name (violates $_2NF$) \\
Rx\_num, $\rightarrow$ num\_refills (violates $_2NF$) \\
Rx\_num, filldate $\rightarrow$ num\_refills\_left (violates $_2NF$)
\bigbreak \noindent
After decomposition, we have:
\bigbreak \noindent
\textbf{Schema in $_2NF$:} 
\begin{itemize}
    \item[\ding{221}] \textit{\textbf{Pharmacy}}(\underline{patient\_id}, \underline{Rx\_num}, \underline{fill\_date})
    \item[\ding{221}] \textit{\textbf{Patient}}(\underline{patient\_id}, patient\_name, address)
    \item[\ding{221}]  \textit{\textbf{Prescriptions}}(\underline{patient\_id}, \underline{Rx\_num}, trademark\_name, generic\_name)
    \item[\ding{221}] \textit{\textbf{TotalRefills}}(\underline{Rx\_num}, num\_refills)
    \item[\ding{221}] \textit{\textbf{RemainingRefills}}(\underline{Rx\_num}, \underline{filldate}, num\_refills\_left)
\end{itemize}
\bigbreak \noindent
\underline{\texttt{Part 3:}} is this relation in $_3NF$? \\
Yes, we are already in $_3NF$.
\subsection*{Question 4}
\begin{mdframed}
Company(EmpID, EmpName, EmpAddr, (ProjID, ProjName, MgrID, MgrName, HoursWorked)) \vspace{1.5mm} \\
\textit{\textbf{Functional Dependencies:}} 
\begin{itemize}
    \item[\ding{228}] EmpID $\rightarrow$ EmpName, EmpAddr
    \item[\ding{228}] ProjID $\rightarrow$ ProjName, MgrID, MgrName
        \item[\ding{228}]  EmpID, ProjID $\rightarrow$ HoursWorked
        \item[\ding{228}] MgrID $\rightarrow$ MgrName
\end{itemize}
\end{mdframed}
\underline{\texttt{Part 1:}} is this relation in $_1NF$? \\
This relation is not in $_1NF$ since there are repeating groups present. We need a primary key that is able to functionally determine all other attributes. The primary key we should go with in this case is: \{EmpID, ProjID\}. \vspace{2mm}\\
This primary key gives us the schema: \\
Company(\underline{EmpID}, EmpName, EmpAddr, \underline{ProjID}, ProjName, MgrID, MgrName, HoursWorked)
\bigbreak \noindent
\underline{\texttt{Part 2:}} is this relation in $_2NF$? \\
This relation is not in $_2NF$ since some subsets of the primary key can functionally determine non-prime attributes. \vspace{2mm} \\
EmpID $\rightarrow$ EmpName, EmpAddr (violates $_2NF$) \\
ProjID $\rightarrow$ ProjName, MgrID, MgrName (violates $_2NF$) \vspace{2mm}\\
We can fix this by performing decomposition on every case that violates $_2NF$. This leaves us with the following schema:
\begin{itemize}
    \item[\ding{228}] \textit{\textbf{Company}}(\underline{EmpID}, \underline{ProjID},HoursWorked)
    \item[\ding{228}] \textit{\textbf{Employee}}(\texttt{\underline{EmpID}}, EmpName, EmpAddr)
    \item[\ding{228}] \textit{\textbf{Project}}(\underline{\texttt{ProjID}}, ProjName, MgrID, MgrName)
\end{itemize}


\bigbreak \noindent
\underline{\texttt{Part 3:}} is this relation in $_3NF$? \\
No, This relation is not in $_3NF$ since we have a non-prime attribue (\texttt{MgrID}), that can functionally determine another non-prime attribute (MgrName). To move into $_3NF$, we need to perform more decomposition. We need to remove \texttt{MgrID} and \texttt{MgrName} from the \textbf{Project} relation, and move it into a new relation. That leaves us with:
\begin{itemize}
    \item[\ding{228}] \textit{\textbf{Company}}(\underline{EmpID}, \underline{ProjID},HoursWorked)
    \item[\ding{228}] \textit{\textbf{Employees}}(\texttt{\underline{EmpID}}, EmpName, EmpAddr)
    \item[\ding{228}] \textit{\textbf{Projects}}(\underline{\texttt{ProjID}}, ProjName)
    \item[\ding{228}] \textit{\textbf{Managers}}(\underline{MgrID}, MgrName)
\end{itemize}

\subsection*{Question 5}
\begin{mdframed}
StockExchange(Company, Symbol, HQ, Date, ClosePrice) \vspace{2mm} \\
\textit{\textbf{Functional Dependencies:}}
\begin{itemize}
    \item[\ding{228}] Symbol, Date $\rightarrow$ Company, HQ, ClosePrice
    \item[\ding{228}] Symbol $\rightarrow$ Company, HQ
    \item[\ding{228}] Symbol $\rightarrow$ HQ
\end{itemize}
\end{mdframed}
\underline{\texttt{Part 1:}} is this relation in $_1NF$? \\
There are no repeating groups present, so this relation is in $_1NF$.
\bigbreak \noindent
However, to move on any further, we need a primary key so we should establish one. Our only candidate key here is \{Symbol, Date\}. We will make this our primary key. Our new schema is: \vspace{2mm} \\
StockExchange(Company, \underline{Symbol}, HQ, \underline{Date}, closePrice)
\bigbreak \noindent
\underline{\texttt{Part 2:}} is this relation in $_2NF$? \\
This relation is not in $_2NF$ since a subset of the primary key is a determinent of non-key attributes.

\noindent Symbol $\rightarrow$ Company, HQ (violates $_2NF$) \\
\noindent Symbol $\rightarrow$ HQ (violates $_2NF$)
\bigbreak \noindent
After performing the decomposition steps for every violation of $_2NF$ (two in this case), we are left with three relations: \vspace{2mm} \\
StockExchange(\underline{Symbol}, \underline{Date}, closePrice) \\
StockCompany(\underline{Symbol}, Company, HQ) \\
StockLocation(\underline{Symbol}, HQ)

\bigbreak \noindent
\underline{\texttt{Part 3:}} is this relation in $_3NF$? \\
Yes, this relation is in $_3NF$
\end{document}

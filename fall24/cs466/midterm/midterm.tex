\documentclass{report}

\input{~/latex/template/preamble.tex}
\input{~/latex/template/macros.tex}

\title{\Huge{Midterm Review}}
\author{\huge{Matt Warner}}
\date{\huge{}}
\pagestyle{fancy}
\fancyhf{}
\rhead{}
\lhead{\leftmark}
\cfoot{\thepage}
% \usepackage[default]{sourcecodepro} \usepackage[T1]{fontenc}
\usepackage{pifont}

\pgfpagesdeclarelayout{boxed}
{
  \edef\pgfpageoptionborder{0pt}
}
{
  \pgfpagesphysicalpageoptions
  {%
    logical pages=1,%
  }
  \pgfpageslogicalpageoptions{1}
  {
    border code=\pgfsetlinewidth{1.5pt}\pgfstroke,%
    border shrink=\pgfpageoptionborder,%
    resized width=.95\pgfphysicalwidth,%
    resized height=.95\pgfphysicalheight,%
    center=\pgfpoint{.5\pgfphysicalwidth}{.5\pgfphysicalheight}%
  }%
}

\pgfpagesuselayout{boxed}

\begin{document}
    \maketitle
\chapter{Overview}
\section{Vocabulary}
\begin{itemize}
    \item[\ding{228}] You're responsible for knowing the terms used in the lectures.
    \item[\ding{228}] There will be some questions specifically on vocabulary.
    \item[\ding{228}] Other questions will use the terms as part of a larger question \ - \ they will not be redefined during the test.
\end{itemize}
\section{ER Diagrams}
\begin{itemize}
    \item[\ding{228}] Be able to read and interpret them.
    \item[\ding{228}] Provide cardinalities/connectivities.
    \item[\ding{228}] Degree of a relationship.
    \item[\ding{228}] Recursive relationships.
    \item[\ding{228}] Inheritance \ - \ specialization/generalization, overlapping/disjoing subtypes.
\end{itemize}
\section{Logical Data Model \ - \ Relational Model}
\begin{itemize}
    \item[\ding{228}] What makes up the relational model?
    \item[\ding{228}]  Relations, tuples, attributes, etc.
    \item[\ding{228}] Keys - primary key, candidate key, superkey.
    \item[\ding{228}] Know the requirements for each type of key.
    \item[\ding{228}] How do the above keys work with respect to functional dependencies?
    \item[\ding{228}]  Foreign Keys \ - \ what are they and how do they work?
    \item[\ding{228}]  Entity Integrity Constraint.
    \item[\ding{228}] Referential Integrity Constraint.
    \item[\ding{228}] Conversion of ER diagrams to $_3NF$ relations.
\end{itemize}	
\section{Normalization}
\begin{itemize}
    \item[\ding{228}] Be able to identify and fix violations of all 3
        \begin{itemize}[label=$\circ$]
            \item $1_NF$ \ - \ no repeating groups, all values must be atomic.
            \item $2_NF$ \ - \ No partial key dependencies.
            \item $_3NF$ \ - \ No transitive dependencies.
        \end{itemize}
\end{itemize}
\section{SQL}
\begin{itemize}
    \item[\ding{228}] Know the commands and be able to use them to make queries.
        \begin{itemize}[label=$\circ$]
            \item DDL \ - \ \texttt{CREATE TABLE, ALTER TABLE, DROP TABLE, (DESCRIBE)} 
            \item DML \ - \ \texttt{INSERT, UPDATE, DELETE, SELECT}
\end{itemize}
\item[\ding{228}] Be sure to know which commands are DDL vs . DML
\item[\ding{228}] \texttt{LIKE operator}
\item[\ding{228}] \texttt{ORDER BY}
\item[\ding{228}] \texttt{DISTINCT}
\item[\ding{228}] \texttt{GROUP BY}
\item[\ding{228}] Group functions
\item[\ding{228}] Subqueries 
    \begin{itemize}[label=$\circ$]
        \item How they get evaluated.
        \item Multi-rpw
        \item Single-value
    \end{itemize}
\end{itemize} 
\chapter{Vocabulary}
\section{Databases}
    \subsection*{Generic Database terms:}
\begin{itemize}
    \item Enterprise \ - \ A generic term for any reasonably large-scale commerical, scientific, technical, or other application.
    \item Operational data \ - \ Data maintained about the operation of an enterprise. Stuff like products, accounts, patients, students, plans. DOES NOT INCLUDE input/output data.
    \item DBMS \ - \ A collection of programs that enables users to create and maintain a database, this collection of programs forms a general-purpose software system that facilitates
        \begin{itemize}[label=$\circ$]
            \item Definition of databases.
            \item Construction of databases.
            \item manipulation of data within a database.
            \item Sharing of data between users/applications.
        \end{itemize}
\end{itemize}
\subsection*{Capabilities of a DBMS (Terms)}
    \item \textbf{Transaction management} \ - \  A feature that provides correct, concurrent access to the database, possibly by many users at the same time.
    \item \textbf{Access Control} \ - \ The ability to limit access to data by unauthorized users along with the capability to check the validity of the data.
    \item \textbf{Resiliency} \ - \ The ability to recover from system failures without losing data.

        \subsection*{Leveled Architecture of a DBMS(Terms)}
    \item \textbf{External Level} \ - \ A view or sub-schema. This is a portion of the logical database.
    \item \textbf{Logical Level} \ - \ Abstraction of the real world as it pertains to the users of the database.
    \item \textbf{Physical Level} \ - \ The collection of files and indicies stored on secondary storage device (HDD, SSD, etc.). This is the actual data.
    \subsection*{Basic Database Terminology}
\item \textbf{Instance} \ - \ An instance of the database is the actual contents of the data. An instance of the database could be an extension of the database, the current state of the database, or even a snapshot of the data at a given point in time.
\item \textbf{Schema} \ - \ The schema of a database a description of what data can be stored. This can be though of as the data members of a class. We have the names, which tell us what what the values are going to be but we dont have the values.
\item \textbf{Data Independence} \ - \ A property of an appropriately designed database system. This has to do with the Leveled Architecture, and the mapping of logical level to physical level, and logical to external.
    \begin{itemize}[label=$\circ$]
        \item \textbf{Physical data independence} \ - \ Physical schema can be changed without modifying logical schema.
        \item \textbf{Logical data independence } \ - \ Logical schema can be changed without having to modify any of the external views.
    \end{itemize}
    \item \textbf{DCL} \ - \ The Data control language.
    \item  \textbf{DDL} \ - \ Data definition language.
    \item \textbf{DMl} \ - \ Data manipulation language.
        \section{ER Diagrams}
        \subsection*{Data Models}
    \item \textbf{Data Models} \ - \ A means of describing the structure of data. An ER Diagram is a conceptual data model, but we can also have logical data models (relational, network, hierarchical, inverted list, or object-oriented). The logical data model used in this course is the relational model.
    \begin{itemize}[label=$\circ$]
        \item \textbf{Conceptual Data Model} \ - \ Shows the structure of the data including how things are related. This is a communication tool and is independent of comericial DBMSes. They are relatively easy to learn and use and help show the semantics or meaning of the data. 
        \item \textbf{Logical Data Models} \ - \ Data is stored in relations (tables). These tables have one value per cell. Commercial relational data models include DB$_2$, Oracle, Ingress, MySQL, Microsoft Access.
    \end{itemize}
    \subsection*{ERD components}
        \item \textbf{Entities} \ - \ Principle objects about which information is kept \ - \ These are ``things'' we store data about. If you were to look at the ER Diagram like a spoken language, the entities are nouns \ - \ Person, Place, thing, event.
        \item \textbf{Relationships} \ - \ Relationships connect on ore more entities together to show an association. A relationship cannot exist without at least one associated entity. Represented as diamonds.
        \item \textbf{Attributes} \ - \ Characterists of entities or of relationships. These represent some small piece of associated data. Represented by either a rounded rectangle or an oval.
            \begin{itemize}[label=$\circ$]
                \item Attributes on Entities
            \end{itemize}




\end{document}

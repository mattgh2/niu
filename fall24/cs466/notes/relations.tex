\documentclass{report}

\input{~/latex/template/preamble.tex}
\input{~/latex/template/macros.tex}

\title{\Huge{Relational Data Model}}
\author{\huge{Matt Warner}}
\date{\huge{}}
\pagestyle{fancy}
\fancyhf{}
\rhead{CSCI 466 - Databases}
\lhead{\leftmark}
\cfoot{\thepage}

\renewcommand{\arraystretch}{1.7}
% \usepackage[default]{sourcecodepro} \usepackage[T1]{fontenc}

\pgfpagesdeclarelayout{boxed}
{
  \edef\pgfpageoptionborder{0pt}
}
{
  \pgfpagesphysicalpageoptions
  {%
    logical pages=1,%
  }
  \pgfpageslogicalpageoptions{1}
  {
    border code=\pgfsetlinewidth{1.5pt}\pgfstroke,%
    border shrink=\pgfpageoptionborder,%
    resized width=.95\pgfphysicalwidth,%
    resized height=.95\pgfphysicalheight,%
    center=\pgfpoint{.5\pgfphysicalwidth}{.5\pgfphysicalheight}%
  }%
}

\pgfpagesuselayout{boxed}

\begin{document}
    \maketitle
    \chapter{Relational Data Model}
    After designing the conceptual model of the Database using an ER diagram, we need to convert the conceptual model into a relation model which can be implemented using any RDBMS language like SQL.
    In the relational data model, our database is made up of one or more \textbf{relations} (tables), where each relation should have a \textit{unique} name. The \textbf{schema} of a relation is written as \textbf{Relation\_Name}($A_1,A_2,\ldots,A_n$). $A_1,A_2,\ldots,A_n$ are placeholders for the attribute names, which become the column headers of the table.
    \bigbreak \noindent
    When there is \textbf{instance} data, it will come in the form of \textbf{tuples} (rows), which have a value for each attribute. No field may contain more than \textit{one} value.
    \begin{figure}[H]
    \centering
    \setlength{\tabcolsep}{35}
    \begin{tabular}{|c|c|c|c|c|}
        \hline
        \multicolumn{5}{|c|}{\textbf{Relation\_Name}} \\
        \hline
        $A_1$ & $A_2$ & $A_3$ & \ldots & $A_n$ \\
        \hline
        $x_1$ & $x_2$ & $x_3$ & \ldots & $x_n$ \\
        \hline
        $y_1$ & $y_2$ & $y_3$ & \ldots & $y_n$ \\
        \hline
        \ldots & \ldots & \ldots & \ldots & \ldots \\
        \hline
    \end{tabular}
    \end{figure}
    \section{Basic Structure}
    \subsubsection*{Attributes}
    The columns of our tables represent the \textit{attributes} of the entities. Each attribute becomes a column heading, and each attribute also has an associated \textbf{domain}. The domain of an attribute is the set of all valid values for it. The domain may be looked at as a data type, but may have additional constraints.
    \subsubsection*{Tuples}
    These are the rows of our tables, we call them \textbf{tuples}. A \textbf{tuple} is a special type of (mathematical) set containing values for each attribute within the relation. Once again, the values are required to be \textit{atomic}, there can be only one value per tuple per attribute.
    \section{Degree \& Cardinality}
    If you recall, the degree of a \textbf{relationship} in an ER diagram was the number of \textit{links} it had to entities. The degree of a \textbf{relation} is the number of attributes present.
    \bigbreak \noindent
    The \textbf{cardinality} of a relation is the number of tuples present.
    \newpage
    \noindent Consider this relation as an example:  
    \begin{figure}[H]
    \centering
    \setlength{\tabcolsep}{17}
    \begin{tabular}{|c|c|c|c|c|}
        \hline
        \multicolumn{5}{|c|}{\textbf{STUDENT}} \\
        \hline
        ROLL\_NO & NAME & ADDRESS & PHONE & AGE \\
        \hline
        1 & RAM & DELHI & 945512351 & 18 \\
        \hline
        2 & RAMESH & GURGAON & 9652431543 & 18 \\
        \hline
        3 & SUJIT & ROHTAK & 554895342 & 20 \\
        \hline
        4  &SURESH&DELHI & & 18 \\
        \hline
    \end{tabular}
    \end{figure}
    \noindent Here, our attributes are \textbf{roll\_no, name, address, phone,} and \textbf{age}. Since there are 5 attributes present here, the degree of this relation is $5$.
    \bigbreak \noindent
    The cardinality of this relation is 4, since there are 4 tuples present.
    \section{Keys}
    From now on, we will be talking about various types of a thing called a \textbf{key}.
    Speaking generally, the purpose of a key is to \textit{uniquely identify} a tuple in some relation.
    \subsection*{Super Keys}
    A \textbf{super key} within a relation is an attribute or set of attributes whose values can uniquely identify any tuple within that relation. Every relation has at least one - the set of all attributes in the relation (since duplicate tuples are considered to be the same tuple). There can potentially be many super keys available in a relation, however some will be more useful than others.
    \subsection*{Candidate Keys}
    A \textbf{candidate key} is a \textit{minimal} super key - the minimum set of attributes neccessary to uniquely identify a tuple within the relation. As a side note, it is still possible for more than one candidate key to exist for a relation.
    \subsection*{Primary Key}
    The \textbf{primary key} for a relation is chosen by the database designer from among the relation's candidate keys. It becomes the ``official'' key that is used to reference tuples within the relation. There can be only one.
\bigbreak \noindent
Once a primary key is chosen, each of the attributes in the relation will be either \textbf{prime} or \textbf{non-prime} with respect to the relation.
\begin{itemize}
    \item A \textbf{prime} attribute is one of the attributes that can be found in any of the candidate keys.
    \item A \textbf{non-prime} attribute is one of the attributes \textit{not found} in any of the candidate keys.
\end{itemize}
Once a primary key is chosen for it, the \textbf{schema} of a relation is written with the primary key's attributes underlined:
$$\text{\textbf{Relation\_Name}}(A_1, A_2, A_3,\ldots, A_n)$$
\subsection*{Foreign Keys}
A \textbf{foreign key} is a tool used to link relations within a database. Since every relation has a primary key that uniquely identifies each tuple, the values of those key attributes can be used from another relation to reference individual tuples.
\bigbreak \noindent
The relation whose primary key is being used is the \textbf{home relation}.
\section{Domain}
The \textbf{domain} of an \textit{attribute} is the set of all possible values it may hold.
The \textbf{domain} of a \textit{set of attributes} is the set of all possible combinations of values for the attributes in the set.
\section{Order Independence}
In relations, the order things appear doesn't matter. There are ways to force them to sort later when we're working with SQL, but the relation itself has no order for either rows or attributes.
\bigbreak \noindent
It doesn't matter what order the attributes appear in, if two relational schemas have the same name, the same attributes, and the same primary key, then they are equivalent. \\
\textit{\textbf{So, all of these are equivalent:}}
$$ R(\underline{A},B,C,D)$$
$$ R(D,C,B,\underline{A})$$
$$ R(\underline{A},D,B,C)$$
Tuples are stored unordered. If you need to have them appear in some order later, you will be able to sort based on the values inside of them using SQL.
    \section{Constraints}
    Constraints are limits imposed on the domain of various attributes. These can come from the system your database is modeling.
    \bigbreak \noindent
    There are also a couple constaints that are \textit{unavoidable }.
    \subsection*{Entity Integrity Constraint}
    The entity integrity constraint applies to all relations. It states that no tuple may exist within a relation that has \textit{null value} for any of attributes that make up the primary key. This is a consequence of the primary key being a candidate key, which is \textbf{minimal} and cannot do its job with any less data.
    \subsection*{Referential Integrity Constraint}
    The referential integrity constraint applies to all foreign keys. It constraint applies to all foreign keys. It constrains the values of foreign keys in relations to values that actually exist as primary keys for tuples within  the home relation. If the foreign key is otherwise allowed to be \texttt{NULL}, then that is also an acceptable value.
    \section{Normalization}
    There are a large number of possible ways to represent each problem with using relations and some choices will perform better than others for various reasons. The option chosen should be the best one, but how do we know which one that is?
    \bigbreak \noindent
    To be able to choose the best option, we need to study problems that can come up and how to avoid them. We also need to study desirable properties and how to guarantee them. Here is a basic example:
    \bigbreak \noindent
    If our database is a single relation with schema \textit{\textbf{SP}} (\underline{SuppName}, SuppAddr, \underline{Item}, Price).
    \bigbreak \noindent
    \textit{\textbf{With out instance data:}}
    \begin{figure}[H]
    \centering
    \setlength{\tabcolsep}{30}
    \begin{tabular}{c c c c}
        \hline
        SuppName & SuppAddr & Item & Price \\
        \hline
        John & 10 Main & Apple & \$2.00 \\
        \hline
        John & 10 Main & Orange & \$2.50 \\
        \hline
        Jane & 20 State & Grape & \$1.25 \\
        \hline
        Jane & 20 State & Apple & \$2.25 \\
        \hline
        Frank & 30 Elm & Apple & \$6.00  \\
        \hline
    \end{tabular}
    \end{figure}
    \bigbreak \noindent
    There are some common things that we might want to do that would cause issues. We refer to these as \textbf{anomalies}, and there are split into three categories.
    \subsection*{Insertion Anomaly}
    Let's say we want to add a new vendor, ``Sally'', and store her address, ``40 Pine'', but she is not selling anything yet. Can this be inserted into the relation SP?
    \begin{figure}[ht]
    \centering
    \setlength{\tabcolsep}{39}
    \begin{tabular}{c c c c}
        \hline
        Sally & 40 Pine & ???  & ??? \\
        \hline
    \end{tabular}
    \end{figure}
    \bigbreak\noindent The answer to this question is \textbf{NO}. The \textit{primary key} is (SuppName, Item), but we only have SuppName. The \textit{entity integrity constraint} is violated if we try to insert the data as a tuple in this relation. It cannot fit. We call this an \textit{insertion anomaly}.
    \subsection*{Deletion Anomaly}
This time, let's say that Frank no longer sells Mango. We want to take that out of the database so nobody can order a mango that is not available. Our new tuple would look like this:
\begin{figure}[ht]
\centering
\setlength{\tabcolsep}{39}
\begin{tabular}{c c c c}
\hline
Frank & 30 Elm & ???  & ??? \\
          \hline
          \end{tabular}
          \end{figure}
\bigbreak \noindent
Can this tuple remain in the relation with the Mango information removed?
\bigbreak \noindent
No, it cannot. The \textit{primary key} is (SuppName, Item), and the Item is going away. The \textit{entity integrity constraint} is violated if we remove the data from the tuple in this relation. We can either keep the whole tuple, advertising fake mango, or delete the whole tuple and lose the information on Frank, which doesn't exist in any other tuples. We call this a \textit{deletion anomaly}.
\subsection*{Update Anomaly}
Next, let's say that John is moving to a different address. We would want to change it once for every item John is selling.
\begin{figure}[ht]
\centering
\setlength{\tabcolsep}{39}
\begin{tabular}{c c c c}
\hline
John & \textbf{10 Main} & Apple & \$2.00 \\
          \hline
John & \textbf{10 Main} & Apple & \$2.00 \\
\hline
          \end{tabular}
          \end{figure}
          \bigbreak \noindent
          This isn't a big deal with only two items, but as John's list of supplied items grows, so does the amount of database work that needs to be done every time he moves. If any of the SuppAddr values for John don't agree, then it may not be clear which is the right address for John. This is an \textit{update anomaly}.``
          \bigbreak \noindent
          In summary, we have:
          \begin{itemize}
              \item Insertion anomalies
                  \begin{itemize}[label=$\circ$]
                      \item When a piece of data cannot be inserted because it violates some \textit{constraint} of the relation.
                      \item Usually is the \textit{entity integrity constraint} being violated, but not always.
                  \end{itemize}
              \item Deletion anomalies
                  \begin{itemize}[label=$\circ$]
                      \item When deleting some piece of data, a \textit{deletion anomaly} is when more data is lost than intended.
                      \item Usually this is caused when the data removed is part of the \textit{primary key}, which would cause a violation of the \texit{entity integrity constraint}.
                  \end{itemize}
              \item Update anomalies
                  \begin{itemize}[label=$\circ$]
                      \item When updating a single value requires changes to multiple tuples, this is an \textit{update anomaly}.
                          \begin{itemize}[label=$\circ$]
                            \item This is caused by unnecessary redundancies in the data. 
                            \item These cause inefficiency, and potential inconsistencies.
                          \end{itemize}
                  \end{itemize}
          \end{itemize}
          \subsection{Redundancy}
          Redundancy is when values are repeated. It can be good if you have an off-site backup of your entire database. However, redundancy on the same phyiscal device is \textbf{unneccessary}. It wastes space and comes with the potential for \textit{update anomailes}.
          \bigbreak \noindent
          The good redundancy is something the database administrator or IT department should handle. When we talk about redundancy in the design of our database, we will be talking about the \textbf{bad} kind.
          \subsection{How can we fix this?}
          In the example, we were only using a single relation. There is no rule that says that a relational database must be made up of a single relation. The way we will solve these anomalies is to add new relations to our database and change the old ones. This is called \textit{decompisition}.
          \subsection{Decomposition}
          Here we represent the original data in two relations, rather than the one
          \bigbreak \noindent
          \textit{\textbf{SP}}(\underline{SuppName}, Item, Price)
         \begin{figure}[ht]
         \centering
         \setlength{\tabcolsep}{39}
         \begin{tabular}{c c c}
         \hline
         \textbf{SuppName} & \textbf{Item} &\textbf{Price} \\
                   \hline
         John & Apple & \$2.00 \\
         \hline 
         John & Orange & \$2.50 \\
         \hline 
         Jane & Grape & \$1.25 \\
         \hline 
         Jane & Apple & \$2.25 \\
         \hline
         Frank & Mango & \$6.00 \\
         \hline
                   \end{tabular}
                   \end{figure} 
            \bigbreak \noindent
            \textit{\textbf{SP}}(\underline{SuppName}, SuppAddr)
        \begin{figure}[H]
        \centering
        \setlength{\tabcolsep}{39}
        \begin{tabular}{c c}
            \hline
            \textbf{SuppName} & \textbf{SuppAddr} \\
        \hline
        John & 10 Main \\
        \hline
        Jane & 20 State \\
        \hline
        Frank & 30 Elm \\
        \hline
    \end{tabular}
\end{figure}
\bigbreak \noindent
Now we can perform the changes discussed eariler that would have caused anomalies.
\bigbreak \noindent
\textit{\textbf{SP}}(\underline{SuppName}, \underline{Item}, Price)
    \begin{figure}[ht]
    \centering
      \setlength{\tabcolsep}{28}
    \begin{tabular}{c c c}
        \hline
        SuppName & Item & Price \\
    \hline
        John & Apple & \$2.00 \\
        \hline
        John & Orange & \$2.50 \\
        \hline
        Jane & Grape & \$1.25 \\
        \hline
        Jane & Apple & \$2.25 \\
        \hline
    \end{tabular}
    \end{figure}
    \bigbreak \noindent
    \textit{\textbf{SP}}(\underline{SuppName}, SuppAddr)
        \begin{figure}[ht]
        \centering
          \setlength{\tabcolsep}{45}
        \begin{tabular}{c c}
            \hline
            SuppName & SuppAddr \\
            \hline
            John & 10 Main \\
            \hline
            Jane & 20 State \\
            \hline
            Frank & 30 Elm \\
            \hline
            Sally & 40 Pine \\
            \hline
        \end{tabular}
        \end{figure}
        \subsubsection{Pros and Cons of Decomposition}
        \begin{itemize}
            \item Pros:
                \begin{itemize}[label=$\circ$]
                    \item Got rid of our anomalies, so all of our data is able to be stored, with less potential for inconsistencies.
                \end{itemize}
            \item Cons:
                \begin{itemize}[label=$\circ$]
                    \item It takes a little longer to run some queries when multiple relations' data is involved simultaneously (in queries that involve a \textit{join}).
                \end{itemize}
        \end{itemize}
        \subsubsection{When to use decomposition}
        One way of designing a database could be to list all of the possible anomalies and then decompose to fix each of them. The problem with this is that any anomalies you don't see coming will not be fixed. 
        \bigbreak \noindent
        We will look at a systematic method of identifying the potential for anomalies. This method is called \textit{normalization}.
    \subsection{Normalization}
    \textit{Normalization} involves making sure that each of your relations follows certain rules. Depending on which rules are followed, each of the relations in your database will be in one or more \textit{normal forms}. These rules are based on \textit{functional dependencies}.
    \subsubsection{Functional Dependencies}
    A \textit{functional dependency} is a statement about which attributes can be inferred from other attributes. If we take $X$ and $Y$ as \textit{sets} of attributes, we can write:
    $$ X \rightarrow Y$$
    If, whenever unique values for \textbf{all} of the attributes in $X$ are known, unique values for \textbf{each} of the attributes of $Y$ are guaranteed to be possible to look up or to infer using those values.
    This is read either as:
    $$ \text{$X$ functionally determines $Y$}, \ or $$
    $$ Y \text{ is functionally dependent upon } X$$
    They are statements about the operational data. Later on, we will see how to read them off of ER diagrams, though they may come from elsewhere as well.
    \begin{mdframed}
        \vspace{-4mm}\subsubsection*{Real-life Examples}
    $$ ZID \rightarrow \text{StudentFirstName, StudentLastName, Birthday}$$
    If I identify a student using their ZID, that student has \textit{one} first name, last name, and birthday.
    $$ \text{StudentFirstName} \nrightarrow  \text{ZID} $$
    The first name is not enough to determine a single ZID, as there are multiple students with the same first name.
    $$ \text{ZID, CourseID, Semester} \rightarrow \text{Grade} $$
    If i know which student, which course, and which semester, I can find a single grade.
\end{mdframed}
\bigbreak \noindent
Keep in mind that \textit{functional dependencies} are constraints present within the operational data your database models. They don't necessarily describe how things work in the real world, but they do have to accurately describe any data you will store in your database. 
\bigbreak \noindent
Additionally, \textit{functional dependencies} \textbf{must} hold for all possible data values. Attempts to add data that does not obey the functional dependencies will result in anomalies. 
\bigbreak \noindent
Furthermore, functional dependencies \textbf{can} be enforced during insertion if the database is set up properly.
\subsection{Armstrong's Axioms}
\textit{Armstrong's Axioms} are a set of rules for operations that are permissible when manipulating \textit{functional dependencies}. I am providing them here as a reference. I will explain them because they can be useful tools, but they will not be on a quiz or test in any formal way.
\subsubsection*{Primary Rules:}
\textit{\textbf{Axiom of reflexivity:}}
$$ \text{If } Y \subseteq X \text{ then } X \rightarrow Y$$
\textit{\textbf{Axiom of augmentation:}}
$$ \text{If } \rightarrow Y, \text{ then } XZ \rightarrow YZ \text{ for any } Z $$
\textit{\textbf{Axiom of transitivity:}}
$$ \text{If } X \rightarrow Y \text{ and } Y \rightarrow Z, \text{ then } X \rightarrow Z$$
\subsubsection*{Secondary Rules}
\textit{\textbf{Decomposition:}}
$$ \text{If } X \rightarrow YZ \text{ then } X \rightarrow Y \text{ and } X \rightarrow Z $$
\textit{\textbf{Composition:}}
$$ \text{If } X \rightarrow Y \text{ and } A \rightarrow B \text{ then } X\A \rightarrow YB$$
\textit{\textbf{Union}}\textbf{ (Notation):}
$$ \text{If } X \rightarrow Y \text{ and } Y \rightarrow Z \text{ then } X \rightarrow YZ $$
\textit{\textbf{Pseudo-transitivity}:}
$$ \text{If } X \rightarrow Y \text{ and } YZ \rightarrow W \text{ then } XZ \rightarrow W $$
\textit{\textbf{Self-determination}:}
$$ I \rightarrow I \text{ for any } I $$
\subsection{Revisiting Keys}
When we talked about \textit{keys}, we talked about how their purpose is to uniquely identify a puple within a relation. Another way of stating this, now that we know about \textit{functional dependencies}, is \textbf{the attributes of a \textbf{superkey} must functionally determine \textit{all} of the attributes of the relation.}
\bigbreak \noindent
\textit{Candidate keys} and \textit{primary keys} \textbf{are} \textit{super keys}, so this is true of them as well, and they also satisfy additional requirements. As an example, say we have the relation \textit{\textbf{R}}(\underline{a},b,c,d,e,f)
$$ a \rightarrow a,b,c,d,e,f$$
But, since it is always the case that $a \rightarrow a$ because of the self-determination axiom, we usually omit the left hand side from the righthand side. So we would usually write this instead:
$$ a \rightarrow b,c,d,e,f$$
\subsection*{Example: Relation with FDs}
Lets say we have the following relation:
$$ \text{\textbf{EmpProj}(\underline{EmpID},\underline{Project}, Supv, Dept, Case)} $$
    \begin{figure}[ht]
    \centering
    \setlength{\tabcolsep}{30}
    \begin{tabular}{c c c c c}
    \hline 
    EmpID & Project & Supv & Dept & Case \\
    \hline
    e1 & p1 & s1 & d1 & c1 \\
    \hline
    e2 & p2 & s2 & d2 & c2 \\
    \hline
    e1 & p3 & s1 & d1 & c3 \\
    \hline
    e3 & p3 & s1 & d1 & c3 \\
    \hline
    \end{tabular}
    \end{figure}
    \bigbreak \noindent
Our functional dependencies are:
\begin{itemize}
    \item EmpID, Project $\rightarrow$ Supv, Dept, Case
    \item  EmpID $\rightarrow$ Supv, Dept
    \item Supv $\rightarrow$ Dept
\end{itemize}
As written, there are some anomalies present. We will use \textit{normalization} to move toward a better design.
\subsection{First Normal Form (1NF)}
You should recall from the introduction to relations that all of the values in the tuple with a relation must be \textit{atomic}. This means that there is a maximum of \textit{one} value per attribute per tuple.
\bigbreak \noindent
The requirement for a relation to be in First Normal Form (1NF) is this same requirement that all of the values must be atomic.
\bigbreak \noindent
What is usually looks like is a table with multiple values in a single cell. A non-1NF relation would not even technically count as a relation. This table has a cell (the one with (z1, z2, z2)) that is non-atomic, so it would not be in 1Nf.
    \begin{figure}[ht]
    \centering
     \setlength{\tabcolsep}{35}
    \begin{tabular}{ c c c}
        \hline
        X & Y & Z \\
    \hline 
    x1 & y1 & z1 \\
       & & z2 \\
       & & z3 \\
    x2 & y2 & z4 \\
    x3 &y3 & z5 \\
    \hline
    \end{tabular}
    \end{figure}
\bigbreak \noindent
It looks like $X$ \textit{would} have been the \textit{primary key}, but it's not doing its job of uniquely determining $Z$, which is showing as a \textit{repeating group} so $X$ can't be a key. The notation for a ``pseudo-relation'' like the one above would be to use inner parenthesis on the repeating group, i.e. \textit{\textbf{R}}(\underline{X}, Y,(Z)).
\bigbreak \noindent
This pseudo-relation is not in 1NF, but has functional dependencies:
$$ X \rightarrow Y $$
$$ X,Z \rightarrow Z \text{\textbf{ but }} X \nrightarrow Z$$
To move this pseudo-relation into an acutal relation that doesn't violate 1NF, we need to choose a \textit{real} primary key that meets the requirements. We do that using the FDs. In this case, ($X,Z$) works.
\bigbreak \noindent
Chaing the primary key yields: \textit{\textbf{R}}(\underline{X},Y,\underline{Z})
    \begin{figure}[ht]
    \centering
      \setlength{\tabcolsep}{35}
    \begin{tabular}{c c c}
    \hline
    X & Y & Z \\
    \hline
    x1&y1&z1 \\
    x1& y1&z2 \\
    x1&y1&z3 \\
    x2&y2&z4 \\
    x3&y2&z5 \\
    \hline
    \end{tabular}
    \end{figure}
Now everything is atmoic, and we are in 1NF. Notice that this did introduce a new update anomaly, but the other normal forms will take care of it. It is more important to get into 1NF for now.




\end{document} 

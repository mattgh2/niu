\documentclass{report}

\input{~/latex/template/preamble.tex}
\input{~/latex/template/macros.tex}

\title{\Huge{Midterm Review}}
\author{\huge{Matt Warner}}
\date{\huge{}}
\pagestyle{fancy}
\fancyhf{}
\rhead{}
\lhead{\leftmark}
\cfoot{\thepage}
% \usepackage[default]{sourcecodepro} \usepackage[T1]{fontenc}

\pgfpagesdeclarelayout{boxed}
{
  \edef\pgfpageoptionborder{0pt}
}
{
  \pgfpagesphysicalpageoptions
  {%
    logical pages=1,%
  }
  \pgfpageslogicalpageoptions{1}
  {
    border code=\pgfsetlinewidth{1.5pt}\pgfstroke,%
    border shrink=\pgfpageoptionborder,%
    resized width=.95\pgfphysicalwidth,%
    resized height=.95\pgfphysicalheight,%
    center=\pgfpoint{.5\pgfphysicalwidth}{.5\pgfphysicalheight}%
  }%
}

\pgfpagesuselayout{boxed}

\begin{document}
    \maketitle
one sided 2 sided parkets
Know the difference between one sided and two sided markets. 
\bigbreak \noindent
Know the difference between owned, paid, and earned media, and ibound marketing
\bigbreak \noindent
Netflix offered 1 million dollars to anyone who can improve their engine by 10 percent \ - \
\bigbreak \noindent
\subsection*{Key terms}
Atoms to bits \\
switching costs.
Same side and cross side (in social media) - same side is one group of people that grows with more people \\
cross side \ - \
\section{ptest}
Which is not one of the four critical characteristics - resource based view of competitive advantage - scalable
\chapter*{Chapter 1}
\begin{itemize}
    \item Why study IS?
        \begin{itemize}[label=$\circ$]
            \item Business models and strategies have rapidly changed based on new and evolving technology.
        \end{itemize}
    \item Hype Cycle
        \begin{itemize}[label=$\circ$]
            \item Understand and describe parts of hype cycle.
                \begin{itemize}[label=$\circ$]
                    \item (1) Emerging innovation 
                        \begin{itemize}[label=$\circ$]
                            \item A potential technology breakthrough kicks things off. Early proof-of-concept stories and media interest trigger significant publicity. Often no usable products exist and commercial viability is unproven.
                        \end{itemize}
                    \item (2) Peak of Inflated expectations
                        \begin{itemize}[label=$\circ$]
                            \item  Early publicity produces a number of success stories - often accompanied by scores of failures. Some companines take action; most do not.
                        \end{itemize}
                    \item (3) Trough of disilusionment
                        \begin{itemize}[label=$\circ$]
                            \item Interest wanes as experiments and implementations fail to deliver. Producers of the technology shake out or fail. Investment cointinues only if the surviving providers improve their products to the statisfaction of early adopters.
                        \end{itemize}
                    \item (4) Slope of enlightenment
                        \begin{itemize}[label=$\circ$]
                            \item More instances of the technology's benefits start to crystallize and become more widely understood. Second - and third generation products appear from technology providers. More enterprises fund pilots; conservative companies remain cautious.
                        \end{itemize}
                    \item (5) Plateau of productivity
                        \begin{itemize}[label=$\circ$]
                            \item Mainstream adoption starts to take off. Criteria for assessing provider viability are more clearly defined. The technology's broad market applicability and relevance are clearly paying off. IF the technology has more than a niche market then it will continue to grow.
                        \end{itemize}
                \end{itemize}
            \item Impact of AI (High level)
        \end{itemize}
\end{itemize}
\chapter*{Chapter 3}
Understand and apply to examples
\begin{itemize}
    \item Fast follower problem 
        \begin{itemize}[label=$\circ$]
            \item When a company may not be the first to market with an innovation but seeks to benefit from observing and adapting what others have successfully done. In other words, its an organization that waits for a competitor to successfully innovate before imitating it with a similar product.
            \item The Fast follower strategy relies on a company releasing an imitation product in rapid time to secure market share before the competition.
            \item Tech can be copied quickly, so followers can indeed by fast.
            \item Snapchat pioneered many of the photo and video sharing features such as ``Stories'' and augmented reality filters, but Facebook properties routinely mimic Snap features, implementing some in as little as four months. Snapchats growth tumbled 82 percent after Instagram Stories launched, and the firm posted a \$2.2 billion loss in its first quarter as a public company.
                

        \end{itemize}
    \item Four critical characteristics - resource based view of competitive advantage.
        \begin{itemize}[label=$\circ$]
            \item The strategic thinking approach suggesting that if a firm is to maintain sustainable competitve advantage, it must control an exploitable resource, or set of resources that have four critical characteristics:
        \begin{itemize}[label=$\circ$]
            \item Valuable
            \item Rare
            \item Imperfectly imitable
            \item Nonsubstitable
        \end{itemize}
    \item Telecom firms began digging up the ground and laying webs of fiberglass to meet growing demands, but problems resulted as rivals were doing the exact same thing.
    \item A technology called dense wave division multiplexing (DWDM) enabled existing fiber to carry more transmissions than ever before.
\begin{itemize}[label=$\circ$]
    \item The end result - the new assests weren't rare, and each day they seemed to be less valuable.
\end{itemize}
        \end{itemize} 
    \item Five primary components of the value chain.
        \begin{itemize}[label=$\circ$]
            \item Inbound logistics
            \item Operations
            \item Outbound logistics
            \item Marketing and sales
            \item Support
        \end{itemize}
    \item Porters 5 forces of competitive analysis.
        \begin{itemize}[label=$\circ$]
            \item Competitive Rivalry
            \item Supplier Power
            \item Buyer Power
            \item Threat of substitution
            \item Threat of New Entrants
        \end{itemize}
\end{itemize}
\chapter*{Chapter 7}
\begin{itemize}
    \item Why do big firms fail?
        \begin{itemize}
            \item Failure to see disruptive innovations as a threat.
            \begin{itemize}[label=$\circ$]
                \item Resources aren't dedicated to developing the potential technology
                \item Firms don't nuture the needs of a new customer base.
            \end{itemize}
        \item Early customers for a disruptive techonolgy care about different features and attributes than incumbent customers.
            \begin{itemize}[label=$\circ$]
                \item A free call over a quality call.
                \item Portable music over high-fidelity music
            \end{itemize}
        \end{itemize}
    \item Characteristics of Giant Killers?
    \item How are large firms combating challenges?
    \item McNamara Fallacy
        \begin{itemize}[label=$\circ$]
            \item Basing decisions based on past data and examples. This falls under four steps:
                \begin{itemize}[label=$\circ$]
                    \item Measure what can be easily measured: Focusing on numbers that are readily available.
                    \item Disregard what can't be easily measured: Ignoring imporant aspects that are harder to quantify.
                    \item Assume what can't be easily measured is not important: Devaluing qualitative elements.
                    \item Assume what can't be measured doesn't exist: Completely dismissing factors that are difficult to quantify.
                \end{itemize}

        \end{itemize}
    \item Apply concepts to how Amazon and Netflix grew and continue to grow (chapter 5 & 8)
\end{itemize}
\chapter*{Chapter 10}
\begin{itemize}
    \item Impact and characteristics of network effects
        \begin{itemize}[label=$\circ$]
            \item When the value of a product or service increases as its number of users expands.
                \begin{itemize}[label=$\circ$]
                    \item Also referred to as ``network externalities'' or ``Metcalfe's Law.''
                    \item Most products are not subject to network effects.
                    \item Presence of network effects are among the most important reasons you'll pick one product or service over another.
                \end{itemize}
        \end{itemize}
    \item One sided vs two sided markets
        \begin{itemize}[label=$\circ$]
            \item \textbf{One sided market} \ - \ Market that derives most of its value from a single class of users.
                \begin{itemize}[label=$\circ$]
                    \item Example of this kind of network is messaging. Value-creating, positive-feedback loop of network effects come mostly from a single group.
                    \item \textbf{same side exchange benefits}: Benefits derived by interaction among members of a single class of participant.
                \end{itemize}
            \item \textbf{Two-sided market} \ - \ Network markets comprised of two distint categories of participant, both of which are needed to deliver value for the network to function.
                \begin{itemize}[label=$\circ$]
                    \item An example of this kind of network is video games. Also mobile payment: the more people who use a given payment platform, the more attractive that platform will be to storefronts and other businesses. If more businesses accept these forms of mobile payment, then this in turn should attract more customers (and so on).
                    \item \textbf{cross-side exchange benefit}: An increase in the number of users on one side of the market, creating a rise in the other side.
                \end{itemize}
            \item \textbf{A network may have both same-side and cross-side benefits - Xbox:}
                \begin{itemize}[label=$\circ$]
                    \item Cross-side benefits - more users of that console attract more developers writing more software titles and vice versa.
                    \item Same-side benefits - Xbox Live network allows users to play against each other.
                \end{itemize}
        \end{itemize}
    \item Recognize when network effects are present
\end{itemize}
\chapter*{Chapter 11}
\begin{itemize}
    \item Understand and identify Owned, Paid, and Earned Media, and Inbound Marketing
        \begin{itemize}[label=$\circ$]
            \item \textbf{Owned}: Communication channels that an organization controls. These can include firm-run blogs and websites, any firm-distributed corporate mobile website or app, and organization accounts on social media such as Twitter, Facebook, Pinterest, YouTube, and Instagram. (visit the starbucks website? That's media owned by starbucks)
            \item \textbf{Paid}: Refers to efforts where an organization pays to leverage a channel or promote a message. Paid media efforts include things such as advertisement and sponsorships. (See a starbucks ad online? Thats paid media).
            \item \textbf{Earned}: Promotions that are not paid for or owned but rather grow organically from customer efforts or other favorable publicity. Social media can be a key driver of earned media (think positive tweets on tiktok or instagram, referring facebook posts, and pins on pinterest)
            \item \textbf{inbound marketing}: Refers to leveraging online channels to draw consumers to the firm with compelling content rather than conventional forms of promotion such as advertising, e-mail marketing, traditional mailings, and sales calls.
        \end{itemize}
    \item Impact of Social media in business
        \begin{itemize}
            \item Enhanced Brand Visibilty
            \item Direct Customer Engagement
            \item Cost-Effective Marketing
            \item Data-Driven Insights
            \item Influencer Marketing
            \item User-Generated Content
            \item Global Reach and Accessibility
            \item Rapid Product Feedback and market Research
            \item Sales and Lead Generations
            \item Crisis Management and Reputation Building
        \end{itemize}	
    \item Crowdsourcing
        \begin{itemize}[label=$\circ$]
            \item The act of taking a job traditionally performed by a designated agent and outsourcing it to an undefined generally large group of people in the form of an open call.
            \item Example: Goldcorp offered up all their data and offered prize money for the best ideas. In a few years, the firm grew into a \$9 billion titan.
            \item Example: Waze used crowdsourcing to build a better map. Bought by google for \$1 billion.
        \end{itemize}
    \item Understand why companies are building Social media team
\end{itemize}
\chapter*{12}
\begin{itemize}
    \item Understand what a sharing economy is
        \begin{itemize}[label=$\circ$]
            \item A sharing economy is essentially a peer-to-peer model of business that allows consumers to share in the creation or use of products, and services. Often taking place across digital platforms, thus it is very much facilitated by technology.
            \item Product owners become providers of rentals.
                \begin{itemize}[label=$\circ$]
                    \item Rooms (Airbnb)
                    \item Cars (turo)
                    \item Boats (Boatsetter, GetMyBoat)
                \end{itemize}
            \item New class of micro-entrepreneurs providing personal services
                \begin{itemize}[label=$\circ$]
                    \item Car rides (Uber, Lyft)
                    \item Pet sitting (Rover, Wag!)
                    \item Meal prep (Feastly)
                    \item Home services (Care.com, Angies list, Handy)
                \end{itemize}
        \end{itemize}
    \item Understand risk and reward in a sharing economy to:
        \begin{itemize}[label=$\circ$]
            \item Suppliers
                \begin{itemize}
                    \item \textbf{Benefits}
                        \begin{itemize}[label=$\circ$]
                            \item Flexibilty
                            \item Income
                            \item Low barrier to entry
                        \end{itemize}
                         \item \textbf{Risks}
                             \begin{itemize}[label=$\circ$]
                                 \item Job security
                                 \item Income not steady
                                 \item Liabilty/upkeep
                                 \item Local laws
                             \end{itemize}
                \end{itemize}
            \item Companies
                \begin{itemize}[label=$\circ$]
                    \item \textbf{Benefits}
                        \begin{itemize}[label=$\circ$]
                            \item Scalability
                            \item Cost efficiency
                            \item Global Reach
                            \item Flexibility
                        \end{itemize}
                    \item \textbf{Risks}
                        \begin{itemize}[label=$\circ$]
                         \item Regulatory Challenges
                         \item Reputation Management
                            \item Worker Retention
                            \item Data security
                        \end{itemize}
                \end{itemize}
            \item Customer
                \begin{itemize}[label=$\circ$]
                    \item \textbf{Benefits}
                    \begin{itemize}[label=$\circ$]
                        \item Convenience
                        \item Variety
                        \item Lower costs
                    \end{itemize}
                \item \textbf{Risks}
                    \begin{itemize}[label=$\circ$]
                        \item Saftey
                        \item Data Privacy
                        \item Inconsistent quality
                    \end{itemize}
                \end{itemize}
            \item Investors
                \begin{itemize}[label=$\circ$]
                    \item \textbf{Benefits}
                        \begin{itemize}[label=$\circ$]
                            \item High upside
                            \item Low capitol
                            \item Disruptive
                        \end{itemize}
                    \item \textbf{Risks}
                        \begin{itemize}[label=$\circ$]
                            \item Market saturation
                            \item Regulatory issues
                            \item Dependence on trust
                        \end{itemize}
                \end{itemize}
        \end{itemize}
    \item What helped drive the sharing economy?
    \begin{itemize}[label=$\circ$]
        \item Cloud Computing
            \begin{itemize}
                \item On-demand scaling
                \item Global operations/reach
                \item Data management for millions of users
                \item On-demand cost - pay as you go
            \end{itemize}
        \item Mobile devices
        \item ML and analytics
        \item Dynamic pricing
        \item Personal experience
        \item Cashless
        \item Real time information
    \end{itemize}
\end{itemize} 
\chapter*{Chapter 13}
\begin{itemize}
    \item What is a social graph?
        \begin{itemize}[label=$\circ$]
            \item A social graph represents the relationships between people (and entities) in a network. It is a foundational concept in social media, business, and data analysis.
            \item Nodes represent individuals or entities in a network.
            \item Edges are the lines between nodes that represent relationships, such as friendships or transactions.
        \end{itemize}
    \item How does it differ from interest graph?
        \begin{itemize}[label=$\circ$]
            \item A interest graph focuses on mapping connections based on shared interests or behaviors, regardless of personal relationships.
            \item Nodes represent users, topics, or pieces of content (e.g., articles, videos, posts) that users interact with or follow .
            \item Edges Represent the relationship between users and the interests or content they engage with.
        \end{itemize}
    \item Key features of a social graph?
\end{itemize}
\chapter*{It overview}
\begin{itemize}
    \item Shadow IT
        \begin{itemize}[label=$\circ$]
        \item When a business team bypasses IT, creating security risks.
        \item Unauthorized tools and applications used by business teams.
        \item IT loses control over security and system integration.
        \end{itemize}
    \item Identify differences in approach between IT and other business functions.
        \begin{itemize}[label=$\circ$]
            \item It focuses on routine tasks (maintenance, security). 
            \item It: Long-term system stability, security, scalability.
            \item Business: Fast time-to-market, short-term gains.
        \end{itemize}
    \item Match key functions of IT with the functions they perform.
        \begin{itemize}[label=$\circ$]
            \item \textbf{Managing and Storing Data across Systems}
            \item \textbf{Cybersecurity and Risk Management}
            \item \textbf{Software Development and Maintenance}
            \item \textbf{Networking and Connectivity}
            \item \textbf{User Support and Helpdesk}
            \item \textbf{IT Governance and Compliance}
            \item \textbf{Project Management and IT Alignment}
            \item \textbf{Vendor and Third-Party Management}
        \end{itemize}
\end{itemize}
\chapter*{Speed of trust}
\begin{itemize}
    \item 4 areas of trust and why they are important
        \begin{itemize}[label=$\circ$]
            \item Integrity: Acting with honesty and congruence.
            \item Inten: Aligning motives and behavior
            \item Capabilities: Having the skills and knowledge
            \item Results: Achieving outcomes and keeping promises.
        \end{itemize}
    \item 5 topics built on pyramid (5 dysfunctions of a team) impact of doing well and not doing well.
        \begin{itemize}[label=$\circ$]
            \item  Lack of Results
            \item Avoidance of Accountability
            \item Lack of Commitment
            \item Fear of Conflic Conflict
            \item Absence of Trust
        \end{itemize}
    \item Impact on IT and Business.
\end{itemize}
\end{document}
